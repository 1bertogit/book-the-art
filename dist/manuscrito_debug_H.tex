% Options for packages loaded elsewhere
\PassOptionsToPackage{unicode}{hyperref}
\PassOptionsToPackage{hyphens}{url}
\PassOptionsToPackage{dvipsnames,svgnames,x11names}{xcolor}
\documentclass[
  brazilian,
  11pt,a4paper,twoside,openright]{book}
\usepackage{xcolor}
\usepackage[top=2.5cm,bottom=2.5cm,inner=3cm,outer=2.5cm]{geometry}
\usepackage{amsmath,amssymb}
\setcounter{secnumdepth}{5}
\usepackage{iftex}
\ifPDFTeX
  \usepackage[T1]{fontenc}
  \usepackage[utf8]{inputenc}
  \usepackage{textcomp} % provide euro and other symbols
\else % if luatex or xetex
  \usepackage{unicode-math} % this also loads fontspec
  \defaultfontfeatures{Scale=MatchLowercase}
  \defaultfontfeatures[\rmfamily]{Ligatures=TeX,Scale=1}
\fi
\usepackage{lmodern}
\ifPDFTeX\else
  % xetex/luatex font selection
  \setmainfont[]{Charter}
  \setsansfont[]{Helvetica Neue}
  \setmonofont[]{Menlo}
\fi
% Use upquote if available, for straight quotes in verbatim environments
\IfFileExists{upquote.sty}{\usepackage{upquote}}{}
\IfFileExists{microtype.sty}{% use microtype if available
  \usepackage[]{microtype}
  \UseMicrotypeSet[protrusion]{basicmath} % disable protrusion for tt fonts
}{}
\usepackage{setspace}
\makeatletter
\@ifundefined{KOMAClassName}{% if non-KOMA class
  \IfFileExists{parskip.sty}{%
    \usepackage{parskip}
  }{% else
    \setlength{\parindent}{0pt}
    \setlength{\parskip}{6pt plus 2pt minus 1pt}}
}{% if KOMA class
  \KOMAoptions{parskip=half}}
\makeatother
\usepackage{graphicx}
\makeatletter
\newsavebox\pandoc@box
\newcommand*\pandocbounded[1]{% scales image to fit in text height/width
  \sbox\pandoc@box{#1}%
  \Gscale@div\@tempa{\textheight}{\dimexpr\ht\pandoc@box+\dp\pandoc@box\relax}%
  \Gscale@div\@tempb{\linewidth}{\wd\pandoc@box}%
  \ifdim\@tempb\p@<\@tempa\p@\let\@tempa\@tempb\fi% select the smaller of both
  \ifdim\@tempa\p@<\p@\scalebox{\@tempa}{\usebox\pandoc@box}%
  \else\usebox{\pandoc@box}%
  \fi%
}
% Set default figure placement to htbp
\def\fps@figure{htbp}
\makeatother
\ifLuaTeX
\usepackage[bidi=basic,shorthands=off]{babel}
\else
\usepackage[bidi=default,shorthands=off]{babel}
\fi
\ifPDFTeX
\else
\babelfont{rm}[]{Charter}
\fi
\ifLuaTeX
  \usepackage{selnolig} % disable illegal ligatures
\fi
\setlength{\emergencystretch}{3em} % prevent overfull lines
\providecommand{\tightlist}{%
  \setlength{\itemsep}{0pt}\setlength{\parskip}{0pt}}
% ═══════════════════════════════════════════════════════════════════════════
% Header Includes — The Art of Eyelid Surgery (Estilo Didático)
% Compatível com Pandoc 3.8+
% ═══════════════════════════════════════════════════════════════════════════

% Compatibilidade Pandoc 3.8+
\providecommand{\NewStructureName}[2]{}
\providecommand{\SetStructureName}[2]{}
\providecommand{\AssignStructureRole}[2]{}

% CORES — PALETA MÉDICA DIDÁTICA
\usepackage{xcolor}
\definecolor{medblue}{RGB}{0,82,147}
\definecolor{darkblue}{RGB}{0,51,102}
\definecolor{lightblue}{RGB}{230,240,250}
\definecolor{chaptergray}{RGB}{80,80,80}
\definecolor{footergray}{RGB}{120,120,120}

% BOXES CLÍNICOS
\usepackage{tcolorbox}
\tcbuselibrary{skins,breakable}

% Box de nota clínica (blockquotes)
\newtcolorbox{notebox}{
  enhanced,
  breakable,
  colback=lightblue,
  colframe=medblue,
  boxrule=1pt,
  arc=2pt,
  left=10pt,
  right=10pt,
  top=8pt,
  bottom=8pt
}

% Converter blockquotes para notebox
\usepackage{etoolbox}
\AtBeginEnvironment{quote}{\begin{notebox}}
\AtEndEnvironment{quote}{\end{notebox}}

% CAPÍTULOS — ABERTURA EDITORIAL
\usepackage{titlesec}

\titleformat{\chapter}[display]
  {\normalfont\huge\bfseries\sffamily}
  {\flushright\textcolor{chaptergray}{\fontsize{72}{72}\selectfont\thechapter}}
  {20pt}
  {\Huge\raggedright\textcolor{darkblue}}
  [\vspace{1cm}\titlerule[2pt]]

\titlespacing*{\chapter}{0pt}{-30pt}{40pt}

% Seções
\titleformat{\section}
  {\normalfont\Large\bfseries\sffamily\color{medblue}}
  {\thesection}
  {1em}
  {}

\titleformat{\subsection}
  {\normalfont\large\bfseries\sffamily\color{darkblue}}
  {\thesubsection}
  {1em}
  {}

% HEADERS E FOOTERS
\usepackage{fancyhdr}

\pagestyle{fancy}
\fancyhf{}
\fancyhead[LE]{\small\sffamily\textcolor{footergray}{\leftmark}}
\fancyhead[RO]{\small\sffamily\textcolor{footergray}{\rightmark}}
\fancyfoot[LE,RO]{\sffamily\textbf{\thepage}}
\fancyfoot[C]{\small\sffamily\textcolor{footergray}{The Art of Eyelid Surgery}}

\renewcommand{\headrulewidth}{0.5pt}
\renewcommand{\headrule}{\hbox to\headwidth{\color{medblue}\leaders\hrule height \headrulewidth\hfill}}
\renewcommand{\footrulewidth}{0pt}

\fancypagestyle{plain}{
  \fancyhf{}
  \fancyfoot[C]{\sffamily\textbf{\thepage}}
  \renewcommand{\headrulewidth}{0pt}
}

% SUMÁRIO
\usepackage{tocloft}
\setlength{\cftbeforechapskip}{0.8em}
\renewcommand{\cftchapfont}{\sffamily\bfseries\color{darkblue}}
\renewcommand{\cftchappagefont}{\sffamily\bfseries}
\renewcommand{\cftsecfont}{\sffamily}
\renewcommand{\cftsecpagefont}{\sffamily}

% LISTAS
\usepackage{enumitem}
\setlist{nosep, leftmargin=*, itemsep=3pt}
\setlist[itemize,1]{label=\textcolor{medblue}{\textbullet}}
\setlist[itemize,2]{label=\textcolor{medblue}{--}}

% LEGENDAS
\usepackage[
  font={small,sf},
  labelfont={bf,color=medblue},
  labelsep=period,
  justification=justified,
  singlelinecheck=false,
  skip=8pt
]{caption}
\usepackage{bookmark}
\IfFileExists{xurl.sty}{\usepackage{xurl}}{} % add URL line breaks if available
\urlstyle{same}
\hypersetup{
  pdftitle={The Art of Eyelid Surgery},
  pdfauthor={Dr.~Marcelo Cury},
  pdflang={pt-BR},
  colorlinks=true,
  linkcolor={NavyBlue},
  filecolor={Maroon},
  citecolor={Blue},
  urlcolor={NavyBlue},
  pdfcreator={LaTeX via pandoc}}

\title{The Art of Eyelid Surgery}
\usepackage{etoolbox}
\makeatletter
\providecommand{\subtitle}[1]{% add subtitle to \maketitle
  \apptocmd{\@title}{\par {\large #1 \par}}{}{}
}
\makeatother
\subtitle{Cirurgia Palpebral e Periorbitária}
\author{Dr.~Marcelo Cury}
\date{2026-01-08}

\begin{document}
\frontmatter
\maketitle

\renewcommand*\contentsname{Sumário}
{
\hypersetup{linkcolor=NavyBlue}
\setcounter{tocdepth}{2}
\tableofcontents
}
\setstretch{1.15}
\mainmatter
\chapter{The Art of Eyelid Surgery}\label{the-art-of-eyelid-surgery}

\section{Do Diagnóstico Preciso ao Rejuvenescimento do
Olhar}\label{do-diagnuxf3stico-preciso-ao-rejuvenescimento-do-olhar}

\textbf{Dr.~Marcelo Cury, MD}

Cirurgião Plástico --- Especialista em Cirurgia Palpebral e
Periorbitária {[}Rio de Janeiro / Brasil{]}

\textbf{1ª Edição --- 2026}

\begin{center}\rule{0.5\linewidth}{0.5pt}\end{center}

\section{Direitos Autorais}\label{direitos-autorais}

© 2026 Dr.~Marcelo Cury --- Todos os direitos reservados.

Nenhuma parte deste livro pode ser reproduzida ou transmitida, no todo
ou em parte, por qualquer meio, sem autorização por escrito do autor.

\begin{center}\rule{0.5\linewidth}{0.5pt}\end{center}

\section{Nota Importante (Uso
Educacional)}\label{nota-importante-uso-educacional}

Este material destina-se exclusivamente a profissionais médicos e
estudantes de medicina, para fins educacionais. Não substitui
treinamento prático supervisionado, julgamento clínico e protocolos
institucionais.

\begin{center}\rule{0.5\linewidth}{0.5pt}\end{center}

\section{Prefácio}\label{prefuxe1cio}

Este livro condensa a essência técnica e o rigor analítico do curso
\textbf{The Art of Eyelid Surgery}. Mais do que um compilado de
técnicas, esta obra sistematiza um raciocínio clínico onde a precisão de
um milímetro define a fronteira entre o rejuvenescimento natural e o
estigma cirúrgico.

Nosso foco é oferecer ao especialista segurança, previsibilidade e um
método replicável para transformar diagnósticos complexos em resultados
de excelência.

\begin{center}\rule{0.5\linewidth}{0.5pt}\end{center}

\section{Sobre o Autor}\label{sobre-o-autor}

\textbf{Dr.~Marcelo Cury} é cirurgião plástico com atuação dedicada à
cirurgia palpebral e rejuvenescimento periorbitário. É criador do curso
online \textbf{The Art of Eyelid Surgery}, no qual sistematiza sua
abordagem cirúrgica baseada em anatomia aplicada, tomada de decisão e
manejo de riscos.

\textbf{Formação}

Graduado em Medicina com Especialização em Cirurgia Plástica pela 38ª
Turma da Escola Ivo Pitanguy (Pontifícia Universidade Católica do Rio de
Janeiro).

\textbf{Títulos e Qualificações}

Detentor do título de especialista em Cirurgia Plástica pela Sociedade
Brasileira de Cirurgia Plástica (SBCP) e pela Associação Médica
Brasileira (AMB).

\textbf{Afiliações Profissionais} - Membro Titular da Sociedade
Brasileira de Cirurgia Plástica (SBCP)

\begin{itemize}
\item
  Membro da International Society of Aesthetic Plastic Surgery (ISAPS)
\item
  Preceptor do Serviço de Cirurgia Plástica do Hospital Federal dos
  Servidores do Estado (Rio de Janeiro)
\end{itemize}

\textbf{Localização}

Rio de Janeiro, Brasil

\begin{center}\rule{0.5\linewidth}{0.5pt}\end{center}

\section{Nota de Origem do
Conteúdo}\label{nota-de-origem-do-conteuxfado}

Todo o conteúdo deste livro foi adaptado e editado a partir do curso
online \textbf{The Art of Eyelid Surgery}, de autoria do Dr.~Marcelo
Cury. A organização em capítulos, a padronização de referências e a
estrutura editorial foram desenvolvidas para facilitar consulta, revisão
e aplicação clínica.

\begin{center}\rule{0.5\linewidth}{0.5pt}\end{center}

\begin{center}\rule{0.5\linewidth}{0.5pt}\end{center}

\begin{center}\rule{0.5\linewidth}{0.5pt}\end{center}

\chapter{Notas legais, escopo e uso responsável
(educacional)}\label{notas-legais-escopo-e-uso-responsuxe1vel-educacional}

\begin{quote}
\textbf{Alerta:} Este material tem finalidade \textbf{educacional} para
profissionais de saúde devidamente treinados.
\end{quote}

\begin{quote}
Não substitui treinamento supervisionado, diretrizes institucionais,
julgamento clínico, nem consentimento informado.
\end{quote}

\section{Escopo do livro}\label{escopo-do-livro}

\begin{itemize}
\item
  Este livro foca em \textbf{rejuvenescimento periorbitário}:
  diagnóstico, planejamento, técnica e sustentação.
\item
  O objetivo é elevar a tomada de decisão do ``resultado aceitável''
  para o resultado excelente.
\end{itemize}

\section{Consentimento e direitos de
imagem}\label{consentimento-e-direitos-de-imagem}

\begin{itemize}
\item
  Toda foto clínica deve ter consentimento escrito, com escopo de uso e
  preservação de privacidade.
\item
  Use padrões locais de ética/publicidade médica.
\end{itemize}

\section{Como este livro deve ser
lido}\label{como-este-livro-deve-ser-lido}

\begin{itemize}
\item
  Primeiro: \textbf{Diagnóstico} e \textbf{Planejamento}.
\item
  Depois: técnica.
\item
  Sempre: ``operar ≠ rejuvenescer'' → trate \textbf{causas}
  (volume/ligamentos/osso), não só sintomas.
\end{itemize}

\section{Convenções}\label{convenuxe7uxf5es}

\begin{itemize}
\item
  Termos em inglês serão definidos na primeira ocorrência e
  padronizados.
\item
  Medidas e testes (ex.: MRD1, snapback) terão definição prática e
  indicação.
\end{itemize}

\begin{center}\rule{0.5\linewidth}{0.5pt}\end{center}

\begin{center}\rule{0.5\linewidth}{0.5pt}\end{center}

\chapter{Capítulo 01 --- Introdução: a filosofia do rejuvenescimento
(operar ≠
rejuvenescer)}\label{capuxedtulo-01-introduuxe7uxe3o-a-filosofia-do-rejuvenescimento-operar-rejuvenescer}

\begin{figure}
\centering
\pandocbounded{\includegraphics[keepaspectratio,alt={Figura\,01.1 --- Operar\,≠\,Rejuvenescer}]{/Users/humbertolopes/Dev/work/marcelo-cury/the_art_of_eyelid_surgery_scaffold/projects/eyelid-surgery/assets/figures/FIG-01-01_filosofia-rejuvenescimento.png}}
\caption{Figura\,01.1 --- Operar\,≠\,Rejuvenescer}
\end{figure}

\begin{quote}
\textbf{Leitura guiada:} este capítulo aborda *Nota de escopo
(essencial):** este livro é educacional para profissionais de saúde
treinados. Não substitui treinamento supervisionado, diretrizes
institucionais, julgamento clínico, consentimento informado e avaliação
oftalmológica quando indicada.
\end{quote}

\textbf{Parte:} Introdução

\section{Objetivo do capítulo}\label{objetivo-do-capuxedtulo}

Ao final, o leitor saberá distinguir a diferença entre \textbf{operar} e
\textbf{rejuvenescer} o olhar: não se trata de ``retirar pele e
gordura'', mas de restaurar \textbf{continuidade de luz e sombra},
respeitando a anatomia individual e reduzindo estigmas evitáveis.

\begin{quote}
\textbf{PÉROLA CLÍNICA}
\end{quote}

\begin{quote}
\textbf{Nota de escopo (essencial):} este livro é educacional para
profissionais de saúde treinados. Não substitui treinamento
supervisionado, diretrizes institucionais, julgamento clínico,
consentimento informado e avaliação oftalmológica quando indicada.
\end{quote}

\section{O que muda na decisão (o
``porquê'')}\label{o-que-muda-na-decisuxe3o-o-porquuxea}

\begin{itemize}
\item
  \textbf{Ponto clínico-chave:} o envelhecimento periorbitário é
  dominado por \textbf{deflação volumétrica} e \textbf{falha de
  sustentação} (tecidos/ligamentos/apoios). ``Excesso'' existe --- mas,
  com frequência, é o sintoma mais visível e não a causa principal.
\item
  Risco evitável: a esqueletização da órbita e a quebra da continuidade
  de luz/sombra (hollow eye / A-frame / sulcos acentuados) quando a
  cirurgia é exclusivamente subtrativa e agressiva.
\item
  Erro comum (nota 7 → nota 10): tratar a pálpebra isoladamente e
  ignorar fatores transversais --- especialmente supercílio (Connell) e
  vetor/suporte lateral --- resultando em melhora parcial com aparência
  ``operada'' ou instável ao longo do tempo.
\end{itemize}

\section{Para quem é / Para quem não
é}\label{para-quem-uxe9-para-quem-nuxe3o-uxe9}

\textbf{Este livro é para você se:} - você quer previsibilidade e
naturalidade (resultado ``sem assinatura cirúrgica'');

\begin{itemize}
\item
  você deseja decidir por \textbf{diagnóstico} (luz/sombra, vetor,
  flacidez, volume, ptose, assimetrias) e não por hábito técnico;
\item
  você aceita que ``menos é mais'' é, muitas vezes, uma regra de
  segurança estética e funcional.
\end{itemize}

\textbf{Este livro não é para você se:} - você procura um passo-a-passo
rápido sem integrar diagnóstico, suporte e volume;

\begin{itemize}
\tightlist
\item
  você acredita que ``bolsa = gordura para tirar'' e ``excesso de pele =
  cortar até caber''.
\end{itemize}

\begin{quote}
\textbf{PÉROLA CLÍNICA}
\end{quote}

\begin{quote}
\mbox{}%
\section{Checklist mental (antes de qualquer
bisturi)}\label{checklist-mental-antes-de-qualquer-bisturi}
\end{quote}

\begin{quote}
\begin{itemize}
\tightlist
\item[$\square$]
  O que domina o caso: \textbf{excesso}, \textbf{deflação},
  \textbf{sustentação} --- ou uma combinação?
\end{itemize}
\end{quote}

\begin{quote}
\begin{itemize}
\tightlist
\item[$\square$]
  A estética do caso é ``luz/sombra'': onde está a sombra que envelhece
  e por quê?
\end{itemize}
\end{quote}

\begin{quote}
\begin{itemize}
\tightlist
\item[$\square$]
  Existe brow ptosis (queda) ou compensação frontal mascarando o
  problema?
\end{itemize}
\end{quote}

\begin{quote}
\begin{itemize}
\tightlist
\item[$\square$]
  Qual é o vetor (positivo/neutro/negativo) e como isso muda risco e
  plano?
\end{itemize}
\end{quote}

\begin{quote}
\begin{itemize}
\tightlist
\item[$\square$]
  A pálpebra inferior precisa de suporte lateral (canto) para ficar
  estável?
\end{itemize}
\end{quote}

\begin{quote}
\begin{itemize}
\tightlist
\item[$\square$]
  O plano é subtrativo, reposicionador e/ou aditivo (volume) --- e em
  que ordem?
\end{itemize}
\end{quote}

\begin{quote}
\begin{itemize}
\tightlist
\item[$\square$]
  O que, se for removido em excesso, será difícil de resgatar?
\end{itemize}
\end{quote}

\section{Anatomia aplicada (apenas o que muda
conduta)}\label{anatomia-aplicada-apenas-o-que-muda-conduta}

\begin{itemize}
\item
  \textbf{Lamelas (anterior/posterior):} decisão cirúrgica de qualidade
  separa pele/músculo de suporte interno; confundir isso gera estigma e
  complicações.
\item
  \textbf{Septo orbitário e gordura periorbitária:} não são
  ``inimigos''; são parte da arquitetura da luz/sombra e da proteção do
  contorno orbitário.
\item
  \textbf{Ligamentos retentores e canto lateral:} transições e
  estabilidade dependem de reconhecer âncoras e limites. Sustentação é
  estética e, muitas vezes, é também segurança funcional.
\end{itemize}

(Se você acrescentar aqui ``zonas de risco'' anatômicas específicas e
detalhes de testes/exame físico, isso normalmente fica melhor no
capítulo de Exame Físico / Anatomia aplicada.)

\section{Método (o raciocínio que o livro
ensina)}\label{muxe9todo-o-raciocuxednio-que-o-livro-ensina}

\begin{enumerate}
\def\labelenumi{\arabic{enumi}.}
\item
  \textbf{Ver} (fotografia e luz/sombra) + \textbf{medir} (exame físico:
  vetor, flacidez, ptose, assimetrias).
\item
  \textbf{Nomear a causa dominante} (excesso vs deflação vs
  sustentação).
\item
  Escolher a estratégia: preservar, redistribuir, sustentar e/ou
  volumizar --- antes de decidir ``quanto tirar''.
\item
  Executar com conservadorismo: o estigma quase sempre nasce do excesso.
\item
  Reavaliar com timing correto: tecido tem tempo; resgate precoce por
  ansiedade costuma piorar o problema.
\end{enumerate}

\begin{quote}
\textbf{PÉROLA CLÍNICA}
\end{quote}

\begin{quote}
\mbox{}%
\section{Erros comuns (e como evitar o caminho do
estigma)}\label{erros-comuns-e-como-evitar-o-caminho-do-estigma}
\end{quote}

\begin{quote}
\begin{itemize}
\tightlist
\item
  \textbf{Subtrair onde o problema é deflação:} melhora momentânea, mas
  envelhece a órbita e marca o olhar.
\end{itemize}
\end{quote}

\begin{quote}
\begin{itemize}
\tightlist
\item
  \textbf{Ignorar supercílio (Connell):} corrige pele, mas falha no
  ``olhar'' (moldura não tratada).
\end{itemize}
\end{quote}

\begin{quote}
\begin{itemize}
\tightlist
\item
  \textbf{Desconsiderar vetor e suporte lateral:} abre a porta para
  instabilidade, retrações e ``olho redondo'' em pacientes de risco.
\end{itemize}
\end{quote}

\begin{quote}
\begin{itemize}
\tightlist
\item
  Prometer simetria perfeita: assimetria prévia é regra; documentação e
  alinhamento de expectativa são parte da técnica.
\end{itemize}
\end{quote}

\section{Notas de ``arte'' (luz/sombra, continuidade, unidades
estéticas)}\label{notas-de-arte-luzsombra-continuidade-unidades-estuxe9ticas}

\begin{itemize}
\item
  Rejuvenescimento é restaurar \textbf{continuidade}: uma transição
  suave entre unidades estéticas vizinhas.
\item
  No inferior, o objetivo não é ``pálpebra plana'', mas uma transição
  pálpebra-malar \textbf{imperceptível}.
\item
  Volume bem posicionado devolve \textbf{luz}; suporte bem planejado
  preserva forma; ressecção excessiva cria sombra iatrogênica.
\end{itemize}

\section{Referências / leituras recomendadas
(opcional)}\label{referuxeancias-leituras-recomendadas-opcional}

\begin{itemize}
\tightlist
\item
  {[}{[}REF:FAGIEN-1999{]}{]} Abordagem conservadora e filosofia do
  rejuvenescimento periorbital.
\end{itemize}

\begin{center}\rule{0.5\linewidth}{0.5pt}\end{center}

\begin{center}\rule{0.5\linewidth}{0.5pt}\end{center}

\chapter{Capítulo 02 --- Luz e sombra: unidades estéticas e continuidade
periorbitária}\label{capuxedtulo-02-luz-e-sombra-unidades-estuxe9ticas-e-continuidade-periorbituxe1ria}

\begin{figure}
\centering
\pandocbounded{\includegraphics[keepaspectratio,alt={Figura 02.1 --- Ilustração principal do capítulo}]{/Users/humbertolopes/Dev/work/marcelo-cury/the_art_of_eyelid_surgery_scaffold/projects/eyelid-surgery/assets/figures/FIG-02-01_luz-sombra-periorbitario.png}}
\caption{Figura 02.1 --- Ilustração principal do capítulo}
\end{figure}

\textbf{Parte:} Introdução

\section{Objetivo do capítulo}\label{objetivo-do-capuxedtulo-1}

Ao final deste capítulo, o leitor será capaz de diferenciar um ``olho
operado'' de um ``olho rejuvenescido'' pelo que realmente importa:
\textbf{o desenho das sombras}. Você aprenderá a identificar como a
\textbf{deflação volumétrica} cria sombras de envelhecimento e como
restaurar a transição suave entre pálpebra e malar sem produzir
estigmas.

\section{O que muda na decisão (o
``porquê'')}\label{o-que-muda-na-decisuxe3o-o-porquuxea-1}

\begin{itemize}
\item
  \textbf{Ponto clínico-chave:} envelhecimento periorbitário é, em
  grande parte, uma história de \textbf{perda de volume} e perda de
  continuidade entre unidades estéticas --- não apenas ``pele
  sobrando''.
\item
  \textbf{Risco evitável:} esqueletização da órbita (hollow eye /
  aspecto cadavérico) quando a cirurgia é exclusivamente subtrativa e
  agressiva.
\item
  Erro comum: avaliar apenas a ``quantidade de pele'' e ignorar a
  fotografia sem flash, que revela as sombras verdadeiras (A-frame, tear
  trough, lid-cheek junction marcada).
\end{itemize}

\begin{quote}
\textbf{PÉROLA CLÍNICA}
\end{quote}

\begin{quote}
\mbox{}%
\section{Regra prática (o que a foto está te
dizendo)}\label{regra-pruxe1tica-o-que-a-foto-estuxe1-te-dizendo}
\end{quote}

\begin{quote}
\begin{itemize}
\tightlist
\item
  \textbf{Sombra de deflação:} falta convexidade → falta luz → o sulco
  ``aparece''.
\end{itemize}
\end{quote}

\begin{quote}
\begin{itemize}
\tightlist
\item
  \textbf{Sombra de ancoragem:} ligamento/aderência ``segura'' a
  transição → a quebra persiste até você respeitar essa mecânica.
\end{itemize}
\end{quote}

\begin{quote}
\begin{itemize}
\tightlist
\item
  \textbf{Sombra de edema (festoon):} volume ``móvel'' e inflamatório →
  não responde como deflação.
\end{itemize}
\end{quote}

\begin{quote}
\begin{itemize}
\tightlist
\item
  Cor ≠ sombra: hiperpigmentação pode parecer olheira, mas não muda com
  flash e não some com subtração.
\end{itemize}
\end{quote}

\section{Para quem este capítulo muda o
jogo}\label{para-quem-este-capuxedtulo-muda-o-jogo}

\textbf{Aplicar estes conceitos quando:} - houver ``dupla convexidade''
no inferior (bolsa + malar) separada por um vale de sombra (tear trough
/ palpebro-malar);

\begin{itemize}
\item
  houver \textbf{A-frame} no superior (sombra triangular medial/central)
  sugerindo deflação e não ``excesso'';
\item
  houver tendência a hollow eye (deflação dominante, pós-blefaroplastia
  prévia, olhos proeminentes/vetor de risco).
\end{itemize}

\textbf{Cautela especial quando:} - o paciente tiver edema malar/festoon
predominante: sombras ``hidráulicas'' (edema) não se comportam como
sombras ``anatômicas'' (deflação);

\begin{itemize}
\tightlist
\item
  a queixa principal for \textbf{cor} (hiperpigmentação) e não
  \textbf{relevo}: cirurgia trata relevo/sombra; cor exige manejo
  dermatológico (quando indicado).
\end{itemize}

\begin{quote}
\textbf{PÉROLA CLÍNICA}
\end{quote}

\begin{quote}
\mbox{}%
\section{Checklist pré-op (luz/sombra em 60
segundos)}\label{checklist-pruxe9-op-luzsombra-em-60-segundos}
\end{quote}

\begin{quote}
\begin{itemize}
\tightlist
\item[$\square$]
  Foto sem flash: onde a sombra nasce? (A-frame, tear trough, lid-cheek
  junction)
\end{itemize}
\end{quote}

\begin{quote}
\begin{itemize}
\tightlist
\item[$\square$]
  Foto \textbf{com flash} (quando aplicável): avaliar ptose (MRD1) e
  assimetrias funcionais
\end{itemize}
\end{quote}

\begin{quote}
\begin{itemize}
\tightlist
\item[$\square$]
  Vetor: positivo / neutro / negativo (perfil: relação globo-malar)
\end{itemize}
\end{quote}

\begin{quote}
\begin{itemize}
\tightlist
\item[$\square$]
  Flacidez: snapback / distraction (define necessidade de suporte
  lateral)
\end{itemize}
\end{quote}

\begin{quote}
\begin{itemize}
\tightlist
\item[$\square$]
  Assimetria prévia documentada (sulco, sobrancelha, fissura palpebral)
\end{itemize}
\end{quote}

\begin{quote}
\begin{itemize}
\tightlist
\item[$\square$]
  Plano de sustentação (quando necessário) + plano de volume (quando
  necessário)
\end{itemize}
\end{quote}

\section{Anatomia aplicada (apenas o que muda
conduta)}\label{anatomia-aplicada-apenas-o-que-muda-conduta-1}

\begin{itemize}
\item
  \textbf{Arcus marginalis (rebordo orbitário):} é a ``linha'' onde luz
  vira sombra. Rejuvenescimento do inferior, muitas vezes, significa
  \textbf{suavizar a leitura desse rebordo} na transição pálpebra-malar.
\item
  Ligamento órbito-malar (orbicular retaining ligament): é uma âncora
  que participa do sulco e da sombra. Se a âncora domina o desenho, a
  transição tende a ficar ``quebrada''.
\item
  \textbf{Zona de risco (conceitual):} entre compartimentos no inferior,
  há estruturas nobres (incluindo o oblíquo inferior). Regra:
  visualizar, respeitar planos e evitar manobras às cegas.
\end{itemize}

(Detalhamento de ``zona de risco'' do oblíquo inferior e anatomia entre
bolsas medial/central funciona melhor no capítulo técnico de pálpebra
inferior/transconjuntival.)

\section{Método prático (sem virar
``passo-a-passo'')}\label{muxe9todo-pruxe1tico-sem-virar-passo-a-passo}

\begin{enumerate}
\def\labelenumi{\arabic{enumi}.}
\item
  \textbf{Localize a sombra dominante} (onde ela começa e por quê).
\item
  Decida se a sombra é principalmente de \textbf{deflação} (falta de
  volume), de \textbf{ancoragem} (ligamento), de edema (festoon) ou
  mista.
\item
  Planeje uma estratégia coerente: preservar/redistribuir/volumizar
  (quando indicado) e sustentar (quando necessário) --- antes de pensar
  em ``quanto tirar''.
\item
  Garanta que o plano reduz estigmas: o objetivo é continuidade, não
  ``pálpebra chapada''.
\end{enumerate}

\section{Variações de estratégia (o raciocínio por
trás)}\label{variauxe7uxf5es-de-estratuxe9gia-o-raciocuxednio-por-truxe1s}

\begin{itemize}
\item
  \textbf{Redistribuição/transposição de gordura:} útil quando há bolsa
  proeminente e vale profundo imediato (tear trough) --- a ideia é usar
  volume existente para suavizar o vale.
\item
  \textbf{Volumização (microfat):} preferível quando há \textbf{deflação
  global} (hollow eye, pós-blefaroplastia subtrativa, pouca bolsa para
  redistribuir) --- a ideia é devolver luz onde falta convexidade.
\item
  Refinamento de pele: quando necessário, deve ser conservador e nunca
  substituir correção de causa (deflação/suporte).
\end{itemize}

(Se você incluir nomes técnicos e variações específicas como Loeb/Hamra,
isso normalmente vai para o capítulo técnico do inferior/terço médio.)

\begin{quote}
\textbf{PÉROLA CLÍNICA}
\end{quote}

\begin{quote}
\mbox{}%
\section{Erros comuns (e como resgatar sem
piorar)}\label{erros-comuns-e-como-resgatar-sem-piorar}
\end{quote}

\begin{quote}
\begin{itemize}
\tightlist
\item
  \textbf{Erro --- ``excesso'' tratado onde há deflação:} melhora
  inicial, mas aprofunda sombras. → prevenir reconhecendo
  hollow/A-frame/tear trough como problema de volume.
\end{itemize}
\end{quote}

\begin{quote}
\begin{itemize}
\tightlist
\item
  \textbf{Erro --- remover bolsa e manter o vale:} bolsa some, mas a
  sombra permanece. → prevenir tratando a transição
  (redistribuição/volume quando indicado).
\end{itemize}
\end{quote}

\begin{quote}
\begin{itemize}
\tightlist
\item
  \textbf{Erro --- confundir cor com sombra:} operar relevo quando o
  problema principal é pigmento. → alinhar expectativa e indicar manejo
  complementar quando aplicável.
\end{itemize}
\end{quote}

\section{Notas de ``arte'' (o que o olho humano
percebe)}\label{notas-de-arte-o-que-o-olho-humano-percebe}

\begin{itemize}
\item
  A estética periorbitária é um jogo de \textbf{curvas contínuas}:
  quando há ``quebra'', a sombra denuncia.
\item
  No inferior, a meta é transformar ``dupla convexidade'' em uma
  \textbf{curva única suave} (continuidade pálpebra-malar).
\item
  No superior, a meta é manter um reflexo de luz \textbf{contínuo} na
  plataforma, evitando sombras triangulares (A-frame) por subtração
  excessiva.
\end{itemize}

\section{Referências / leituras recomendadas
(opcional)}\label{referuxeancias-leituras-recomendadas-opcional-1}

\begin{itemize}
\tightlist
\item
  {[}{[}REF:LAMBROS-2007{]}{]} Estética periorbital e continuidade de
  unidades.
\end{itemize}

\begin{center}\rule{0.5\linewidth}{0.5pt}\end{center}

\begin{center}\rule{0.5\linewidth}{0.5pt}\end{center}

\chapter{Capítulo 03 --- Envelhecimento multifatorial: deflation,
ligamentos e
osso}\label{capuxedtulo-03-envelhecimento-multifatorial-deflation-ligamentos-e-osso}

\begin{figure}
\centering
\pandocbounded{\includegraphics[keepaspectratio,alt={Figura\,03.1 --- Envelhecimento multifatorial}]{/Users/humbertolopes/Dev/work/marcelo-cury/the_art_of_eyelid_surgery_scaffold/projects/eyelid-surgery/assets/figures/FIG-03-01_envelhecimento-multifatorial.png}}
\caption{Figura\,03.1 --- Envelhecimento multifatorial}
\end{figure}

\textbf{Parte:} Introdução

\section{Objetivo do capítulo}\label{objetivo-do-capuxedtulo-2}

Ao final deste capítulo, o leitor saberá diagnosticar o envelhecimento
periorbitário como um fenômeno \textbf{multifatorial} --- combinação de
\textbf{deflação volumétrica}, \textbf{falha de sustentação ligamentar}
e mudança do suporte ósseo --- e, com isso, trocar a lógica subtrativa
(``ressecar'') por uma estratégia mais previsível (``preservar,
estruturar e repor quando indicado'').

\section{O que muda na decisão (o
``porquê'')}\label{o-que-muda-na-decisuxe3o-o-porquuxea-2}

\begin{itemize}
\item
  \textbf{Ponto clínico-chave:} em muitos pacientes, o que envelhece
  primeiro é a \textbf{perda de volume} (deflação) e a \textbf{quebra de
  transições}, não ``pele sobrando''.
\item
  Risco evitável: esqueletização orbitária (hollow eye / cadaveric look)
  quando a cirurgia remove gordura em um paciente já deflacionado.
\item
  Erro comum: tentar corrigir a estética lateral (ruga lateral/Connell)
  e a ``pele a mais'' com blefaroplastia isolada, ignorando que o
  problema pode estar em descenso frontotemporal + deflação + suporte.
\end{itemize}

\begin{quote}
\textbf{PÉROLA CLÍNICA}
\end{quote}

\begin{quote}
\mbox{}%
\section{Mapa mental do envelhecimento (em 20
segundos)}\label{mapa-mental-do-envelhecimento-em-20-segundos}
\end{quote}

\begin{quote}
\begin{itemize}
\tightlist
\item
  \textbf{Deflação (volume):} falta convexidade → falta luz → o sulco
  ``aparece''.
\end{itemize}
\end{quote}

\begin{quote}
\begin{itemize}
\tightlist
\item
  \textbf{Ligamentos (âncoras):} a transição fica ``presa'' → a quebra
  persiste mesmo após subtração.
\end{itemize}
\end{quote}

\begin{quote}
\begin{itemize}
\tightlist
\item
  \textbf{Osso (suporte):} quando o suporte muda, a leitura do rebordo e
  das sombras muda junto.
\end{itemize}
\end{quote}

\begin{quote}
Regra: ``Excesso'' pode existir, mas frequentemente é efeito; a causa
costuma ser combinação dos três itens acima.
\end{quote}

\section{Aplicação prática}\label{aplicauxe7uxe3o-pruxe1tica}

\textbf{Aplicar este raciocínio quando:} - houver \textbf{A-frame} /
sulco superior marcado sugerindo deflação (não apenas excesso);

\begin{itemize}
\item
  houver ``dupla convexidade'' no inferior (bolsa + malar) separada por
  vale (tear trough / lid-cheek junction marcada);
\item
  houver sinais de \textbf{descenso} na moldura do olhar
  (supercílio/temporal) que mudam a leitura da pálpebra.
\end{itemize}

\textbf{Cautela especial quando:} - o paciente apresentar \textbf{vetor
de risco} (olho proeminente / suporte malar fraco), pois pequenas
subtrações podem gerar grande estigma;

\begin{itemize}
\tightlist
\item
  a queixa dominante for \textbf{cor} (hiperpigmentação) mais do que
  relevo (sombra). Cirurgia muda relevo; cor pode exigir outro manejo.
\end{itemize}

\begin{quote}
\textbf{PÉROLA CLÍNICA}
\end{quote}

\begin{quote}
\mbox{}%
\section{Checklist pré-op
(multifatorial)}\label{checklist-pruxe9-op-multifatorial}
\end{quote}

\begin{quote}
\begin{itemize}
\tightlist
\item[$\square$]
  Foto \textbf{sem flash}: onde a sombra nasce (A-frame, tear trough,
  lid-cheek)?
\end{itemize}
\end{quote}

\begin{quote}
\begin{itemize}
\tightlist
\item[$\square$]
  Foto \textbf{com flash} (quando aplicável): avaliar ptose e
  assimetrias funcionais (ex.: \textbf{MRD1}).
\end{itemize}
\end{quote}

\begin{quote}
\begin{itemize}
\tightlist
\item[$\square$]
  Vetor: positivo / neutro / negativo (perfil: globo × malar).
\end{itemize}
\end{quote}

\begin{quote}
\begin{itemize}
\tightlist
\item[$\square$]
  Flacidez: snapback / distraction (define necessidade de suporte
  lateral).
\end{itemize}
\end{quote}

\begin{quote}
\begin{itemize}
\tightlist
\item[$\square$]
  Assimetria prévia documentada (sulco, fissura, sobrancelha).
\end{itemize}
\end{quote}

\begin{quote}
\begin{itemize}
\tightlist
\item[$\square$]
  O problema dominante é: deflação, âncora ligamentar, suporte ósseo ---
  ou mistura?
\end{itemize}
\end{quote}

\begin{quote}
\begin{itemize}
\tightlist
\item[$\square$]
  Plano de sustentação (quando necessário) + plano de volume (quando
  necessário).
\end{itemize}
\end{quote}

\section{Anatomia aplicada (apenas o que muda
conduta)}\label{anatomia-aplicada-apenas-o-que-muda-conduta-2}

\begin{itemize}
\item
  Ligamento órbito-malar (orbicular retaining ligament): participa do
  sulco e da quebra pálpebra-malar. Se a ancoragem domina o caso,
  ``tirar bolsa'' não resolve o vale.
\item
  \textbf{Compartimentos gordurosos (ex.: ROOF, SOOF e correlatos):}
  envelhecimento reduz volume útil e altera a leitura de luz/sombra. O
  erro clássico é tratar gordura como ``excesso universal''.
\item
  \textbf{Suporte ósseo:} quando o rebordo ``aparece'' mais (por
  mudanças do suporte), a sombra fica mais legível; isso muda a
  estratégia (preservar/repor em vez de subtrair).
\end{itemize}

Detalhes anatômicos finos e ``zonas de risco'' específicas (ex.: ramo
frontal do facial, medidas laterais e marcos do brow/temporal) ficam
melhores no capítulo de anatomia aplicada/brow.

\section{Método de raciocínio (o algoritmo do
capítulo)}\label{muxe9todo-de-raciocuxednio-o-algoritmo-do-capuxedtulo}

\begin{enumerate}
\def\labelenumi{\arabic{enumi}.}
\item
  Identifique a \textbf{sombra dominante} (sem flash) e o que a causa.
\item
  Classifique a causa: deflação vs âncora ligamentar vs suporte (ou
  combinação).
\item
  Decida a estratégia: \textbf{preservar}, \textbf{reposicionar},
  sustentar e/ou repor volume (quando indicado) --- antes de decidir
  ``quanto tirar''.
\item
  Planeje para evitar estigmas: o objetivo é continuidade, não
  ``pálpebra chapada''.
\end{enumerate}

\begin{quote}
\textbf{PÉROLA CLÍNICA}
\end{quote}

\begin{quote}
\mbox{}%
\section{Erros comuns e resgate (sem
piorar)}\label{erros-comuns-e-resgate-sem-piorar}
\end{quote}

\begin{quote}
\begin{itemize}
\tightlist
\item
  \textbf{Erro --- subtrair onde o problema é deflação:} aprofunda
  sombras → resgate costuma exigir reposição planejada de volume.
\end{itemize}
\end{quote}

\begin{quote}
\begin{itemize}
\tightlist
\item
  \textbf{Erro --- blefaroplastia isolada em cenário Connell:} melhora
  parcial com piora da moldura → resgate geralmente envolve
  reposicionamento da moldura (quando indicado).
\end{itemize}
\end{quote}

\begin{quote}
\begin{itemize}
\tightlist
\item
  \textbf{Erro --- ignorar vetor/suporte:} abre porta para instabilidade
  e ``olho redondo'' em pacientes de risco → resgate costuma ser mais
  complexo do que a prevenção.
\end{itemize}
\end{quote}

\section{Notas de ``arte'' (luz/sombra e
transições)}\label{notas-de-arte-luzsombra-e-transiuxe7uxf5es}

\begin{itemize}
\item
  Juventude é \textbf{transição imperceptível} entre unidades estéticas;
  envelhecimento é a leitura de uma ``linha''.
\item
  Deflação rouba luz; ancoragens criam quebras; suporte muda o desenho
  da sombra.
\item
  O objetivo estético mais confiável é restaurar \textbf{curvas
  contínuas} (especialmente na junção pálpebra-malar), com
  conservadorismo.
\end{itemize}

\section{Técnica (visão geral +
variações)}\label{tuxe9cnica-visuxe3o-geral-variauxe7uxf5es}

Conteúdo de passo-a-passo, liberação, acesso e variações técnicas
(incluindo nomes de técnicas) deve ir para os capítulos técnicos
(pálpebra inferior/terço médio, supercílio e volumização), mantendo este
capítulo como ``modelo mental''.

\section{Pós-operatório e
follow-up}\label{puxf3s-operatuxf3rio-e-follow-up}

Protocolos de pós, timing e sinais de alarme ficam melhores concentrados
no capítulo de pós/complicações.

\section{Referências / leituras recomendadas
(opcional)}\label{referuxeancias-leituras-recomendadas-opcional-2}

\begin{itemize}
\item
  {[}{[}REF:ROHRICH-2008{]}{]} Rohrich, Rod J. --- Compartimentos de
  gordura e envelhecimento facial.
\item
  {[}{[}REF:MARTEN-2008{]}{]} Marten, Timothy --- Fat grafting na região
  periorbitária.
\item
  {[}{[}REF:MENDELSON-2008{]}{]} Mendelson, Bryan --- Ligamentos
  retentores e anatomia aplicada.
\item
  {[}{[}REF:MASSRY-2012{]}{]} Massry, Guy --- Midface aging e Ogee
  curve.
\end{itemize}

\begin{center}\rule{0.5\linewidth}{0.5pt}\end{center}

\chapter{Capítulo 04 --- Anatomia cirúrgica aplicada: lamelas, septo e
ligamentos
retentores}\label{capuxedtulo-04-anatomia-ciruxfargica-aplicada-lamelas-septo-e-ligamentos-retentores}

\begin{figure}
\centering
\pandocbounded{\includegraphics[keepaspectratio,alt={Figura\,04.1 --- Anatomia cirúrgica aplicada}]{/Users/humbertolopes/Dev/work/marcelo-cury/the_art_of_eyelid_surgery_scaffold/projects/eyelid-surgery/assets/figures/FIG-04-01_lamelas-septo-ligamentos.png}}
\caption{Figura\,04.1 --- Anatomia cirúrgica aplicada}
\end{figure}

\textbf{Parte:} Parte 0 --- Princípios e Segurança

\textbf{Objetivo do capítulo:} Ao final deste capítulo, o leitor
compreenderá a pálpebra não como um retalho de pele, mas como uma
estrutura trilamelar dinâmica, sendo capaz de basear sua estratégia
cirúrgica na competência de cada componente anatômico.

\section{O que muda na decisão (o
``porquê'')}\label{o-que-muda-na-decisuxe3o-o-porquuxea-3}

\begin{itemize}
\item
  \textbf{A distinção entre cantopexia e cantoplastia:} A decisão de
  abrir a comissura (cantoplastia) depende da necessidade de liberar o
  ligamento órbito-malar para mobilização vertical, enquanto a
  cantopexia atua apenas no suporte sem desestruturar o retináculo
  lateral .
\item
  \textbf{Vetor negativo e suporte:} A anatomia do rebordo ósseo em
  relação ao globo (vetor) dita a agressividade da ressecção; em vetores
  negativos, a anatomia da lamela posterior é o único anteparo contra o
  ectrópio.
\item
  \textbf{Respeito à bomba lacrimal:} A preservação de 2 a 3 mm de
  músculo orbicular pré-tarsal é o que separa um resultado estético
  estático de uma pálpebra funcional e estável no pós-operatório
  {[}{[}REF: Cury{]}{]}.
\end{itemize}

\section{Aplicação prática}\label{aplicauxe7uxe3o-pruxe1tica-1}

\textbf{Aplicar quando:}

\begin{itemize}
\item
  Houver necessidade de reconstrução palpebral após exérese de tumores.
\item
  O diagnóstico de frouxidão ligamentar (snapback test positivo) exigir
  reforço estrutural.
\item
  A transição pálpebra-face (lid-cheek junction) estiver marcada pela
  descida do compartimento de gordura malar.
\item
  Houver retração cicatricial que demande liberação de retratores e
  expansores de lamela.
\end{itemize}

\textbf{Cautela quando:}

\begin{itemize}
\item
  O paciente apresentar olho seco severo; a manipulação da lamela
  posterior e do septo pode exacerbar a exposição corneana.
\item
  Houver vetor negativo pronunciado; a gravidade atua contra a anatomia,
  exigindo ancoragem óssea mais superior que o tubérculo de Whitnall.
\item
  Existirem cirurgias prévias; a fibrose altera os planos entre o septo
  e a aponeurose do levantador, aumentando o risco de ptose iatrogênica.
\end{itemize}

\section{Checklist pré-op (itens de
segurança)}\label{checklist-pruxe9-op-itens-de-seguranuxe7a}

\begin{itemize}
\item[$\square$]
  Foto sem flash para avaliação de luz, sombra e festoons.
\item[$\square$]
  Foto com flash para documentar ptoses leves mascaradas pelo excesso de
  pele.
\item[$\square$]
  Avaliação do Vetor: Positivo / Neutro / Negativo.
\item[$\square$]
  Teste de Snapback e Distraction para competência da lamela posterior.
\item[$\square$]
  Teste da lamela anterior (pinch test) para estimativa de ressecção de
  pele.
\item[$\square$]
  Plano de sustentação definido: Cantopexia vs.~Cantoplastia.
\item[$\square$]
  Plano de volume: Microfat graft para transição de unidades estéticas.
\end{itemize}

\section{Anatomia aplicada (apenas o que muda o
bisturi)}\label{anatomia-aplicada-apenas-o-que-muda-o-bisturi}

\begin{itemize}
\item
  \textbf{Lamelas Anterior e Posterior:} A pálpebra é dividida pelo
  septo orbitário. A lamela anterior (pele e orbicular) é elástica; a
  posterior (tarso e conjuntiva) é o arcabouço rígido. Em reconstruções,
  a regra de ouro é: se usar um retalho para uma lamela, use um enxerto
  para a outra para garantir suprimento vascular .
\item
  \textbf{Septo Orbitário:} Atua como a ``porta'' das bolsas de gordura.
  Sua fraqueza permite a herniação gordurosa. No acesso
  transconjuntival, o septo é preservado quando o acesso é pós-septal,
  diminuindo o risco de retração palpebral.
\item
  \textbf{Ligamento Órbito-malar (LOM):} Um ligamento osteocutâneo que
  fixa o orbicular ao rebordo inferior. Sua liberação é mandatória para
  tratar o sulco nasojugal e elevar o terço médio.
\item
  ``Zona de risco'': Músculo oblíquo inferior. Localizado entre as
  bolsas de gordura medial e central da pálpebra inferior. Sua lesão
  durante a blefaroplastia causa diplopia.
\end{itemize}

\begin{quote}
\textbf{📎 FIGURA NECESSÁRIA (Cap. 04):}
\end{quote}

\begin{quote}
Corte sagital da pálpebra: Lamelas anterior e posterior, septo, tarso
\end{quote}

\begin{quote}
\emph{Estilo: Diagrama técnico-didático, cores neutras, legendas claras}
\end{quote}

\section{Técnica (visão geral)}\label{tuxe9cnica-visuxe3o-geral}

\subsection{Visão geral}\label{visuxe3o-geral}

\begin{itemize}
\item
  Identificação da frouxidão estrutural através de testes dinâmicos.
\item
  Escolha da via de acesso (transconjuntival para gordura isolada
  vs.~transcutânea para excesso de pele).
\item
  Liberação seletiva de ligamentos retentores (LOM) para suavizar a
  transição pálpebra-face.
\item
  Ancoragem da lamela posterior (tarsal strip ou cantopexia) ao
  periósteo do rebordo lateral.
\item
  Ajuste conservador da lamela anterior (pele) após a estabilização
  estrutural.
\end{itemize}

\subsection{Variações e
indicações}\label{variauxe7uxf5es-e-indicauxe7uxf5es}

\begin{itemize}
\item
  \textbf{Tarsal Strip:} Indicado quando há frouxidão horizontal real
  (distração \textgreater{} 6-8 mm). Exige cantotomia e cantólise.
\item
  \textbf{Cantopexia de Mladick:} Suporte muscular/ligamentar indicado
  em casos estéticos com frouxidão leve, sem necessidade de abrir a
  comissura .
\end{itemize}

\section{Erros comuns (e como
resgatar)}\label{erros-comuns-e-como-resgatar}

\begin{itemize}
\item
  \textbf{Ressecção excessiva de pele na pálpebra inferior} →
  Consequência: Escleral show ou ectrópio → Prevenção: Sempre realizar a
  manobra de sustentação (cantopexia) antes de marcar a pele → Resgate:
  Enxerto de pele total ou retalho de transposição da pálpebra superior.
\item
  Não liberar o ligamento órbito-malar em pacientes com sulco profundo →
  Consequência: Persistência da sombra na transição pálpebra-face →
  Prevenção: Dissecção submuscular até o rebordo ósseo → Resgate:
  Preenchimento tardio com microfat graft. \textbf{Princípio da
  Reconstrução de Lamelas:}
\end{itemize}

Nunca utilize dois tecidos vascularizados (dois retalhos) para
reconstruir ambas as lamelas simultaneamente; o resultado será grosseiro
e espesso. Combine sempre um retalho (suporte sanguíneo) com um enxerto
(suporte estrutural). \textbf{O Músculo Oblíquo Inferior:}

Ao abordar a bolsa central e medial na blefaroplastia inferior,
identifique o oblíquo inferior. Ele é a estrutura mais nobre da região e
deve ser ativamente afastado antes de qualquer cauterização ou ressecção
de gordura.

\section{Notas de ``arte''}\label{notas-de-arte}

A beleza da região periorbitária reside na \textbf{continuidade de luz}.
O objetivo da manipulação anatômica é transformar uma transição abrupta
(degrau pálpebra-face) em uma curva suave e convexa. Ao tensionar o
orbicular e liberar o LOM, eliminamos as ``ilhas'' de sombra que dão o
aspecto de cansaço, preservando a forma amendoada original do olho.

\section{Pós-operatório e
follow-up}\label{puxf3s-operatuxf3rio-e-follow-up-1}

\begin{itemize}
\item
  \textbf{24--72h:} Avaliar quemose e oclusão palpebral. Pequenas
  torrafias laterais provisórias podem prevenir quemose severa em
  grandes descolamentos.
\item
  \textbf{7--14d:} Retirada de pontos de sustentação (nylon 5-0).
  Observar a posição do canto lateral (deve estar 1-2 mm acima do canto
  medial).
\item
  \textbf{Sinais de alarme:} Dor súbita com proptose (hematoma
  retrobulbar --- emergência cirúrgica) ou incapacidade de fechamento
  palpebral com ceratite.
\end{itemize}

\section{Referências / leituras
recomendadas}\label{referuxeancias-leituras-recomendadas}

\begin{itemize}
\item
  {[}{[}REF: Codner --- Eyelid \& Periorbital Surgery{]}{]}
\item
  {[}{[}REF: Rohrich --- Fat Compartments of the Face{]}{]}
\item
  {[}{[}REF: Mladick --- Muscle Suspension Techniques{]}{]}
\end{itemize}

\begin{center}\rule{0.5\linewidth}{0.5pt}\end{center}

\chapter{Capítulo 05 --- Mapas de risco e erros de plano: zonas de
segurança vs
perigo}\label{capuxedtulo-05-mapas-de-risco-e-erros-de-plano-zonas-de-seguranuxe7a-vs-perigo}

\begin{figure}
\centering
\pandocbounded{\includegraphics[keepaspectratio,alt={Figura 05.1 --- Ilustração principal do capítulo}]{/Users/humbertolopes/Dev/work/marcelo-cury/the_art_of_eyelid_surgery_scaffold/projects/eyelid-surgery/assets/figures/FIG-05-01_mapa-risco-zonas.png}}
\caption{Figura 05.1 --- Ilustração principal do capítulo}
\end{figure}

\textbf{Parte:} Parte 0 --- Princípios e Segurança

\section{Objetivo do capítulo}\label{objetivo-do-capuxedtulo-3}

Ao final deste capítulo, o leitor saberá mapear zonas de risco
\textbf{neurovascular} e \textbf{muscular} na órbita e região temporal,
escolhendo planos de clivagem que protejam função motora e preservem a
integridade da superfície ocular.

\section{O que muda na decisão (o
``porquê'')}\label{o-que-muda-na-decisuxe3o-o-porquuxea-4}

\begin{itemize}
\item
  \textbf{Vetor negativo vs ressecção:} em pacientes com vetor negativo
  (globo mais proeminente que o suporte malar), a decisão tende a migrar
  de ``retirar pele'' para ``sustentar e, quando indicado, volumizar''.
  Ignorar vetor é fonte frequente de retração iatrogênica.
\item
  \textbf{Plano temporal e proteção neural:} na região temporal, a
  escolha correta do plano de dissecção é determinante para proteger o
  ramo temporo-frontal do nervo facial. {[}{[}REF:KNIZE-2001{]}{]}
\item
  \textbf{Preservação funcional do orbicular:} preservar faixa adequada
  do orbicular pré-tarsal ajuda a manter estabilidade marginal e função
  da bomba lacrimal (separando resultado funcional de estigma).
\end{itemize}

\section{Aplicação prática}\label{aplicauxe7uxe3o-pruxe1tica-2}

\textbf{Aplicar quando:} - blefaroplastia inferior transcutânea
(identificação do oblíquo inferior);

\begin{itemize}
\item
  brow lift (mapeamento de nervos supraorbitários e ramos do facial);
\item
  manejo de hipertrofia do orbicular;
\item
  reoperação (planos distorcidos por fibrose).
\end{itemize}

\textbf{Evitar / adiar quando:} - fenômeno de Bell ausente/fraco ou olho
seco severo (prioridade funcional);

\begin{itemize}
\item
  proptose não diagnosticada (risco de lagoftalmo e exposição);
\item
  edema malar ativo/instável sem diagnóstico sistêmico.
\end{itemize}

\begin{quote}
\textbf{PÉROLA CLÍNICA}
\end{quote}

\begin{quote}
\mbox{}%
\section{Checklist pré-op (itens de
segurança)}\label{checklist-pruxe9-op-itens-de-seguranuxe7a-1}
\end{quote}

\begin{quote}
\begin{itemize}
\tightlist
\item[$\square$]
  Foto sem flash (luz/sombra, bolsas e sulcos)
\end{itemize}
\end{quote}

\begin{quote}
\begin{itemize}
\tightlist
\item[$\square$]
  Foto com flash (quando aplicável): MRD1 e ptoses leves
\end{itemize}
\end{quote}

\begin{quote}
\begin{itemize}
\tightlist
\item[$\square$]
  Vetor: positivo / neutro / negativo
\end{itemize}
\end{quote}

\begin{quote}
\begin{itemize}
\tightlist
\item[$\square$]
  Fenômeno de Bell: presente / ausente
\end{itemize}
\end{quote}

\begin{quote}
\begin{itemize}
\tightlist
\item[$\square$]
  Flacidez: snapback / distraction (\textgreater{} 6 mm sugere
  frouxidão)
\end{itemize}
\end{quote}

\begin{quote}
\begin{itemize}
\tightlist
\item[$\square$]
  Assimetrias prévias documentadas (sobrancelha e fenda)
\end{itemize}
\end{quote}

\begin{quote}
\begin{itemize}
\tightlist
\item[$\square$]
  Teste da lamela anterior (pinch test) para ressecção conservadora
\end{itemize}
\end{quote}

\section{Anatomia aplicada (apenas o que muda
conduta)}\label{anatomia-aplicada-apenas-o-que-muda-conduta-3}

\begin{itemize}
\item
  \textbf{Músculo oblíquo inferior:} entre bolsa medial e central na
  pálpebra inferior; estrutura crítica que define limite seguro ao
  tratar gordura.
\item
  \textbf{Ramo temporo-frontal do facial:} corre em planos superficiais
  da região temporoparietal; segurança depende de plano correto e
  respeito aos marcos.
\item
  \textbf{Forames supraorbitários:} referência para proteger
  sensibilidade frontal durante liberação/abordagem glabelar.
\item
  Septo orbitário: fronteira anatômica; sua manipulação e fibrose podem
  influenciar estabilidade e retração.
\end{itemize}

\begin{quote}
\textbf{📎 FIGURA NECESSÁRIA (Cap. 05):}
\end{quote}

\begin{quote}
Mapa de risco cirúrgico: Zonas verdes (seguras), amarelas (atenção),
vermelhas (perigo)
\end{quote}

\begin{quote}
\emph{Estilo: Diagrama técnico-didático, cores neutras, legendas claras}
\end{quote}

\section{Técnica (visão geral +
variações)}\label{tuxe9cnica-visuxe3o-geral-variauxe7uxf5es-1}

(Se este capítulo for ``Princípios e Segurança'', o passo-a-passo pode
ser resumido aqui e o detalhamento completo fica nos capítulos
técnicos.)

\subsection{Visão geral}\label{visuxe3o-geral-1}

\begin{itemize}
\item
  Identificação externa de forames e trajetos nervosos
\item
  Infiltração/hidrodissecção para facilitar separação de planos (quando
  aplicável)
\item
  Dissecção com hemostasia rigorosa para visualização segura de
  estruturas nobres {[}{[}REF:ROHRICH-2008{]}{]}
\item
  Preservação da lamela posterior como arcabouço de suporte
\end{itemize}

\subsection{Variações e
indicações}\label{variauxe7uxf5es-e-indicauxe7uxf5es-1}

\begin{itemize}
\item
  \textbf{Transconjuntival (pós-septal):} preserva lamela anterior; útil
  para reduzir risco de retração em perfis selecionados
\item
  \textbf{Subciliar com retalho miocutâneo:} quando há excesso real de
  pele/tratamento de festoons, geralmente exigindo plano de suporte
  lateral
\end{itemize}

\section{Erros comuns (e como
resgatar)}\label{erros-comuns-e-como-resgatar-1}

\begin{itemize}
\item
  \textbf{Lesão do oblíquo inferior} → diplopia → prevenção:
  identificação ativa antes de tratar bolsas medial/central → resgate:
  avaliação especializada (estrabismo).
\item
  \textbf{Ressecção agressiva de pele em vetor negativo} →
  ectrópio/exposição → prevenção: ressecção conservadora +
  suporte/volume conforme indicação → resgate: enxerto/retalhos conforme
  necessidade.
\item
  \textbf{Plano errado na região temporal} → déficit motor/ptose de
  sobrancelha → prevenção: plano correto + marcos + dissecção controlada
  → resgate: estratégias tardias de simetria (variável conforme caso).
\end{itemize}

\begin{quote}
\textbf{PÉROLA CLÍNICA}
\end{quote}

\begin{quote}
\mbox{}%
\section{Checklist de 60s --- Mnemônico de Segurança
(V-B-O)}\label{checklist-de-60s-mnemuxf4nico-de-seguranuxe7a-v-b-o}
\end{quote}

\begin{quote}
\begin{enumerate}
\def\labelenumi{\arabic{enumi})}
\tightlist
\item
  \textbf{V}etor: se negativo, pele é ``ouro'' --- ressecção mínima e
  suporte bem pensado.
\end{enumerate}
\end{quote}

\begin{quote}
\begin{enumerate}
\def\labelenumi{\arabic{enumi})}
\setcounter{enumi}{1}
\tightlist
\item
  \textbf{B}ell: se ruim, prioridade é fechar a pálpebra com folga.
\end{enumerate}
\end{quote}

\begin{quote}
\begin{enumerate}
\def\labelenumi{\arabic{enumi})}
\setcounter{enumi}{2}
\tightlist
\item
  \textbf{O}blíquo: se mexer em gordura inferior, identifique o músculo
  antes.
\end{enumerate}
\end{quote}

\begin{quote}
\textbf{PÉROLA CLÍNICA}
\end{quote}

\begin{quote}
\mbox{}%
\section{Zona de risco --- Septo orbitário (alerta de
plano)}\label{zona-de-risco-septo-orbituxe1rio-alerta-de-plano}
\end{quote}

\begin{quote}
Evite manipulação desnecessária do septo orbitário quando o objetivo é
limitado. Exposição e trauma sem indicação podem aumentar risco de
fibrose e alterar dinâmica palpebral.
\end{quote}

\section{Notas de ``arte''}\label{notas-de-arte-1}

Preservar unidades estéticas exige respeitar \textbf{zonas de
transição}. Uma pálpebra ``bem operada'' não é a mais esticada: é a que
mantém \textbf{continuidade de luz} entre rebordo e bochecha. Erros de
plano criam degraus de sombra que denunciam o procedimento.

\section{Pós-operatório e
follow-up}\label{puxf3s-operatuxf3rio-e-follow-up-2}

(Se você quer um capítulo único de pós/complicações, mova tudo para lá.)

\begin{itemize}
\item
  \textbf{24--72h:} acuidade visual e motilidade (excluir emergência)
\item
  \textbf{7--14d:} posição do canto lateral
\item
  \textbf{Alarme:} dor intensa + proptose + baixa visual = emergência
\end{itemize}

\section{Referências / leituras
recomendadas}\label{referuxeancias-leituras-recomendadas-1}

\begin{itemize}
\item
  {[}{[}REF:MCCORD-1995{]}{]} McCord --- Eyelid Surgery
\item
  {[}{[}REF:MCCORD-1995{]}{]} Codner --- Periorbital Safety Map
\item
  {[}{[}REF:ZIDE-1985{]}{]} Zide --- Surgical Anatomy of the Orbit
\end{itemize}

\begin{center}\rule{0.5\linewidth}{0.5pt}\end{center}

\chapter{Capítulo 06 --- Checklist mental do resultado ``nota 10''
(princípios
replicáveis)}\label{capuxedtulo-06-checklist-mental-do-resultado-nota-10-princuxedpios-replicuxe1veis}

\textbf{Parte:} Parte 0 --- Princípios e Segurança

\section{Objetivo do capítulo}\label{objetivo-do-capuxedtulo-4}

Ao final deste capítulo, o leitor será capaz de internalizar um
protocolo de avaliação crítica que separa resultados puramente técnicos
de resultados esteticamente superiores, priorizando naturalidade e
segurança funcional.

\section{O que muda na decisão (o
``porquê'')}\label{o-que-muda-na-decisuxe3o-o-porquuxea-5}

\begin{itemize}
\item
  \textbf{O vetor orbitofacial:} Identificar vetor negativo altera a
  estratégia de suporte; ignorar a projeção malar em relação ao globo
  tende a aumentar risco de retração palpebral ou exposição escleral
  indesejada.
\item
  \textbf{A preservação do volume:} A transição entre ``olhar jovem'' e
  ``olhar operado'' frequentemente reside no volume; ressecção agressiva
  de bolsas aumenta risco de sulco profundo e esqueletização.
\item
  \textbf{A harmonia da fenda:} Focar apenas em pele negligencia a
  posição da pálpebra inferior (MRD2); um resultado ``nota 10''
  geralmente exige margem palpebral inferior bem posicionada, sem
  aparência de ``olho aberto''/exposição.
\end{itemize}

\begin{quote}
\textbf{PÉROLA CLÍNICA}
\end{quote}

\begin{quote}
\mbox{}%
\subsection{Checklist de 60 segundos
(pré-op)}\label{checklist-de-60-segundos-pruxe9-op}
\end{quote}

\begin{quote}
\begin{itemize}
\tightlist
\item[$\square$]
  \textbf{Vetor:} positivo, neutro ou negativo?
\end{itemize}
\end{quote}

\begin{quote}
\begin{itemize}
\tightlist
\item[$\square$]
  \textbf{Snap-back test:} retorno imediato ou lento?
\end{itemize}
\end{quote}

\begin{quote}
\begin{itemize}
\tightlist
\item[$\square$]
  \textbf{Distraction test:} \textgreater{} 6 mm sugere frouxidão
  tendínea relevante
\end{itemize}
\end{quote}

\begin{quote}
\begin{itemize}
\tightlist
\item[$\square$]
  Sulco lacrimal: profundo (pede volume/transição) ou raso?
\end{itemize}
\end{quote}

\begin{quote}
\begin{itemize}
\tightlist
\item[$\square$]
  Posição do supercílio: há compensação frontal ativa?
\end{itemize}
\end{quote}

\begin{quote}
\begin{itemize}
\tightlist
\item[$\square$]
  Simetria da fenda: MRD1 e MRD2 documentados
\end{itemize}
\end{quote}

\section{Aplicação prática}\label{aplicauxe7uxe3o-pruxe1tica-3}

\textbf{Aplicar quando:} - houver queixa de ``olhar cansado'' sem
excesso de pele aparente (suspeitar de volume/transição);

\begin{itemize}
\item
  pacientes com olhos proeminentes ou hipoplasia malar (vetor negativo);
\item
  assimetrias dinâmicas visíveis durante fala/sorriso;
\item
  revisão de cirurgias prévias em que o ``menos'' foi ignorado.
\end{itemize}

\textbf{Cautela quando:} - houver olho seco importante ou teste de
Schirmer limítrofe (prioridade funcional);

\begin{itemize}
\item
  expectativas irreais de eliminar rugas dinâmicas (pés de galinha), que
  tendem a persistir;
\item
  edema malar crônico/festoons, que geralmente exigem lógica diferente
  da blefaroplastia clássica.
\end{itemize}

\section{Anatomia aplicada (o que muda a
conduta)}\label{anatomia-aplicada-o-que-muda-a-conduta}

\begin{itemize}
\item
  \textbf{Vetor malar-orbital:} relação globo × suporte infraorbitário.
  Em vetor negativo, a pálpebra carece de suporte ósseo; isso
  frequentemente exige plano de sustentação lateral mais robusto.
\item
  \textbf{Septo orbitário:} barreira anatômica; manipulação
  excessiva/fibrose pode aumentar risco de retração.
\item
  \textbf{Coxins de gordura (SOOF/ROOF e correlatos):} a forma de
  preservar/redistribuir/volumizar (quando indicado) define a transição
  entre unidades estéticas.
\item
  Zona de risco (princípio): estruturas do canto medial (canalículos) e
  trajetos nervosos temporais; a regra é evitar manobras ``às cegas'' e
  respeitar planos.
\end{itemize}

\begin{quote}
\textbf{PÉROLA CLÍNICA}
\end{quote}

\begin{quote}
\mbox{}%
\subsection{Regra prática: a ``lei da conservação''
(anti-estigma)}\label{regra-pruxe1tica-a-lei-da-conservauxe7uxe3o-anti-estigma}
\end{quote}

\begin{quote}
Em cirurgia palpebral, tecido removido é difícil de recuperar. Na
dúvida, a preservação conservadora de pele e volume tende a reduzir
risco de estigma do ``olho operado''. Quando indicado,
reposicionamento/redistribuição e/ou gordura autóloga podem ser
preferíveis à subtração pura.
\end{quote}

\section{Método de raciocínio}\label{muxe9todo-de-raciocuxednio}

\begin{enumerate}
\def\labelenumi{\arabic{enumi}.}
\item
  \textbf{Diagnóstico da unidade:} avaliar a pálpebra como parte das
  transições pálpebra-supercílio e pálpebra-bochecha.
\item
  \textbf{Mapeamento de sombras:} localizar onde a luz ``quebra'';
  objetivo é transformar vales (sombras) em curvas contínuas.
\item
  \textbf{Definição do suporte:} decidir se lamela posterior + canto
  lateral suportam a tensão planejada.
\item
  Planejamento de volume: definir se o déficit é de tecido (pele) ou de
  estrutura (volume/suporte).
\end{enumerate}

\section{Erros comuns e resgate}\label{erros-comuns-e-resgate}

\begin{itemize}
\item
  \textbf{Erro: ressecção excessiva de gordura inferior}
\item
  \textbf{Consequência:} olhar encovado e esqueletizado.
\item
  \textbf{Prevenção:} julgamento conservador de volume; evitar
  ``perseguir planura'' quando a sombra é de deflação.
\item
  Resgate: lipoenxertia (microfat) em plano apropriado ou preenchimento
  profundo quando indicado.
\item
  \textbf{Erro: ignorar vetor negativo}
\item
  \textbf{Consequência:} escleral show e ectrópio progressivo.
\item
  \textbf{Prevenção:} planejar sustentação lateral e ressecção mínima de
  pele conforme risco.
\item
  Resgate: suspensão lateral/correção conforme necessidade; em casos
  selecionados, enxerto de pele e/ou espaçador posterior.
\end{itemize}

\section{Notas de arte}\label{notas-de-arte-2}

A estética ``nota 10'' depende da \textbf{luz}. Uma pálpebra jovem não
tem interrupções abruptas na transição para a face. O cirurgião atua
como escultor: suaviza o ``degrau'' do rebordo orbitário inferior e
preserva a leitura contínua da curva (Ogee), especialmente em visão 3/4.

\section{Referências}\label{referuxeancias}

\begin{itemize}
\item
  {[}{[}REF:FLOWERS-1993{]}{]} Flowers --- conceitos de vetores orbitais
  e suporte palpebral.
\item
  {[}{[}REF:HAMRA-1995{]}{]} Hamra --- preservação/reposicionamento e
  lógica de transições.
\item
  {[}{[}REF:PUTTERMAN-1975{]}{]} Putterman --- dinâmica palpebral e
  medidas (MRD).
\end{itemize}

\section{Próximo capítulo (sugestão de
escopo)}\label{pruxf3ximo-capuxedtulo-sugestuxe3o-de-escopo}

Detalhar tecnicamente reposicionamento/redistribuição de gordura
(transconjuntival vs transcutânea), com: indicação por fenótipo
(vetor/deflação/festoons), passos macro, riscos e checkpoints de
segurança.

\begin{center}\rule{0.5\linewidth}{0.5pt}\end{center}

\chapter{Capítulo 07 --- Fotodocumentação estratégica: sem flash, com
flash e
padronização}\label{capuxedtulo-07-fotodocumentauxe7uxe3o-estratuxe9gica-sem-flash-com-flash-e-padronizauxe7uxe3o}

\begin{figure}
\centering
\pandocbounded{\includegraphics[keepaspectratio,alt={Figura\,07.1 --- Fotodocumentação estratégica}]{/Users/humbertolopes/Dev/work/marcelo-cury/the_art_of_eyelid_surgery_scaffold/projects/eyelid-surgery/assets/figures/FIG-07-01_fotodocumentacao-sem-flash-com-flash-padronizada.png}}
\caption{Figura\,07.1 --- Fotodocumentação estratégica}
\end{figure}

\textbf{Parte:} Parte I --- Diagnóstico

\section{Objetivo do capítulo}\label{objetivo-do-capuxedtulo-5}

Ao final, o leitor saberá padronizar a captura de imagens para revelar a
verdadeira topografia periorbital (bolsas, sulcos e textura), documentar
assimetrias prévias e reduzir ruído de expectativa --- tanto no
planejamento cirúrgico quanto na comunicação com o paciente.

\section{O que muda na decisão (o
``porquê'')}\label{o-que-muda-na-decisuxe3o-o-porquuxea-6}

\begin{itemize}
\item
  \textbf{A ``mentira'' do flash frontal:} usar apenas flash tende a
  achatar relevo e suavizar sombras, mascarando bolsas, festoons e a
  profundidade real do sulco lacrimal (tear trough). A luz
  ambiente/superior costuma revelar melhor o relevo que o paciente
  percebe no espelho.
\item
  \textbf{O risco da assimetria não vista:} pacientes frequentemente não
  percebem assimetrias discretas (ptose leve, distopia de canto, sulcos
  desiguais). Se não estiver fotografado e apontado antes, vira
  atribuição pós-operatória.
\item
  \textbf{Erro de paralaxe:} fotos com câmera acima/abaixo do plano
  ocular alteram falsamente MRD2, relação esclera-limbo e leitura de
  exposição, induzindo erro de planejamento.
\end{itemize}

\section{Indicações e
contra-indicações}\label{indicauxe7uxf5es-e-contra-indicauxe7uxf5es}

\textbf{Indicar quando:} - na prática, em praticamente todos os casos de
cirurgia palpebral (funcional ou estética);

\begin{itemize}
\item
  consultas de retorno, para monitorar edema, cicatriz e dinâmica;
\item
  planejamento de reposição/reposicionamento de volume (sombra e
  transição precisam ser visíveis).
\end{itemize}

\textbf{Evitar / adiar quando:} - paciente estiver maquiada (remover
completamente para avaliar cor, textura e vascularização);

\begin{itemize}
\tightlist
\item
  ambientes com iluminação mista incontrolável (luz natural intensa +
  luz artificial), que distorce cor e balanço de branco.
\end{itemize}

\begin{quote}
\textbf{PÉROLA CLÍNICA}
\end{quote}

\begin{quote}
\mbox{}%
\subsection{Checklist de configuração
(pré-op)}\label{checklist-de-configurauxe7uxe3o-pruxe9-op}
\end{quote}

\begin{quote}
\begin{itemize}
\tightlist
\item[$\square$]
  \textbf{Foto SEM flash:} essencial para captar sombras, bolsas,
  festoons e profundidade do sulco nasojugal.
\end{itemize}
\end{quote}

\begin{quote}
\begin{itemize}
\tightlist
\item[$\square$]
  \textbf{Foto COM flash:} útil para documentar discromias, textura e
  reflexo pupilar (corneal light reflex), além de facilitar comparação.
\end{itemize}
\end{quote}

\begin{quote}
\begin{itemize}
\tightlist
\item[$\square$]
  \textbf{Plano de Frankfurt:} câmera na altura dos olhos, lente
  paralela ao chão (evitar \emph{chin-up}/\emph{chin-down}).
\end{itemize}
\end{quote}

\begin{quote}
\begin{itemize}
\tightlist
\item[$\square$]
  Padronização do olhar: primário (frente), supraversão (olhar para
  cima), fechamento suave, sorriso (pés de galinha/dinâmica).
\end{itemize}
\end{quote}

\begin{quote}
\begin{itemize}
\tightlist
\item[$\square$]
  Distância focal fixa: preferir 85--100mm (ou equivalente) para reduzir
  distorções; evitar grande-angular curta.
\end{itemize}
\end{quote}

\begin{quote}
\begin{itemize}
\tightlist
\item[$\square$]
  Cabelo preso: expor fronte e têmporas para avaliar cauda da
  sobrancelha e moldura lateral.
\end{itemize}
\end{quote}

\section{Anatomia aplicada (o que a luz
revela)}\label{anatomia-aplicada-o-que-a-luz-revela}

\begin{itemize}
\item
  \textbf{Sulco nasojugal (tear trough):} sob flash frontal pode
  ``sumir''; sob luz ambiente/superior, a sombra demarcada pode sugerir
  necessidade de estratégia para transição (liberação ligamentar e/ou
  volume, conforme indicação).
\item
  \textbf{Bolsas de gordura (septo):} a proeminência real costuma
  aparecer melhor com iluminação incidente lateral ou superior
  (\emph{top-down lighting}).
\item
  \textbf{Zona de risco visual (scleral show):} a foto de perfil ajuda a
  documentar a relação globo--malar (vetor). Se a córnea estiver
  claramente à frente do suporte malar, registra-se risco aumentado de
  exposição/retração com estratégias subtrativas.
\end{itemize}

\begin{quote}
\textbf{📎 FIGURA NECESSÁRIA (Cap. 07):}
\end{quote}

\begin{quote}
Comparativo: Foto sem flash (revela sombras/volumes) vs com flash
(achata)
\end{quote}

\begin{quote}
\emph{Estilo: Diagrama técnico-didático, cores neutras, legendas claras}
\end{quote}

\section{Técnica de captura (visão
geral)}\label{tuxe9cnica-de-captura-visuxe3o-geral}

\subsection{Visão geral}\label{visuxe3o-geral-2}

\begin{itemize}
\item
  \textbf{Fundo:} preferir azul médio, cinza ou preto fosco (reduz
  contaminação de cor e reflexos).
\item
  \textbf{Posicionamento:} paciente sentado, coluna reta, cabeça neutra;
  operador no mesmo nível dos olhos.
\item
  \textbf{Sequência de 5 ângulos:} frontal; oblíqua D/E (45°); perfil
  D/E (90°).
\item
  Sequência dinâmica: repouso; sorriso; fechamento passivo; olhar para
  cima.
\item
  Diagnóstico de falha: se a foto pré-op parece ``melhor'' que o
  paciente ao vivo, a iluminação provavelmente foi benevolente demais
  (flash frontal e/ou ângulo errado).
\end{itemize}

\subsection{Variações e
indicações}\label{variauxe7uxf5es-e-indicauxe7uxf5es-2}

\begin{itemize}
\item
  \textbf{Macro de cicatriz:} luz rasante/lateral para evidenciar relevo
  em revisões.
\item
  \textbf{Comparativos seriados:} repetir ângulo, distância focal e
  iluminação nas consultas de retorno para comparação honesta.
\end{itemize}

\section{Nota sobre selfies (gestão de
expectativa)}\label{nota-sobre-selfies-gestuxe3o-de-expectativa}

Selfies de celular (lente grande-angular curta) distorcem proporções e
podem alterar a leitura das pálpebras e do terço médio. Não devem ser
usadas como parâmetro médico. Quando o paciente traz selfies, o papel da
consulta é explicar distorção de lente/ângulo e reconduzir a comparação
para documentação padronizada.

\begin{quote}
\textbf{PÉROLA CLÍNICA}
\end{quote}

\begin{quote}
\mbox{}%
\subsection{\texorpdfstring{Erro ``nota 7'': a foto
\emph{chin-up}}{Erro ``nota 7'': a foto chin-up}}\label{erro-nota-7-a-foto-chin-up}
\end{quote}

\begin{quote}
Muitos pacientes levantam o queixo ao serem fotografados para disfarçar
papada/flacidez.
\end{quote}

\begin{quote}
\textbf{Problema:} isso muda a relação globo--pálpebra e pode simular
ptose, mascarar exposição ou alterar a leitura de MRD2.
\end{quote}

\begin{quote}
\textbf{Correção:} alinhar cabeça no plano neutro; manter referência
consistente (Plano de Frankfurt) e câmera paralela ao chão.
\end{quote}

\section{Erros comuns (e como
resgatar)}\label{erros-comuns-e-como-resgatar-2}

\begin{itemize}
\item
  Erro: lente grande-angular (celular / \textless{} 50mm equivalente)
\item
  \textbf{Consequência:} distorção facial (nariz maior, orelhas
  ``somem'', face estreita), alterando percepção estética e medidas
  aparentes.
\item
  \textbf{Prevenção:} usar 85--100mm (ou equivalente) e distância
  constante; em celular, preferir zoom óptico/tele (2x/3x).
\item
  \textbf{Resgate:} refazer fotos; não planejar com base em imagem
  distorcida.
\item
  \textbf{Erro: iluminação assimétrica}
\item
  \textbf{Consequência:} uma olheira/sulco parece mais fundo apenas por
  sombra da sala.
\item
  \textbf{Prevenção:} padronizar fontes (bilaterais equidistantes) e
  evitar luz mista.
\item
  Resgate: se o registro antigo é ruim, apoiar decisão em exame físico
  bem descrito e refazer documentação.
\end{itemize}

\section{Notas de ``arte'' (luz/sombra)}\label{notas-de-arte-luzsombra}

A fotografia médica deve capturar \textbf{transições}, não
``embelezar''. O objetivo da cirurgia é reduzir o degrau (sombra dura)
entre pálpebra e malar. A foto sem flash tende a ser o juiz mais severo
dessa transição. Documentação forte é a que reproduz a leitura que o
paciente tem no cotidiano.

\section{Referências}\label{referuxeancias-1}

\begin{itemize}
\item
  {[}{[}REF:GUNTER-2007{]}{]} Padrões de fotografia médica em cirurgia
  plástica facial (ex.: Gunter / Rohrich).
\item
  {[}{[}REF:HIRMAND-2010{]}{]} Iluminação padronizada e leitura do sulco
  lacrimal (ex.: Hirmand).
\end{itemize}

\section{Próximo capítulo
(sugestão)}\label{pruxf3ximo-capuxedtulo-sugestuxe3o}

Detalhar ``Técnica de Marcação Cirúrgica'' (superior vs inferior),
enfatizando: marcação conservadora, risco por vetor/suporte, checkpoints
de simetria (MRD1/MRD2) e como evitar ressecção excessiva.

\begin{center}\rule{0.5\linewidth}{0.5pt}\end{center}

\chapter{Capítulo 08 --- Exame físico: vetores, flacidez, testes e
assimetrias}\label{capuxedtulo-08-exame-fuxedsico-vetores-flacidez-testes-e-assimetrias}

\begin{figure}
\centering
\pandocbounded{\includegraphics[keepaspectratio,alt={Figura 08.1 --- Ilustração principal do capítulo}]{/Users/humbertolopes/Dev/work/marcelo-cury/the_art_of_eyelid_surgery_scaffold/projects/eyelid-surgery/assets/figures/FIG-08-01_vetores-testes-assimetrias.png}}
\caption{Figura 08.1 --- Ilustração principal do capítulo}
\end{figure}

\textbf{Parte:} Parte I --- Diagnóstico

\section{Objetivo do capítulo}\label{objetivo-do-capuxedtulo-6}

Ao final, o leitor saberá executar um protocolo de avaliação biomecânica
e funcional que orienta a escolha entre abordagem subtrativa
(pele/gordura) versus necessidade de sustentação/estruturação, reduzindo
risco de complicações como ectrópio, lagoftalmo e exposição corneana.

\section{O que muda na decisão (o
``porquê'')}\label{o-que-muda-na-decisuxe3o-o-porquuxea-7}

\begin{itemize}
\item
  \textbf{O ``veto'' do vetor negativo:} identificar vetor negativo
  (globo mais proeminente que o suporte malar) frequentemente
  contraindica a blefaroplastia inferior ``padrão'' subtrativa. Em geral
  exige sustentação cantal e/ou estratégia de volume, sob maior risco de
  \emph{scleral show} e retração.
\item
  \textbf{A armadilha da ptose latente (Hering):} não testar a Lei de
  Hering é uma causa comum de ``surpresa'' pós-operatória (ptose
  contralateral aparente após correção do lado mais caído).
\item
  \textbf{Frouxidão horizontal subestimada:} \emph{distraction}
  \textgreater{} 6 mm e/ou \emph{snap-back} lento sugerem baixa
  tolerância a retirada de pele sem suporte (cantopexia/cantoplastia
  conforme caso).
\end{itemize}

\begin{quote}
\textbf{PÉROLA CLÍNICA}
\end{quote}

\begin{quote}
\mbox{}%
\subsection{Checklist de segurança
(pré-op)}\label{checklist-de-seguranuxe7a-pruxe9-op}
\end{quote}

\begin{quote}
\begin{itemize}
\tightlist
\item[$\square$]
  \textbf{Vetor orbital:} positivo / neutro / negativo
\end{itemize}
\end{quote}

\begin{quote}
\begin{itemize}
\tightlist
\item[$\square$]
  \textbf{Distraction test:} \textgreater{} 6 mm sugere frouxidão
  tendínea relevante
\end{itemize}
\end{quote}

\begin{quote}
\begin{itemize}
\tightlist
\item[$\square$]
  \textbf{Snap-back:} retorno imediato vs lento/incompleto (risco de
  ectrópio)
\end{itemize}
\end{quote}

\begin{quote}
\begin{itemize}
\tightlist
\item[$\square$]
  Lei de Hering: ocluir o olho ``pior'' e observar queda do
  contralateral
\end{itemize}
\end{quote}

\begin{quote}
\begin{itemize}
\tightlist
\item[$\square$]
  MRD1: referência clínica em geral \textasciitilde4--5 mm (interpretar
  no contexto)
\end{itemize}
\end{quote}

\begin{quote}
\begin{itemize}
\tightlist
\item[$\square$]
  MRD2: referência clínica em geral \textasciitilde5 mm (se maior,
  avaliar retração/exposição prévia)
\end{itemize}
\end{quote}

\begin{quote}
\begin{itemize}
\tightlist
\item[$\square$]
  Fenômeno de Bell: presente / ausente
\end{itemize}
\end{quote}

\begin{quote}
\begin{itemize}
\tightlist
\item[$\square$]
  Schirmer I (se indicado): 5 min sem anestesia (história de olho seco)
\end{itemize}
\end{quote}

\section{Indicações e
contra-indicações}\label{indicauxe7uxf5es-e-contra-indicauxe7uxf5es-1}

\textbf{Indicar quando:} - rotina para qualquer candidato a cirurgia
periorbital;

\begin{itemize}
\item
  suspeita de ptose (uni ou bilateral) ou assimetria de fenda;
\item
  pacientes idosos (perda de tônus cantal é frequente).
\end{itemize}

\textbf{Cautela quando:} - \textbf{Fenômeno de Bell ausente/fraco:}
aumenta risco de ceratite de exposição; exige conservadorismo em
ressecções e respeito à dinâmica de fechamento.

\begin{itemize}
\tightlist
\item
  \textbf{Olho seco importante (ex.: Schirmer muito baixo):} prioridade
  funcional; evitar agressividade, especialmente em ressecções e
  manipulação do orbicular.
\end{itemize}

\section{Anatomia aplicada (mecânica e
suporte)}\label{anatomia-aplicada-mecuxe2nica-e-suporte}

\begin{itemize}
\item
  \textbf{Tendão cantal lateral (LCT):} âncora primária de sustentação.
  Com idade, tende a alongar/afrouxar, reduzindo estabilidade da margem
  e tolerância à tensão cicatricial.
\item
  \textbf{Complexo malar (suporte):} funciona como ``prateleira''
  infraorbitária. Em hipoplasia malar (vetor negativo), o suporte ósseo
  é menor e a pálpebra depende mais do canto lateral.
\item
  \textbf{Músculo de Müller:} componente simpático; ansiedade/ambiente
  do consultório pode ``falsear'' altura palpebral, mascarando ptose
  discreta.
\end{itemize}

\begin{quote}
\textbf{📎 FIGURA NECESSÁRIA (Cap. 08):}
\end{quote}

\begin{quote}
Diagrama: Vetor positivo vs neutro vs negativo --- relação globo/malar
\end{quote}

\begin{quote}
\emph{Estilo: Diagrama técnico-didático, cores neutras, legendas claras}
\end{quote}

\section{Técnica de exame (raciocínio
clínico)}\label{tuxe9cnica-de-exame-raciocuxednio-cluxednico}

\subsection{Visão geral}\label{visuxe3o-geral-3}

\begin{itemize}
\item
  \textbf{1) Análise de vetor (perfil)}
\item
  Método simples: alinhar um objeto reto (ex.: abaixador) tangenciando
  malar e referência orbitária; observar relação córnea--suporte malar.
\item
  Se a córnea projeta claramente à frente do suporte malar, documentar
  como vetor negativo (risco aumentado para abordagem subtrativa).
\item
  \textbf{2) Testes de tônus e flacidez (frontal)}
\item
  \textbf{Distraction:} tracionar pálpebra inferior; afastamento
  aumentado sugere frouxidão.
\item
  Snap-back: soltar pálpebra; retorno deve ser imediato sem necessidade
  de piscar. Retorno lento/incompleto reforça necessidade de suporte.
\item
  \textbf{3) Avaliação de ptose (frontal)}
\item
  Medir \textbf{MRD1} e documentar simetria.
\item
  Aplicar \textbf{Hering:} elevar/ocluir o lado mais ptótico e observar
  queda do contralateral.
\item
  \textbf{4) Assimetrias estruturais}
\item
  Documentar sobrancelha, sulco palpebral, altura pupilar, canto
  medial/lateral e dominância de expressão (fala/sorriso).
\end{itemize}

\subsection{Variações e
indicações}\label{variauxe7uxf5es-e-indicauxe7uxf5es-3}

\begin{itemize}
\tightlist
\item
  \textbf{Teste do ``squinch'' (contração forçada):} diferencia
  hipertrofia do orbicular pré-tarsal (\emph{roll}) de bolsas de
  gordura.
\item
  Se o volume aumenta com contração, tende a ser músculo.
\item
  Se mantém, tende a ser gordura/compartimento.
\end{itemize}

\begin{quote}
\textbf{PÉROLA CLÍNICA}
\end{quote}

\begin{quote}
\mbox{}%
\subsection{Zona de risco: ``olho redondo''
iatrogênico}\label{zona-de-risco-olho-redondo-iatroguxeanico}
\end{quote}

\begin{quote}
Ignorar \emph{snap-back} lento e remover pele na pálpebra inferior
aumenta risco de \emph{round eye} (arredondamento do canto lateral),
exposição escleral e ectrópio.
\end{quote}

\begin{quote}
\textbf{Regra prática:} com retorno lento, a lógica tende a ser
\textbf{suporte primeiro} (pexia/estruturas) e ressecção mínima --- não
``cortar para resolver''.
\end{quote}

\section{Erros comuns (e como
resgatar)}\label{erros-comuns-e-como-resgatar-3}

\begin{itemize}
\item
  \textbf{Erro: não diagnosticar ptose assimétrica (Hering)}
\item
  \textbf{Consequência:} corrige-se a pálpebra ``pior'' e, dias depois,
  a ``boa'' aparenta cair.
\item
  \textbf{Prevenção:} testar Hering sistematicamente e documentar.
\item
  Resgate: correção contralateral (levator/Müller conforme indicação),
  após estabilização.
\item
  \textbf{Erro: ignorar assimetria esquelética}
\item
  \textbf{Consequência:} paciente atribui diferença pós-op à cirurgia
  quando a base anatômica já era assimétrica.
\item
  \textbf{Prevenção:} documentar e mostrar antes (espelho + fotos
  padronizadas).
\item
  Resgate: gestão de expectativa; camuflagem com volume quando
  apropriado (ou não intervir, se risco \textgreater{} benefício).
\end{itemize}

\section{Notas de ``arte''
(luz/sombra)}\label{notas-de-arte-luzsombra-1}

A beleza do olhar depende de \textbf{simetria perceptiva}, não de
simetria milimétrica. Parte do exame é mapear o que é estrutural
(osso/suporte) versus o que é ajustável (sulco, volume, sustentação),
evitando promessas de ``igualdade matemática''.

\section{Referências}\label{referuxeancias-2}

\begin{itemize}
\item
  {[}{[}REF:JELKS-1993{]}{]} Jelks --- vetor negativo e implicações na
  pálpebra inferior
\item
  {[}{[}REF:BODIAN-1982{]}{]} Bodian --- Lei de Hering aplicada à
  ptose/blefaro
\item
  {[}{[}REF:FAGIEN-1999{]}{]} Fagien --- frouxidão horizontal e escolha
  de suporte lateral
\end{itemize}

\section{Próximo capítulo
(sugestão)}\label{pruxf3ximo-capuxedtulo-sugestuxe3o-1}

Escolher entre:

\begin{itemize}
\item
  Cap. 09 --- Anestesia e segurança de vasoconstrição (quando usar,
  quando evitar, checklist e complicações), ou
\item
  Iniciar técnica cirúrgica: Blefaroplastia Superior (marcação,
  preservação de volume, armadilhas e resgates).
\end{itemize}

\begin{center}\rule{0.5\linewidth}{0.5pt}\end{center}

\chapter{Capítulo 09 --- Consulta e expectativa: alinhar pedido do
paciente com necessidade
anatômica}\label{capuxedtulo-09-consulta-e-expectativa-alinhar-pedido-do-paciente-com-necessidade-anatuxf4mica}

\begin{figure}
\centering
\pandocbounded{\includegraphics[keepaspectratio,alt={Figura 09.1 --- Ilustração principal do capítulo}]{/Users/humbertolopes/Dev/work/marcelo-cury/the_art_of_eyelid_surgery_scaffold/projects/eyelid-surgery/assets/figures/FIG-09-01_consulta-pedido-anatomia-plano.png}}
\caption{Figura 09.1 --- Ilustração principal do capítulo}
\end{figure}

\textbf{Parte:} Parte I --- Diagnóstico

\section{Objetivo do capítulo}\label{objetivo-do-capuxedtulo-7}

Ao final, o leitor saberá filtrar candidatos cirúrgicos, traduzindo
queixas subjetivas (``olhar triste'', ``olheira'', ``pele demais'') em
hipóteses anatômicas testáveis, e reconhecer sinais de alerta
psicológicos/comportamentais que sugerem adiar, recusar ou encaminhar.

\section{O que muda na decisão (o
``porquê'')}\label{o-que-muda-na-decisuxe3o-o-porquuxea-8}

\begin{itemize}
\item
  \textbf{A falácia da ``apenas uma pelezinha'':} pacientes
  frequentemente apontam excesso cutâneo quando a causa dominante é
  ptose de supercílio. Operar só pele sem tratar/considerar a moldura
  frontal pode aproximar a sobrancelha dos cílios, piorar a leitura
  lateral e gerar insatisfação.
\item
  \textbf{O risco de percepção distorcida:} sinais compatíveis com
  transtorno dismórfico corporal (TDC) e/ou expectativas rígidas são
  fator de risco para conflito e insatisfação, independentemente do
  resultado técnico. A decisão clínica inclui proteger o paciente e o
  cirurgião.
\item
  \textbf{Assimetria pré-existente ``invisível'':} o cérebro do paciente
  se habitua a assimetrias. Se não for documentado e demonstrado antes,
  tende a ser percebido como ``erro'' depois.
\end{itemize}

\section{Indicações e
contra-indicações}\label{indicauxe7uxf5es-e-contra-indicauxe7uxf5es-2}

\textbf{Indicar quando:} - a queixa anatômica coincide com o exame
físico (ex.: bolsa referida + herniação real);

\begin{itemize}
\item
  o paciente aceita o objetivo como ``melhora'' e entende limites;
\item
  motivação é interna (desejo próprio), com compreensão do
  pós-operatório.
\end{itemize}

\textbf{Evitar / adiar quando:} - expectativa irreal (referência
estética incompatível com estrutura óssea/pele);

\begin{itemize}
\item
  crise aguda (luto, divórcio recente, depressão descompensada) ---
  cirurgia não é intervenção emocional;
\item
  padrão de ``doctor shopping'' e discurso agressivo/desqualificador
  recorrente.
\end{itemize}

\begin{quote}
\textbf{PÉROLA CLÍNICA}
\end{quote}

\begin{quote}
\mbox{}%
\subsection{Checklist de ``red flags'' (comportamento e
expectativa)}\label{checklist-de-red-flags-comportamento-e-expectativa}
\end{quote}

\begin{quote}
\begin{itemize}
\tightlist
\item[$\square$]
  \textbf{Foco excessivo em defeito mínimo:} queixa desproporcional ao
  achado objetivo.
\end{itemize}
\end{quote}

\begin{quote}
\begin{itemize}
\tightlist
\item[$\square$]
  \textbf{Hostilidade / desrespeito à equipe:} padrão de relação
  difícil.
\end{itemize}
\end{quote}

\begin{quote}
\begin{itemize}
\tightlist
\item[$\square$]
  \textbf{Pedido de ``garantia'':} exige promessa de resultado
  específico (``igual a 20 anos atrás'').
\end{itemize}
\end{quote}

\begin{quote}
\begin{itemize}
\tightlist
\item[$\square$]
  Rigidez cognitiva: não tolera limitações explicadas.
\end{itemize}
\end{quote}

\begin{quote}
\begin{itemize}
\tightlist
\item[$\square$]
  Indecisão crônica: não consegue definir a queixa; muda alvos a cada
  frase.
\end{itemize}
\end{quote}

\begin{quote}
\begin{itemize}
\tightlist
\item[$\square$]
  Histórico de litígio: menção a processos contra outros profissionais.
\end{itemize}
\end{quote}

\section{Checklist pré-op (o ``contrato''
verbal)}\label{checklist-pruxe9-op-o-contrato-verbal}

\begin{itemize}
\item[$\square$]
  \textbf{Teste do espelho:} paciente aponta com um dedo o que incomoda
  (o que ele ``vê'').
\item[$\square$]
  \textbf{Assimetria demonstrada:} sobrancelhas, fenda, sulcos e canto
  lateral mostrados com espelho + fotos.
\item[$\square$]
  \textbf{Limite de correção explicado:} rugas dinâmicas (pés de
  galinha) tendem a persistir sem tratamento específico.
\item[$\square$]
  Cicatrizes e maturação: localização, extensão e tempo (sem prometer
  invisibilidade).
\item[$\square$]
  Recuperação: edema social (dias a poucas semanas) e maturação final
  (meses) alinhadas.
\item[$\square$]
  Riscos específicos: olho seco transitório, equimose, quemoses, retoque
  seletivo quando indicado.
\item[$\square$]
  Motivação confirmada: pessoal vs externa; estabilidade emocional
  mínima.
\end{itemize}

\section{Anatomia aplicada (tradução da
queixa)}\label{anatomia-aplicada-traduuxe7uxe3o-da-queixa}

\begin{itemize}
\item
  \textbf{Complexo frontal--palpebral:} ``pele demais'' pode ser
  compensação do frontal. Ao relaxar o frontal no pós-op, a sobrancelha
  pode parecer cair; isso deve ser antecipado e discutido.
\item
  \textbf{Transição pálpebra--malar (tear trough):} ``olheira'' pode ser
  pigmento, vascular ou sombra. Cirurgia trata relevo/sombra; pigmento
  exige estratégia dermatológica/qualidade de pele.
\item
  \textbf{Globo ocular:} proeminente vs enoftálmico. Explicar que a
  cirurgia não muda a posição do globo; molda tecidos ao redor e depende
  de suporte.
\end{itemize}

\begin{quote}
\textbf{📎 FIGURA NECESSÁRIA (Cap. 09):}
\end{quote}

\begin{quote}
Fluxograma: Queixa do paciente → Hipótese anatômica → Plano cirúrgico
\end{quote}

\begin{quote}
\emph{Estilo: Diagrama técnico-didático, cores neutras, legendas claras}
\end{quote}

\section{Método de consulta (raciocínio
clínico)}\label{muxe9todo-de-consulta-raciocuxednio-cluxednico}

\subsection{Visão geral}\label{visuxe3o-geral-4}

\begin{enumerate}
\def\labelenumi{\arabic{enumi}.}
\item
  \textbf{Escuta ativa:} ``O que te trouxe aqui hoje?'' (permitir
  narrativa curta sem interrupção).
\item
  \textbf{Teste do espelho:} pedir para demonstrar; observar gesto (puxa
  demais? foca em microdetalhe? muda de alvo?).
\item
  \textbf{Diagnóstico em voz alta (educação):} traduzir queixa em
  anatomia (``isso aqui é bolsa/sulco/supercílio'').
\item
  Alinhamento de expectativa: ``Posso melhorar X; Y tende a persistir
  sem tratamento específico''.
\item
  Plano compartilhado: extensão (superior, inferior, ambos) e
  procedimentos associados (suporte lateral, volume, qualidade de pele).
\end{enumerate}

\subsection{Variações e
indicações}\label{variauxe7uxf5es-e-indicauxe7uxf5es-4}

\begin{itemize}
\item
  Paciente ``engenheiro'': quer detalhes técnicos; usar desenho,
  checklist e consentimento robusto.
\item
  Paciente ``vago'': ``quero um refresh''; estabelecer limites claros e
  metas objetivas para evitar frustração.
\end{itemize}

\begin{quote}
\textbf{PÉROLA CLÍNICA}
\end{quote}

\begin{quote}
\mbox{}%
\subsection{Heurística prática: a ``regra do
puxão''}\label{heuruxedstica-pruxe1tica-a-regra-do-puxuxe3o}
\end{quote}

\begin{quote}
Se o paciente precisa puxar excessivamente a pele/face para ``simular''
o resultado desejado, a necessidade pode ser de suporte/moldura
(temporal/brow/face), não apenas blefaroplastia.
\end{quote}

\begin{quote}
Use isso como sinal para explicar limites e evitar promessas baseadas em
um movimento que a cirurgia palpebral não reproduz.
\end{quote}

\section{Erros comuns (e como
resgatar)}\label{erros-comuns-e-como-resgatar-4}

\begin{itemize}
\item
  \textbf{Erro: operar paciente com forte suspeita de TDC / percepção
  distorcida}
\item
  \textbf{Consequência:} insatisfação persistente, múltiplas demandas e
  risco de conflito.
\item
  \textbf{Prevenção:} triagem rigorosa; se a consulta gera desconforto
  clínico, adiar/recusar e encaminhar.
\item
  Resgate: evitar reoperações impulsivas; documentar, apoiar e orientar
  encaminhamento adequado.
\item
  \textbf{Erro: prometer simetria}
\item
  \textbf{Consequência:} paciente passa a medir milimetricamente no
  pós-op.
\item
  \textbf{Prevenção:} explicar variação natural (``irmãs, não gêmeas''),
  documentar assimetrias prévias e repetir limites.
\item
  Resgate: revisitar documentação pré-op e reancorar a conversa em metas
  realistas; considerar retoque apenas quando há indicação objetiva e
  tecido estabilizado.
\end{itemize}

\section{Notas de ``arte'' (gestão da
percepção)}\label{notas-de-arte-gestuxe3o-da-percepuxe7uxe3o}

A arte da consulta é vender \textbf{naturalidade} e reduzir medo do
``olhar operado''. Explicar que a estratégia conservadora (preservar
volume e função do orbicular quando indicado) visa manter identidade
facial e apenas ``descansar'' a expressão --- sem estigmas.

\section{Referências}\label{referuxeancias-3}

\begin{itemize}
\item
  {[}{[}REF:SARWER-2006{]}{]} Sarwer --- triagem/psicologia e risco de
  insatisfação em cirurgia estética
\item
  {[}{[}REF:FAGIEN-1999{]}{]} Fagien --- satisfação/expectativa em
  blefaroplastia e abordagem conservadora
\item
  {[}{[}REF:GUNTER-2007{]}{]} Gunter --- fotografia e documentação de
  assimetrias pré-operatórias
\end{itemize}

\section{Próximo capítulo
(sugestão)}\label{pruxf3ximo-capuxedtulo-sugestuxe3o-2}

Iniciar técnica cirúrgica:

\begin{itemize}
\tightlist
\item
  Cap. 10 --- Blefaroplastia Superior: marcação, pele-only vs
  pele-músculo, preservação de volume, armadilhas e resgates.
\end{itemize}

\begin{center}\rule{0.5\linewidth}{0.5pt}\end{center}

\chapter{Capítulo 10 --- Algoritmos por fenótipo: superior, inferior,
terço médio e casos
mistos}\label{capuxedtulo-10-algoritmos-por-fenuxf3tipo-superior-inferior-teruxe7o-muxe9dio-e-casos-mistos}

\begin{figure}
\centering
\pandocbounded{\includegraphics[keepaspectratio,alt={Figura 10.1 --- Ilustração principal do capítulo}]{/Users/humbertolopes/Dev/work/marcelo-cury/the_art_of_eyelid_surgery_scaffold/projects/eyelid-surgery/assets/figures/FIG-10-01_algoritmo-fenotipos.png}}
\caption{Figura 10.1 --- Ilustração principal do capítulo}
\end{figure}

\textbf{Parte:} Parte II --- Planejamento

\section{Objetivo do capítulo}\label{objetivo-do-capuxedtulo-8}

Ao final, o leitor será capaz de classificar o paciente em fenótipos
clínico-anatômicos e aplicar um algoritmo lógico para selecionar a
combinação de manobras (pele, músculo, gordura, canto lateral,
supercílio e terço médio), evitando a blefaroplastia ``tamanho único''.

\section{O que muda na decisão (o
``porquê'')}\label{o-que-muda-na-decisuxe3o-o-porquuxea-9}

\begin{itemize}
\item
  \textbf{Personalização baseada no defeito dominante:} blefaroplastia é
  um conjunto de manobras. O erro recorrente é usar protocolo subtrativo
  (pele/gordura) em fenótipos de deflação, vetor negativo ou frouxidão,
  produzindo estigma e complicações.
\item
  \textbf{Hierarquia da correção:} problemas estruturais e funcionais
  (ptose, tônus cantal, vetor, olho seco) têm precedência sobre detalhes
  estéticos (pele, bolsas). Ignorar essa ordem é uma fonte clássica de
  falhas.
\item
  \textbf{Terço médio como fundação em casos selecionados:} em
  hipoplasia malar, vetor negativo e sulco pálpebra--malar marcado,
  tratar apenas a pálpebra inferior pode aumentar risco de retração. O
  algoritmo deve incluir suporte (volume/suspensão) quando indicado.
\end{itemize}

\section{Indicações e
contra-indicações}\label{indicauxe7uxf5es-e-contra-indicauxe7uxf5es-3}

\textbf{Indicar quando:} - planejamento de blefaroplastia primária ou
secundária;

\begin{itemize}
\item
  casos com ptose associada a dermatochalase;
\item
  pacientes com edema malar/festoons (diferenciar de bolsa verdadeira).
\end{itemize}

\textbf{Evitar / adiar quando:} - queixa é predominantemente
dermatológica (discromia/elastose) sem alteração estrutural relevante
--- considerar \emph{resurfacing} antes ou como principal abordagem;

\begin{itemize}
\tightlist
\item
  sinais de percepção distorcida/expectativa rígida sem alvo anatômico
  corrigível.
\end{itemize}

\begin{quote}
\textbf{PÉROLA CLÍNICA}
\end{quote}

\begin{quote}
\mbox{}%
\subsection{Checklist de classificação
fenotípica}\label{checklist-de-classificauxe7uxe3o-fenotuxedpica}
\end{quote}

\begin{quote}
\begin{itemize}
\tightlist
\item[$\square$]
  \textbf{Superior:} pele apenas / pele + gordura (medial) / pele +
  ptose (aponeurose) / pele + queda de supercílio
\end{itemize}
\end{quote}

\begin{quote}
\begin{itemize}
\tightlist
\item[$\square$]
  Inferior (pele): excesso real (pinch test) vs ``falsa sobra'' por
  deflação/frouxidão
\end{itemize}
\end{quote}

\begin{quote}
\begin{itemize}
\tightlist
\item[$\square$]
  Inferior (bolsa): protrusão verdadeira (septal) vs pseudobolsa
  (edema/festoon)
\end{itemize}
\end{quote}

\begin{quote}
\begin{itemize}
\tightlist
\item[$\square$]
  \textbf{Tônus:} pexia (snap-back lento) vs plastia (distraction
  aumentado) conforme gravidade
\end{itemize}
\end{quote}

\begin{quote}
\begin{itemize}
\tightlist
\item[$\square$]
  \textbf{Transição:} sulco nasojugal profundo / lid-cheek junction
  marcada → precisa de reposição/redistribuição de volume?
\end{itemize}
\end{quote}

\begin{quote}
\begin{itemize}
\tightlist
\item[$\square$]
  Vetor: positivo/neutro/negativo (implica suporte e conservadorismo
  subtrativo)
\end{itemize}
\end{quote}

\section{Anatomia aplicada (zonas de
decisão)}\label{anatomia-aplicada-zonas-de-decisuxe3o}

\begin{itemize}
\item
  \textbf{Junção pálpebra--malar (lid--cheek junction):} objetivo
  estético central é reduzir a ``quebra'' (sombra/linha) e recuperar
  continuidade. Se a transição é longa e baixa (envelhecida), o
  algoritmo tende a demandar volume e/ou suporte.
\item
  \textbf{Ligamento retentor do orbicular (ORL):} ajuda a definir tear
  trough e ``degrau'' pálpebra--malar. Em estratégias de
  reposicionamento/redistribuição de gordura, sua liberação pode ser
  necessária, conforme técnica e indicação.
\item
  \textbf{Zona de risco (vetor negativo):} fenótipo em que o globo
  projeta anterior ao suporte malar. Subtração (pele/gordura) tende a
  ser de maior risco sem suporte lateral/estratégia de volume.
\end{itemize}

\begin{quote}
\textbf{📎 FIGURA NECESSÁRIA (Cap. 10):}
\end{quote}

\begin{quote}
Fluxograma de decisão: Superior / Inferior / Terço médio / Casos mistos
\end{quote}

\begin{quote}
\emph{Estilo: Diagrama técnico-didático, cores neutras, legendas claras}
\end{quote}

\section{Técnica (método de
raciocínio)}\label{tuxe9cnica-muxe9todo-de-raciocuxednio}

\subsection{Algoritmo --- pálpebra
superior}\label{algoritmo-puxe1lpebra-superior}

\begin{enumerate}
\def\labelenumi{\arabic{enumi}.}
\item
  \textbf{É apenas pele?} → considerar \emph{skin-only} em casos
  selecionados, com ressecção conservadora e avaliação do supercílio.
\item
  \textbf{Pele + componente músculo/orbicular?} → considerar abordagem
  com manejo do orbicular quando houver indicação
  (peso/hipertrofia/roll), preservando função.
\item
  \textbf{Ptose associada?} → adicionar correção de ptose
  (aponeurose/Müller conforme diagnóstico e testes).
\item
  Queda de cauda de supercílio? → adicionar browpexy/lifting temporal
  conforme caso. Evitar ``compensar'' retirando pele em excesso.
\end{enumerate}

\subsection{Algoritmo --- pálpebra
inferior}\label{algoritmo-puxe1lpebra-inferior}

\begin{enumerate}
\def\labelenumi{\arabic{enumi}.}
\item
  \textbf{Bolsa sem excesso de pele (mais jovem):} → transconjuntival
  (preservação da lamela anterior).
\item
  \textbf{Bolsa + pele leve:} → transconjuntival + \emph{pinch}
  conservador e/ou \emph{resurfacing} (selecionar conforme pele).
\item
  \textbf{Bolsa + sulco marcado (tear trough/lid-cheek):} →
  transconjuntival com estratégia de volume (redistribuição/transposição
  de gordura ou lipoenxertia, conforme fenótipo).
\item
  Frouxidão horizontal / vetor negativo: → suporte lateral
  (cantopexia/cantoplastia conforme gravidade) associado à abordagem
  escolhida; subtração deve ser mínima e guiada por testes.
\end{enumerate}

\subsection{Casos mistos (terço
médio)}\label{casos-mistos-teruxe7o-muxe9dio}

\begin{itemize}
\tightlist
\item
  Se houver ptose de SOOF e aprofundamento do ``V'' pálpebra--malar, o
  algoritmo pode exigir estratégia de terço médio (suspensão e/ou
  volume) para reconstituir a transição, em vez de tratar apenas a
  pálpebra.
\end{itemize}

\begin{quote}
\textbf{PÉROLA CLÍNICA}
\end{quote}

\begin{quote}
\mbox{}%
\subsection{Regra prática: teste da ``pele sobrante'' (pálpebra
inferior)}\label{regra-pruxe1tica-teste-da-pele-sobrante-puxe1lpebra-inferior}
\end{quote}

\begin{quote}
Peça ao paciente para olhar para cima e abrir a boca.
\end{quote}

\begin{quote}
Se a pele inferior fica tensa, \textbf{o excesso real é pequeno} --- a
queixa tende a ser volume/suporte, não pele. Subtração aqui aumenta
risco de ectrópio.
\end{quote}

\section{Erros comuns (e como
resgatar)}\label{erros-comuns-e-como-resgatar-5}

\begin{itemize}
\item
  Erro: tratar ptose de supercílio como excesso de pele
\item
  \textbf{Consequência:} cicatriz alta, sobrancelha relativamente mais
  baixa, olhar pesado e resultado ``operado''.
\item
  \textbf{Prevenção:} estabilizar testa/supercílio no exame; se ``some''
  o excesso de pele, a causa é moldura frontal.
\item
  \textbf{Resgate:} brow/temporal lift (conforme indicação e timing).
\item
  Erro: esvaziar o ``olho fundo'' (fenótipo senil/deflação)
\item
  \textbf{Consequência:} órbita esqueletizada, aumento de sombras e
  estigma.
\item
  \textbf{Prevenção:} algoritmo de preservação/redistribuição; remover
  apenas o que ultrapassa claramente o rebordo.
\item
  \textbf{Resgate:} reposição volumétrica (microfat/nanofat conforme
  objetivo: volume vs qualidade de pele).
\end{itemize}

\section{Notas de ``arte'' (harmonia)}\label{notas-de-arte-harmonia}

O algoritmo não olha a pálpebra isolada: ele avalia continuidade com
testa e bochecha. Em casos mistos, subcorrigir pálpebra e corrigir a
unidade vizinha (sobrancelha/malar) costuma ser mais natural do que
supercorrigir pálpebra e criar uma ``ilha'' esticada numa face
envelhecida.

\section{Referências}\label{referuxeancias-4}

\begin{itemize}
\item
  {[}{[}REF:ROHRICH-2008{]}{]} Rohrich --- abordagem sistemática para
  blefaroplastia inferior (passos/segurança)
\item
  {[}{[}REF:JELKS-1993{]}{]} Jelks --- classificação de vetores e
  implicações
\item
  {[}{[}REF:HAMRA-1995{]}{]} Hamra / Goldberg ---
  preservação/redistribuição de gordura e rejuvenescimento periorbital
\end{itemize}

\section{Próximo capítulo}\label{pruxf3ximo-capuxedtulo}

Cap. 11 --- Blefaroplastia Superior: marcação e execução (skin-only vs
skin-muscle), preservação de volume e armadilhas.

\begin{center}\rule{0.5\linewidth}{0.5pt}\end{center}

\chapter{Capítulo 11 --- Marcação e medidas: superior e inferior
(conservadorismo e
simetria)}\label{capuxedtulo-11-marcauxe7uxe3o-e-medidas-superior-e-inferior-conservadorismo-e-simetria}

\begin{figure}
\centering
\pandocbounded{\includegraphics[keepaspectratio,alt={Figura 11.1 --- Ilustração principal do capítulo}]{/Users/humbertolopes/Dev/work/marcelo-cury/the_art_of_eyelid_surgery_scaffold/projects/eyelid-surgery/assets/figures/FIG-11-01_marcacao-medidas-simetria.png}}
\caption{Figura 11.1 --- Ilustração principal do capítulo}
\end{figure}

\textbf{Parte:} Parte II --- Planejamento

\section{Objetivo do capítulo}\label{objetivo-do-capuxedtulo-9}

Ao final, o leitor dominará a ``arte da preservação'' na marcação
cirúrgica, definindo limites de ressecção que mantêm fechamento
palpebral completo, sulco natural e simetria funcional, reduzindo risco
de lagoftalmo iatrogênico.

\section{O que muda na decisão (o
``porquê'')}\label{o-que-muda-na-decisuxe3o-o-porquuxea-10}

\begin{itemize}
\item
  \textbf{A gravidade é o juiz:} marcar deitado aumenta risco de
  super-ressecção. A marcação deve ser feita sentado (≈90°), com
  expressão e posição real do supercílio/pele.
\item
  \textbf{Segurança antes de estética:} a ressecção é guiada por
  \emph{teste de fechamento} e pelo \emph{pinch test}, não por ``quanto
  sobra na foto''. Uma marcação conservadora é mais fácil de ajustar do
  que um lagoftalmo para ``desfazer''.
\item
  \textbf{Extensão lateral planejada:} insatisfação lateral costuma vir
  de marcações curtas. A incisão deve acompanhar linhas naturais (pé de
  galinha/RSTL), com discreta ascensão lateral para tratar
  \emph{hooding}.
\end{itemize}

\section{Indicações e
contra-indicações}\label{indicauxe7uxf5es-e-contra-indicauxe7uxf5es-4}

\textbf{Indicar quando:} - etapa pré-anestesia em praticamente todos os
casos (antes de infiltração volumosa), para evitar distorção de
referências e permitir avaliação dinâmica.

\textbf{Evitar / adiar quando:} - edema palpebral agudo/alergia no dia
(distorce o ``excesso'');

\begin{itemize}
\tightlist
\item
  sedação prévia que impeça abertura/fechamento confiáveis durante o
  \emph{pinch test}.
\end{itemize}

\begin{quote}
\textbf{PÉROLA CLÍNICA}
\end{quote}

\begin{quote}
\mbox{}%
\subsection{Checklist de marcação
(pré-anestesia)}\label{checklist-de-marcauxe7uxe3o-pruxe9-anestesia}
\end{quote}

\begin{quote}
\begin{itemize}
\tightlist
\item[$\square$]
  \textbf{Posição:} sentado, coluna reta, olhar no horizonte
\end{itemize}
\end{quote}

\begin{quote}
\begin{itemize}
\tightlist
\item[$\square$]
  \textbf{Caneta:} dermatográfica fina (evitar ponta grossa)
\end{itemize}
\end{quote}

\begin{quote}
\begin{itemize}
\tightlist
\item[$\square$]
  \textbf{Pele:} desengordurar para fixar a tinta
\end{itemize}
\end{quote}

\begin{quote}
\begin{itemize}
\tightlist
\item[$\square$]
  Sulco (linha inferior --- superior): homens 7--8 mm; mulheres 8--10 mm
  (a partir da margem ciliar, ajustando ao biotipo)
\end{itemize}
\end{quote}

\begin{quote}
\begin{itemize}
\tightlist
\item[$\square$]
  Linha superior: definida pelo \emph{pinch test} (não por ``medida
  fixa'')
\end{itemize}
\end{quote}

\begin{quote}
\begin{itemize}
\tightlist
\item[$\square$]
  Teste de fechamento: após desenhar, pedir fechamento suave (sem
  esforço) e confirmar que não há tensão excessiva
\end{itemize}
\end{quote}

\section{Anatomia aplicada (referências
cutâneas)}\label{anatomia-aplicada-referuxeancias-cutuxe2neas}

\begin{itemize}
\item
  \textbf{Tarso e sulco:} a linha inferior deve respeitar a anatomia do
  tarso e o padrão do sulco do paciente. Marcações muito altas tendem a
  criar sulco artificial/estigmatizado.
\item
  \textbf{Canto medial:} evitar ultrapassar referência do ponto lacrimal
  em direção nasal para reduzir risco de \emph{webbing}.
\item
  \textbf{Pele fina vs espessa:} a cicatriz deve permanecer na pele
  palpebral fina (onde ela ``some''); subir demais para pele espessa
  aumenta visibilidade.
\end{itemize}

\begin{quote}
\textbf{📎 FIGURA NECESSÁRIA (Cap. 11):}
\end{quote}

\begin{quote}
Diagrama de marcação superior: Linha do sulco + fuso, extensão lateral
em RSTL
\end{quote}

\begin{quote}
\emph{Estilo: Diagrama técnico-didático, cores neutras, legendas claras}
\end{quote}

\section{Técnica (o pinch test)}\label{tuxe9cnica-o-pinch-test}

\subsection{Blefaroplastia superior --- visão
geral}\label{blefaroplastia-superior-visuxe3o-geral}

\begin{enumerate}
\def\labelenumi{\arabic{enumi}.}
\item
  \textbf{Definir sulco (linha inferior):} marcar seguindo o padrão
  anatômico do paciente (≈8--10 mm no centro, ajustando por
  sexo/biotipo).
\item
  \textbf{Pinch test:} pinçar pele acima da linha inferior. A quantidade
  adequada \textbf{suaviza} sem eversão de cílios e sem dificultar
  fechamento suave.
\item
  Desenhar o fuso: marcar pontos do pinçamento e conectar. Em geral,
  maior largura no terço lateral.
\item
  Extensão lateral: ``morrer'' dentro de ruga natural (pé de galinha),
  com discreta ascensão (evitar cauda descendente).
\end{enumerate}

\subsection{\texorpdfstring{Pálpebra inferior --- visão geral
(\emph{skin pinch} quando
indicado)}{Pálpebra inferior --- visão geral (skin pinch quando indicado)}}\label{puxe1lpebra-inferior-visuxe3o-geral-skin-pinch-quando-indicado}

\begin{itemize}
\item
  Marcar subciliar (≈2 mm abaixo dos cílios) \textbf{somente} se houver
  excesso real.
\item
  Fazer \emph{pinch} com o paciente olhando para cima e \textbf{boca
  aberta} (tensão máxima). Se ainda houver sobra, a ressecção tende a
  ser pequena (≈1--2 mm, conservadora).
\end{itemize}

\subsection{Variações e
indicações}\label{variauxe7uxf5es-e-indicauxe7uxf5es-5}

\begin{itemize}
\item
  \textbf{Homens:} sulco mais baixo (≈7--8 mm) e desenho mais
  conservador para evitar feminização.
\item
  \textbf{Asiáticos:} decidir previamente objetivo (criar sulco vs
  preservar anatomia). Marcação e fixação seguem lógica específica e
  mais baixa.
\end{itemize}

\begin{quote}
\textbf{PÉROLA CLÍNICA}
\end{quote}

\begin{quote}
\mbox{}%
\subsection{Zona de risco: ``dog ear''/webbing
medial}\label{zona-de-risco-dog-earwebbing-medial}
\end{quote}

\begin{quote}
Evite ressecção medial agressiva.
\end{quote}

\begin{quote}
\textbf{Regra prática:} máxima ressecção no terço lateral, mínima/nula
no terço medial, conforme \emph{pinch test} e anatomia.
\end{quote}

\section{Erros comuns (e como
resgatar)}\label{erros-comuns-e-como-resgatar-6}

\begin{itemize}
\item
  \textbf{Erro:} marcar ``simétrico'' em sobrancelhas assimétricas
  \textbf{Consequência:} sulcos em alturas diferentes (ou percepção de
  assimetria maior no pós-op) \textbf{Prevenção:} ajustar marcação pela
  meta estética (simetria do sulco), não pela simetria do desenho
  cutâneo; considerar browpexy/lifting quando indicado Resgate: revisão
  seletiva (excisão adicional conservadora no lado necessário) e/ou
  correção de supercílio, respeitando timing de cicatrização
\item
  Erro: ressecção excessiva de pele (lagoftalmo) \textbf{Consequência:}
  incapacidade de fechar completamente; risco de exposição corneana
  \textbf{Prevenção:} \emph{pinch test} + teste de fechamento suave;
  preservar pele suficiente para fechamento passivo (medidas são guias;
  a função é o critério) \textbf{Resgate:} lubrificação intensa e
  proteção; considerar enxerto de pele total em casos persistentes (ex.:
  retroauricular), após estabilização
\end{itemize}

\section{Notas de ``arte''
(continuidade)}\label{notas-de-arte-continuidade}

A cicatriz lateral deve se camuflar em transições naturais e RSTL. A
marcação ``boa'' não termina abruptamente: ela se dissolve em ruga
existente, preservando a continuidade do olhar e evitando o traço
clássico do ``operado''.

\section{Referências}\label{referuxeancias-5}

\begin{itemize}
\item
  {[}{[}REF:FAGIEN-1999{]}{]} Fagien --- \emph{pinch} cutâneo e
  conservadorismo
\item
  {[}{[}REF:FLOWERS-1993{]}{]} Flowers --- princípios de marcação e
  preservação
\item
  {[}{[}REF:REES-1984{]}{]} Rees --- segurança em excisão de pele
  palpebral
\end{itemize}

\begin{center}\rule{0.5\linewidth}{0.5pt}\end{center}

\chapter{Capítulo 12 --- Anestesia, infiltração, hemostasia e pós
imediato
(segurança)}\label{capuxedtulo-12-anestesia-infiltrauxe7uxe3o-hemostasia-e-puxf3s-imediato-seguranuxe7a}

\begin{figure}
\centering
\pandocbounded{\includegraphics[keepaspectratio,alt={Figura 12.1 --- Ilustração principal do capítulo}]{/Users/humbertolopes/Dev/work/marcelo-cury/the_art_of_eyelid_surgery_scaffold/projects/eyelid-surgery/assets/figures/FIG-12-01_anestesia-infiltracao-hemostasia-pos-imediato.png}}
\caption{Figura 12.1 --- Ilustração principal do capítulo}
\end{figure}

\textbf{Parte:} Parte II --- Planejamento (Transição para Execução)

\section{Objetivo do capítulo}\label{objetivo-do-capuxedtulo-10}

Ao final, o leitor saberá instituir um protocolo de anestesia,
infiltração e hemostasia que favoreça campo cirúrgico exangue, melhore
visualização anatômica e reduza o risco da complicação mais temida:
hematoma retrobulbar.

\section{O que muda na decisão (o
``porquê'')}\label{o-que-muda-na-decisuxe3o-o-porquuxea-11}

\begin{itemize}
\item
  \textbf{O tempo é um agente farmacológico:} incisar imediatamente após
  infiltrar costuma ``roubar'' o pico vasoconstrictor da epinefrina. Em
  geral, aguardar \textasciitilde10--15 minutos muda a qualidade do
  campo operatório.
\item
  \textbf{Hidrodissecção como ferramenta:} infiltração não é só
  analgesia; é criação de plano. Se a agulha não progride com baixa
  resistência, o plano provavelmente está errado.
\item
  \textbf{Sedação consciente vs.~profunda:} em cirurgia palpebral,
  cooperação e tônus basal importam. Sedação profunda pode falsear
  avaliação de pele/posição e impedir checagens dinâmicas, aumentando
  risco de erro de planejamento.
\end{itemize}

\section{Indicações e
contra-indicações}\label{indicauxe7uxf5es-e-contra-indicauxe7uxf5es-5}

\textbf{Indicar quando:} - blefaroplastias estéticas e funcionais em
regime de anestesia local com sedação leve (oral ou venosa), conforme
perfil do paciente e protocolo do serviço;

\begin{itemize}
\tightlist
\item
  pacientes hipertensos controlados, com monitorização hemodinâmica
  adequada.
\end{itemize}

\textbf{Evitar / adiar quando:} - \textbf{hipertensão descompensada no
dia} (ex.: níveis persistentemente elevados apesar de
repouso/medicação), por aumento de risco de sangramento e complicações;

\begin{itemize}
\tightlist
\item
  coagulopatias / uso de anticoagulantes/antiagregantes: conduta deve
  ser definida com cardiologia/anestesia, considerando risco trombótico
  vs.~hemorrágico e protocolo institucional; incluir também suplementos
  com potencial efeito antiagregante.
\end{itemize}

\begin{quote}
\textbf{PÉROLA CLÍNICA}
\end{quote}

\begin{quote}
\mbox{}%
\subsection{Checklist de infiltração e sala
(segurança)}\label{checklist-de-infiltrauxe7uxe3o-e-sala-seguranuxe7a}
\end{quote}

\begin{quote}
\begin{itemize}
\tightlist
\item[$\square$]
  \textbf{Solução (exemplo comum):} lidocaína 1--2\% + epinefrina
  (p.ex., 1:100.000 ou 1:200.000), respeitando dose máxima e
  comorbidades.
\end{itemize}
\end{quote}

\begin{quote}
\begin{itemize}
\tightlist
\item[$\square$]
  \textbf{Tampão (opcional):} bicarbonato 8,4\% (p.ex., 1:10) para
  reduzir ardor à injeção.
\end{itemize}
\end{quote}

\begin{quote}
\begin{itemize}
\tightlist
\item[$\square$]
  \textbf{Proteção ocular:} colírio anestésico + proteção corneana
  quando houver manipulação profunda/laser, conforme preferência do
  serviço.
\end{itemize}
\end{quote}

\begin{quote}
\begin{itemize}
\tightlist
\item[$\square$]
  Latência: cronometrar intervalo entre infiltração e incisão (em geral
  \textasciitilde10--15 min).
\end{itemize}
\end{quote}

\begin{quote}
\begin{itemize}
\tightlist
\item[$\square$]
  Pressão arterial: monitorização seriada (frequentemente a cada 3--5
  min em fase crítica); manter alvo hemodinâmico acordado com anestesia.
\end{itemize}
\end{quote}

\begin{quote}
\begin{itemize}
\tightlist
\item[$\square$]
  Prontidão para emergência: material para descompressão orbitária
  imediata disponível e equipe alinhada com o protocolo.
\end{itemize}
\end{quote}

\section{Anatomia aplicada (nervos e
planos)}\label{anatomia-aplicada-nervos-e-planos}

\begin{itemize}
\item
  \textbf{Arcadas vasculares:} vasos marginais correm próximos ao tarso;
  infiltração profunda e não controlada pode gerar hematoma antes mesmo
  da incisão.
\item
  \textbf{Nervos sensitivos:} supratroclear, supraorbital, infraorbital
  e zigomático-facial orientam bloqueios/analgesia regional.
\item
  \textbf{Plano de segurança:} iniciar superficial (subcutâneo) para
  turgor cutâneo e depois pré-septal; evitar injeções profundas ``às
  cegas'' no pós-septal para reduzir risco de lesão vascular orbitária.
\end{itemize}

\begin{quote}
\textbf{📎 FIGURA NECESSÁRIA (Cap. 12):}
\end{quote}

\begin{quote}
Mapa de pontos de bloqueio anestésico periorbitário
\end{quote}

\begin{quote}
\emph{Estilo: Diagrama técnico-didático, cores neutras, legendas claras}
\end{quote}

\section{Técnica (sequência
lógica)}\label{tuxe9cnica-sequuxeancia-luxf3gica}

\subsection{Visão geral}\label{visuxe3o-geral-5}

\begin{enumerate}
\def\labelenumi{\arabic{enumi}.}
\item
  \textbf{Marcação:} sempre antes da infiltração (Cap. 11).
\item
  \textbf{Sedação:} leve, com paciente responsivo a comando verbal.
\item
  \textbf{Infiltração (lenta e progressiva):} agulha fina; iniciar por
  área menos sensível; avançar e injetar lentamente.
\item
  Dispersão do anestésico: compressão/massagem suave para distribuir e
  reduzir bolhas locais.
\item
  Espera: aguardar o branqueamento (\emph{blanching}) e o tempo de
  latência acordado.
\end{enumerate}

\subsection{Variações e
indicações}\label{variauxe7uxf5es-e-indicauxe7uxf5es-6}

\begin{itemize}
\tightlist
\item
  \textbf{Bloqueio regional:} em procedimentos extensos ou pacientes
  muito sensíveis, bloqueios supraorbital/infraorbital podem reduzir
  desconforto e diminuir necessidade de volume local (com menor
  distorção), conforme rotina do cirurgião.
\end{itemize}

\section{Erros comuns (e como
resgatar)}\label{erros-comuns-e-como-resgatar-7}

\begin{itemize}
\item
  \textbf{Erro:} incisão precoce (antes do pico vasoconstrictor)
  \textbf{Consequência:} campo sangrento, pior visualização, mais
  edema/equimose e maior chance de hemostasia ``reativa''
  \textbf{Prevenção:} usar relógio: infiltrar ambos os lados e aguardar
  o tempo de latência planejado Resgate: pausar, compressão dirigida por
  alguns minutos, reavaliar pontos de sangramento e preferir hemostasia
  precisa (ex.: bipolar)
\item
  Erro: uso indiscriminado de cautério monopolar próximo a estruturas
  nobres \textbf{Consequência:} maior risco de lesão térmica,
  carbonização e fibrose, além de dano em tecidos adjacentes
  \textbf{Prevenção:} preferir cautério bipolar/energia controlada e
  hemostasia por planos, com parcimônia \textbf{Resgate:} suporte local
  conforme lesão; na prática, a ``correção'' real é prevenção (ajuste de
  técnica e energia)
\end{itemize}

\begin{quote}
\textbf{PÉROLA CLÍNICA}
\end{quote}

\begin{quote}
\mbox{}%
\subsection{Zona de risco: hematoma
retrobulbar}\label{zona-de-risco-hematoma-retrobulbar}
\end{quote}

\begin{quote}
Emergência máxima da blefaroplastia: sangramento orbitário com risco
visual.
\end{quote}

\begin{quote}
\begin{itemize}
\tightlist
\item
  \textbf{Sinais:} dor intensa desproporcional, proptose, piora rápida
  de equimose/tensão, alteração visual, limitação de motilidade.
\end{itemize}
\end{quote}

\begin{quote}
\begin{itemize}
\tightlist
\item
  \textbf{Conduta:} priorizar intervenção imediata conforme protocolo do
  serviço; não atrasar conduta por exames quando a clínica sugere risco
  visual iminente. Considerar abertura de suturas, descompressão lateral
  (cantotomia/cantólise) e acionamento do time de retaguarda.
\end{itemize}
\end{quote}

\section{Notas de ``arte''
(pós-imediato)}\label{notas-de-arte-puxf3s-imediato}

A arte aqui é preservação tecidual: menos sangue no leito cirúrgico
tende a significar menos inflamação/fibrose e recuperação social mais
rápida. Medidas simples (compressas frias protegidas, cabeceira elevada)
impactam edema e conforto.

\section{Pós-operatório imediato (check de
saída)}\label{puxf3s-operatuxf3rio-imediato-check-de-sauxedda}

\begin{itemize}
\item
  \textbf{Visão:} checar função visual de forma simples e documentar
  antes da alta.
\item
  \textbf{Dor:} dor intensa/progressiva é sinal de alarme (avaliar
  imediatamente).
\item
  \textbf{Orientações:} gelo com proteção, elevar cabeceira, evitar
  manobras de Valsalva e esforço.
\end{itemize}

\section{Referências}\label{referuxeancias-6}

\begin{itemize}
\item
  {[}{[}REF:ROHRICH-2008{]}{]} Rohrich --- segurança e princípios gerais
  em cirurgia facial sob anestesia local
\item
  {[}{[}REF:HASS-2004{]}{]} Hass / Most --- protocolos e manejo de
  hematoma retrobulbar
\item
  {[}{[}REF:FAGIEN-1999{]}{]} Fagien --- infiltração/hidrodissecção e
  hemostasia em cirurgia periorbital
\end{itemize}

\begin{center}\rule{0.5\linewidth}{0.5pt}\end{center}

\chapter{Capítulo 13 --- Brow management: por que blef isolada falha
(Connell) e
indicações}\label{capuxedtulo-13-brow-management-por-que-blef-isolada-falha-connell-e-indicauxe7uxf5es}

\begin{figure}
\centering
\pandocbounded{\includegraphics[keepaspectratio,alt={Figura 13.1 --- Ilustração principal do capítulo}]{/Users/humbertolopes/Dev/work/marcelo-cury/the_art_of_eyelid_surgery_scaffold/projects/eyelid-surgery/assets/figures/FIG-13-01_brow-management-comparativo.png}}
\caption{Figura 13.1 --- Ilustração principal do capítulo}
\end{figure}

\begin{quote}
\textbf{Leitura guiada:} este capítulo aborda o leitor aprenderá a
indicar estabilização (pexy) ou elevação (lift) antes de definir
ressecções palpebrais.
\end{quote}

\textbf{Parte:} Parte III --- Terço Superior

\section{Objetivo do capítulo}\label{objetivo-do-capuxedtulo-11}

Ao final, o leitor saberá diagnosticar a ``falsa'' dermatochalase
causada por ptose de supercílio e entender por que remover pele
palpebral em paciente com compensação frontal (Sinal de Connell) aumenta
risco de resultado estigmatizado. O leitor aprenderá a indicar
estabilização (pexy) ou elevação (lift) antes de definir ressecções
palpebrais.

\section{O que muda na decisão (o
``porquê'')}\label{o-que-muda-na-decisuxe3o-o-porquuxea-12}

\begin{itemize}
\item
  \textbf{Princípio da cortina e do varão:} a pálpebra é a cortina; o
  supercílio é o varão. Se o ``capuz'' aumenta porque o varão caiu,
  cortar a cortina sem reposicionar o varão encurta o terço superior e
  desequilibra as proporções.
\item
  \textbf{Sinal de Connell (a armadilha):} pacientes com ptose de
  supercílio usam o frontal cronicamente para ``abrir'' os olhos. Com
  relaxamento (sedação, fadiga, supino), o supercílio desce e a pálpebra
  parece ter ``pele a mais''. Marcação/ressecção sem compensar esse
  efeito favorece super-ressecção e piora estética/funcional.
\item
  \textbf{Textura e identidade:} pele do supercílio é mais espessa e a
  palpebral é mais fina. Blefaroplastia isolada em ptose de supercílio
  aproxima pele grossa dos cílios, apaga a transição e tende a produzir
  olhar pesado e ``operado''.
\end{itemize}

\section{Indicações e
contra-indicações}\label{indicauxe7uxf5es-e-contra-indicauxe7uxf5es-6}

\textbf{Indicar manejo do supercílio quando:} - houver ptose visível do
supercílio (com queda global ou predominantemente lateral);

\begin{itemize}
\item
  houver \emph{hooding} lateral que ultrapassa o rebordo orbitário
  lateral;
\item
  houver rugas frontais profundas (sugestivas de compensação ativa);
\item
  \textbf{teste de relaxamento manual:} ao bloquear o frontal, a
  ``dermatochalase'' palpebral aumenta de forma relevante.
\end{itemize}

\textbf{Evitar / ter cautela quando:} - \textbf{homens:} sobrancelha
masculina é mais baixa e retificada; elevações excessivas podem
feminizar. Em muitos casos, estabilização (browpexy) é preferível a
elevação ampla.

\begin{itemize}
\item
  \textbf{olhos proeminentes / tendência a lagoftalmo:} elevar demais o
  supercílio pode aumentar exposição e sensação de olho seco.
\item
  calvície/linha capilar alta (para acessos coronais/pré-capilares):
  priorizar acessos temporais limitados ou técnicas diretas bem
  indicadas.
\end{itemize}

\begin{quote}
\textbf{PÉROLA CLÍNICA}
\end{quote}

\begin{quote}
\mbox{}%
\subsection{Checklist pré-op: Teste de Connell (compensação
frontal)}\label{checklist-pruxe9-op-teste-de-connell-compensauxe7uxe3o-frontal}
\end{quote}

\begin{quote}
\begin{enumerate}
\def\labelenumi{\arabic{enumi}.}
\tightlist
\item
  Peça ao paciente para olhar para frente em repouso.
\end{enumerate}
\end{quote}

\begin{quote}
\begin{enumerate}
\def\labelenumi{\arabic{enumi}.}
\setcounter{enumi}{1}
\tightlist
\item
  Bloqueie o músculo frontal com pressão suave na testa (sem empurrar o
  supercílio para baixo artificialmente).
\end{enumerate}
\end{quote}

\begin{quote}
\begin{enumerate}
\def\labelenumi{\arabic{enumi}.}
\setcounter{enumi}{2}
\tightlist
\item
  Peça para fechar os olhos e abrir mantendo o bloqueio.
\end{enumerate}
\end{quote}

\begin{quote}
\begin{enumerate}
\def\labelenumi{\arabic{enumi}.}
\setcounter{enumi}{3}
\tightlist
\item
  \textbf{Interpretação:} se a sobrancelha cai e a ``sobra'' palpebral
  aumenta de forma significativa, há ptose latente/compensação.
  Blefaroplastia isolada deve ser conservadora e, frequentemente,
  associada a estabilização/elevação do supercílio.
\end{enumerate}
\end{quote}

\section{Anatomia aplicada (o que sustenta o
supercílio)}\label{anatomia-aplicada-o-que-sustenta-o-supercuxedlio}

\begin{itemize}
\item
  \textbf{ROOF (gordura retro-orbicular):} contribui para plenitude e
  ``peso'' do supercílio. Com o envelhecimento, pode deslizar
  inferiormente e simular dermatochalase. Em muitos casos,
  reposicionamento/redistribuição é mais coerente do que ressecção
  agressiva.
\item
  \textbf{Ancoragens periorbitárias superiores:} estruturas retentoras e
  transições de fáscia/ periósteo definem quanto o supercílio ``cede'' e
  como ele responde à pexia/lift.
\item
  \textbf{Nervo facial (ramo temporal):} risco em dissecções temporais
  superficiais. A segurança é respeitar planos fasciais (fáscia
  temporoparietal vs.~fáscia temporal profunda) e manter técnica de
  baixo trauma.
\end{itemize}

\begin{quote}
\textbf{📎 FIGURA NECESSÁRIA (Cap. 13):}
\end{quote}

\begin{quote}
Teste de Connell ilustrado: Bloqueio do frontal e observação da ptose
\end{quote}

\begin{quote}
\emph{Estilo: Diagrama técnico-didático, cores neutras, legendas claras}
\end{quote}

\section{Técnica (estratégia de
decisão)}\label{tuxe9cnica-estratuxe9gia-de-decisuxe3o}

Não existe técnica única: há um gradiente de invasividade e indicação.

\subsection{Visão geral (menu de
opções)}\label{visuxe3o-geral-menu-de-opuxe7uxf5es}

\begin{enumerate}
\def\labelenumi{\arabic{enumi}.}
\item
  \textbf{Estabilização (internal browpexy):} via incisão da
  blefaroplastia. Objetivo principal é evitar piora e controlar queda
  lateral discreta.
\item
  \textbf{Elevação lateral (temporal lift):} acesso temporal para
  corrigir \emph{hooding} lateral e reposicionar cauda do supercílio.
\item
  \textbf{Elevação direta:} excisão acima do supercílio (boa em idosos
  com rugas profundas ou paralisia facial), com custo de cicatriz mais
  visível.
\item
  Lifting endoscópico/coronal: reposicionamento global de fronte/glabela
  (ver capítulo correspondente). \#\#\# Onde a técnica falha (ponto
  crítico)
\end{enumerate}

Se o diagnóstico for ``excesso de pele'' quando o problema predominante
é ``posição/volume do supercílio (ROOF + ptose)'', a blefaroplastia
isolada tende a criar pálpebra vazia e manter a sobrancelha baixa. O
plano deve distinguir: \textbf{Volume vs.~Posição vs.~Pele.}

\section{Erros comuns (e como
resgatar)}\label{erros-comuns-e-como-resgatar-8}

\begin{itemize}
\item
  \textbf{Erro:} ignorar ptose/compensação frontal e operar apenas a
  pálpebra \textbf{Consequência:} encurtamento da distância
  cílio-sobrancelha, olhar pesado/``bravo'', persistência de
  \emph{hooding} lateral e maior risco de insatisfação
  \textbf{Prevenção:} aplicar Teste de Connell; planejar
  estabilização/elevação do supercílio quando indicado; blefaroplastia
  mais conservadora Resgate: brow lift/pexy secundário (mais difícil
  quando já houve ressecção palpebral significativa); gestão rigorosa de
  expectativa
\item
  Erro: ressecção excessiva de ROOF (ou ``esvaziamento'' do supercílio)
  \textbf{Consequência:} perda de plenitude, esqueleto orbitário mais
  aparente e envelhecimento do terço superior \textbf{Prevenção:}
  preservar e, quando apropriado, reposicionar/redistribuir em vez de
  retirar \textbf{Resgate:} lipoenxertia em planos adequados (microfat)
  conforme indicação e perfil tecidual
\end{itemize}

\begin{quote}
\textbf{PÉROLA CLÍNICA}
\end{quote}

\begin{quote}
\mbox{}%
\subsection{Regra prática: ordem das
manobras}\label{regra-pruxe1tica-ordem-das-manobras}
\end{quote}

\begin{quote}
Em procedimentos combinados (blefaroplastia + manejo do supercílio),
planeje e estabilize/posicione o supercílio primeiro (na lógica do
caso). Só então refine a marcação/ressecção palpebral. Inverter a ordem
aumenta o risco de super-ressecção e de fechamento incompleto.
\end{quote}

\section{Notas de ``arte'' (unidades
estéticas)}\label{notas-de-arte-unidades-estuxe9ticas}

A beleza do terço superior depende de \textbf{posição e plenitude}, não
apenas de ``altura''. Elevar uma sobrancelha vazia pode criar aparência
artificial. Em alguns casos, a associação de reposicionamento e
reposição volumétrica (quando indicada) melhora naturalidade.

\section{Pós-operatório}\label{puxf3s-operatuxf3rio}

\begin{itemize}
\item
  monitorar assimetria de altura (edema unilateral pode simular paresia
  transitória);
\item
  parestesia frontal pode ocorrer em descolamentos; orientar que a
  recuperação pode levar semanas a meses, conforme técnica e extensão.
\end{itemize}

\section{Referências}\label{referuxeancias-7}

\begin{itemize}
\item
  {[}{[}REF:CONNELL-1978{]}{]} Connell --- compensação frontal e
  implicações estéticas
\item
  {[}{[}REF:KNIZE-2001{]}{]} Knize --- anatomia do supercílio/ROOF e
  princípios de brow lift
\item
  {[}{[}REF:LAMBROS-2007{]}{]} Lambros --- estética do terço superior e
  envelhecimento
\end{itemize}

\begin{center}\rule{0.5\linewidth}{0.5pt}\end{center}

\chapter{Capítulo 14 --- Técnicas de brow lift: temporal, endoscópico,
Castañares
modificado}\label{capuxedtulo-14-tuxe9cnicas-de-brow-lift-temporal-endoscuxf3pico-castauxf1ares-modificado}

\begin{figure}
\centering
\pandocbounded{\includegraphics[keepaspectratio,alt={Figura 14.1 --- Ilustração principal do capítulo}]{/Users/humbertolopes/Dev/work/marcelo-cury/the_art_of_eyelid_surgery_scaffold/projects/eyelid-surgery/assets/figures/FIG-14-01_browlift-temporal-endoscopico.png}}
\caption{Figura 14.1 --- Ilustração principal do capítulo}
\end{figure}

\textbf{Parte:} Parte III --- Terço Superior

\section{Objetivo do capítulo}\label{objetivo-do-capuxedtulo-12}

Ao final, o leitor saberá navegar pelo ``menu'' de opções de elevação do
supercílio, selecionando a técnica (temporal, endoscópica ou direta) com
base em linha capilar, padrão de ptose (lateral vs global), qualidade de
pele/rugas e necessidade de tratar glabela, evitando a aplicação
genérica de uma única técnica para todos os pacientes.

\section{O que muda na decisão (o
``porquê'')}\label{o-que-muda-na-decisuxe3o-o-porquuxea-13}

\begin{itemize}
\item
  \textbf{Linha capilar dita o acesso:} a altura da testa (glabela ao
  trichion) orienta o risco estético. Em testas longas, técnicas que
  elevam ainda mais a linha capilar tendem a piorar proporção e devem
  ser evitadas, favorecendo abordagens pré-triquiais ou diretas quando
  bem indicadas.
\item
  \textbf{Vetor de tração:} o erro comum é tracionar verticalmente e
  criar aspecto de ``susto''. Em geral, o vetor rejuvenescedor mais
  natural é oblíquo e lateral. Em muitos pacientes, o lift temporal
  isolado resolve o \emph{hooding} lateral sem necessidade de elevação
  global.
\item
  \textbf{Castañares sem estigma:} a técnica direta supraciliar é
  altamente previsível em posição/contorno quando bem indicada (idosos,
  paralisia facial, rugas profundas, necessidade funcional),
  especialmente com incisão biselada para preservar folículos e camuflar
  cicatriz.
\end{itemize}

\section{Indicações e
contra-indicações}\label{indicauxe7uxf5es-e-contra-indicauxe7uxf5es-7}

\textbf{Indicar Lift Temporal (lateral) quando:} - ptose predominante no
terço lateral (cauda);

\begin{itemize}
\item
  paciente jovem/meia-idade sem necessidade de tratar glabela
  (corrugadores/prócerus);
\item
  queixa principal for \emph{hooding} lateral.
\end{itemize}

\textbf{Indicar Endoscópico quando:} - ptose generalizada (medial e
lateral);

\begin{itemize}
\item
  necessidade de tratar musculatura glabelar com mínima cicatriz
  visível;
\item
  testa curta ou normal (aceita elevação discreta da linha capilar).
\end{itemize}

\textbf{Indicar Castañares (direto) quando:} - idosos com rugas
profundas (melhor camuflagem);

\begin{itemize}
\item
  ptose severa funcional ou paralisia facial (necessidade de fixação
  rígida e previsível);
\item
  homens com calvície/linha capilar desfavorável para incisões no couro
  cabeludo.
\end{itemize}

\textbf{Evitar / ter cautela quando:} - \textbf{endoscópico em testa
alta:} risco de piorar proporção frontal e exposição da linha capilar;

\begin{itemize}
\item
  \textbf{Castañares em jovens com pele lisa:} maior chance de cicatriz
  perceptível;
\item
  expectativa de ``arco alto'' em homens: risco de feminização/estigma.
\end{itemize}

\begin{quote}
\textbf{PÉROLA CLÍNICA}
\end{quote}

\begin{quote}
\mbox{}%
\subsection{Checklist de Segurança (Zona de
Risco)}\label{checklist-de-seguranuxe7a-zona-de-risco}
\end{quote}

\begin{quote}
\begin{itemize}
\tightlist
\item[$\square$]
  \textbf{Linha de Pitanguy:} do trago até \textasciitilde1,5 cm lateral
  ao supercílio (referência clínica para zona do ramo temporo-frontal do
  facial).
\end{itemize}
\end{quote}

\begin{quote}
\begin{itemize}
\tightlist
\item[$\square$]
  \textbf{Veia sentinela:} se identificada na dissecção temporal,
  redobre atenção ao plano (estrutura de alerta anatôm).
\end{itemize}
\end{quote}

\begin{quote}
\begin{itemize}
\tightlist
\item[$\square$]
  \textbf{Plano de dissecção:} temporal em plano seguro sobre a fáscia
  temporal profunda; fronte preferencialmente subperiosteal conforme
  técnica.
\end{itemize}
\end{quote}

\begin{quote}
\begin{itemize}
\tightlist
\item[$\square$]
  Liberação vs tração: elevação confiável exige liberação adequada de
  ancoragens; ``puxar sem soltar'' tende a falhar ou transferir tensão
  para pele/cicatriz.
\end{itemize}
\end{quote}

\section{Anatomia aplicada (planos de
deslizamento)}\label{anatomia-aplicada-planos-de-deslizamento}

\begin{itemize}
\item
  \textbf{Plano seguro temporal:} referência prática é a \textbf{fáscia
  temporal profunda} (aspecto branco/brilhante e relativamente imóvel).
  Planos mais superficiais (com gordura amarela evidente) aumentam risco
  de lesão do ramo temporal do facial.
\item
  \textbf{Ancoragens supraorbitárias (arcus/retentores):} a sobrancelha
  é estabilizada por aderências periosteais e retentores. Sem liberação
  adequada (periosteotomia/divulsão conforme técnica), a elevação é
  limitada e a tensão migra para pele.
\item
  Galea aponeurótica: camada de força para suspensão e redistribuição de
  tensão. Fixações duráveis ancoram em galea e/ou periósteo (ou túneis
  ósseos), conforme a técnica.
\end{itemize}

\begin{quote}
\textbf{📎 FIGURA NECESSÁRIA (Cap. 14):}
\end{quote}

\begin{quote}
Diagrama: Planos temporais (superficial vs profundo)
\end{quote}

\begin{quote}
\emph{Estilo: Diagrama técnico-didático, cores neutras, legendas claras}
\end{quote}

\section{Técnica (visão geral e
diferenças)}\label{tuxe9cnica-visuxe3o-geral-e-diferenuxe7as}

\subsection{1) Lift Temporal (acesso
limitado)}\label{lift-temporal-acesso-limitado}

\begin{itemize}
\item
  \textbf{Incisão:} 3--4 cm intra-capilar na região temporal.
\item
  \textbf{Dissecção:} em plano seguro sobre a fáscia temporal profunda
  até borda orbitária lateral/arco zigomático, conforme técnica.
\item
  \textbf{Liberação crítica:} liberação das ancoragens laterais para
  permitir mobilidade real do supercílio.
\item
  Fixação: suspensão do retalho móvel a estrutura fixa
  (FTP/periósteo/galea) com fio inabsorvível ou de longa duração, com
  vetor oblíquo-lateral.
\end{itemize}

\subsection{2) Endoscópico (fronte
completa)}\label{endoscuxf3pico-fronte-completa}

\begin{itemize}
\item
  \textbf{Acessos:} 3--5 incisões pequenas atrás da linha capilar.
\item
  Dissecção: subperiosteal até glabela/raiz nasal.
\item
  \textbf{Manobra chave:} identificação e preservação dos feixes
  supraorbitais/supratrocleares e tratamento controlado da musculatura
  glabelar (redução dinâmica duradoura).
\item
  Fixação: parafusos, túneis ósseos ou dispositivos absorvíveis (ex.:
  Endotine), conforme preferência e caso.
\end{itemize}

\subsection{3) Castañares modificado
(direto)}\label{castauxf1ares-modificado-direto}

\begin{itemize}
\item
  \textbf{Marcação:} fuso de pele imediatamente acima dos pelos da
  sobrancelha, definindo correção de altura e contorno.
\item
  Incisão: biselada (30--45°) para preservar folículos e permitir
  crescimento de pelos através da cicatriz.
\item
  \textbf{Ressecção:} pele/subcutâneo, preservando músculo frontal salvo
  indicação específica.
\item
  \textbf{Fechamento:} camadas meticulosas para evitar ``depressão'' e
  trilho de cicatriz.
\end{itemize}

\section{Erros comuns (e como
resgatar)}\label{erros-comuns-e-como-resgatar-9}

\begin{itemize}
\item
  \textbf{Erro:} lesão do ramo temporo-frontal do nervo facial
  \textbf{Consequência:} paresia/paralisia frontal com assimetria de
  elevação do supercílio e expressão \textbf{Prevenção:} respeito
  estrito ao plano seguro (temporal sobre FTP ou fronte subperiosteal
  conforme técnica); evitar cautério monopolar ``às cegas'' na região
  temporal lateral Resgate: observação (muitos casos são neuropraxia);
  se secção confirmada e precoce, considerar reparo microcirúrgico;
  toxina botulínica no lado contralateral para camuflagem temporária
  quando apropriado
\item
  Erro: alopecia em cicatriz (temporal/endoscópica) por trauma térmico
  ou tensão \textbf{Consequência:} área calva visível e cicatriz
  alargada \textbf{Prevenção:} técnica atraumática, cautério bipolar,
  respeito ao suprimento cutâneo, fechamento sem tensão excessiva
  \textbf{Resgate:} revisão cicatricial em tempo oportuno e/ou
  transplante capilar focal conforme caso
\end{itemize}

\begin{quote}
\textbf{PÉROLA CLÍNICA}
\end{quote}

\begin{quote}
\mbox{}%
\subsection{Regra prática: a ``bossa''
temporal}\label{regra-pruxe1tica-a-bossa-temporal}
\end{quote}

\begin{quote}
Após suspensão no lifting temporal, pode surgir pucker/dog-ear
intra-capilar.
\end{quote}

\begin{quote}
\begin{itemize}
\tightlist
\item
  \textbf{Conduta:} evitar ``perseguir'' esse excesso com ressecções
  agressivas.
\end{itemize}
\end{quote}

\begin{quote}
\begin{itemize}
\tightlist
\item
  \textbf{Expectativa:} redistribuição costuma ocorrer ao longo de
  semanas (orientar o paciente no pré-op).
\end{itemize}
\end{quote}

\begin{quote}
Excesso de tensão para aplanar pode aumentar risco de alopecia e perda
de fixação.
\end{quote}

\section{Notas de ``arte'' (o Apex)}\label{notas-de-arte-o-apex}

Onde deve ficar o ponto mais alto da sobrancelha depende de sexo e
fenótipo:

\begin{itemize}
\item
  \textbf{Clássico:} próximo ao limbo lateral.
\item
  \textbf{Moderno:} entre limbo lateral e canto externo.
\item
  \textbf{Erros típicos:} apex central (``arco de fantasia'') ou lateral
  excessivo (``Spock'').
\end{itemize}

A curva deve ser suave (sem ``\^{}''). Em homens, o objetivo costuma ser
retificação e reposicionamento no rebordo, evitando arqueamento.

\section{Pós-operatório}\label{puxf3s-operatuxf3rio-1}

\begin{itemize}
\item
  \textbf{Temporal/Endoscópico:} curativo compressivo 24--48h conforme
  preferência para reduzir seroma/hematoma.
\item
  \textbf{Castañares:} retirada de suturas em tempo adequado para
  minimizar marca (``trilho''), com orientação rigorosa de cicatriz.
\item
  \textbf{Sensibilidade:} parestesia/prurido no couro cabeludo são
  frequentes por tração nervosa; recuperação pode levar semanas a meses.
\end{itemize}

\section{Referências}\label{referuxeancias-8}

\begin{itemize}
\item
  Zona de perigo temporal e referências clínicas (veia sentinela /
  planos) {[}{[}REF:KNIZE-2001{]}{]}
\item
  Incisão biselada e preservação folicular {[}{[}REF:CONNELL-1978{]}{]}
\item
  Comparativos de técnicas e durabilidade (endo vs coronal vs temporal)
  {[}{[}REF:KNIZE-2001{]}{]}
\end{itemize}

\begin{center}\rule{0.5\linewidth}{0.5pt}\end{center}

\chapter{Capítulo 15 --- Blefaroplastia superior: pele, gordura
(preservação) e glândula
lacrimal}\label{capuxedtulo-15-blefaroplastia-superior-pele-gordura-preservauxe7uxe3o-e-gluxe2ndula-lacrimal}

\begin{figure}
\centering
\pandocbounded{\includegraphics[keepaspectratio,alt={Figura 15.1 --- Ilustração principal do capítulo}]{/Users/humbertolopes/Dev/work/marcelo-cury/the_art_of_eyelid_surgery_scaffold/projects/eyelid-surgery/assets/figures/FIG-15-01_blef-superior-preservacao.png}}
\caption{Figura 15.1 --- Ilustração principal do capítulo}
\end{figure}

\textbf{Parte:} Parte III --- Terço Superior

\section{Objetivo do capítulo}\label{objetivo-do-capuxedtulo-13}

Ao final, o leitor saberá executar uma blefaroplastia superior
\textbf{volumétrica e conservadora}, distinguindo gordura a
\textbf{preservar} daquela a \textbf{reduzir} com segurança, e
identificando prolapso da glândula lacrimal para reposicioná-la (pexy),
evitando ressecção inadvertida.

\section{O que muda na decisão (o
``porquê'')}\label{o-que-muda-na-decisuxe3o-o-porquuxea-14}

\begin{itemize}
\item
  \textbf{Gordura é estrutura, não ``sobra'':} o conceito de
  ``esvaziar'' para definir sulco é obsoleto. Ressecção agressiva
  (especialmente central) aumenta risco de deformidade em
  \textbf{A-frame} e esqueletização.
\item
  \textbf{Plenitude lateral é frequentemente ``outra coisa'':} volume
  superolateral persistente costuma ser glândula lacrimal prolapsada
  e/ou ROOF, e não gordura pré-aponeurótica ``de bolsa''. Tentar
  esvaziar sem identificação positiva é erro anatômico.
\item
  Músculo como volume e vascularização: ressecção rotineira de faixa de
  orbicular deve ser evitada em pálpebras finas ou sulcos profundos.
  Preservar músculo ajuda a manter plenitude e perfusão do retalho
  cutâneo.
\end{itemize}

\section{Indicações e
contra-indicações}\label{indicauxe7uxf5es-e-contra-indicauxe7uxf5es-8}

\textbf{Indicar quando:} - dermatochalase com ou sem herniação de
gordura;

\begin{itemize}
\item
  sensação de peso lateral por prolapso lacrimal/ROOF;
\item
  assimetria de sulco palpebral (considerar ptose e assimetria óssea
  associadas).
\end{itemize}

\textbf{Cautela extrema quando:} - \textbf{olhos encovados (deep-set /
deflation):} ressecção de gordura tende a ser contraindicada; priorizar
pele mínima (pinch-only) e preservação total de volume;

\begin{itemize}
\item
  \textbf{olho seco severo:} minimizar manipulação lacrimal e garantir
  fechamento passivo perfeito;
\item
  blefarocalásia (síndrome): operar em fase quiescente e alinhar risco
  de recidiva de frouxidão.
\end{itemize}

\begin{quote}
\textbf{PÉROLA CLÍNICA}
\end{quote}

\begin{quote}
\mbox{}%
\subsection{Checklist Pré-op de
Segurança}\label{checklist-pruxe9-op-de-seguranuxe7a}
\end{quote}

\begin{quote}
\begin{itemize}
\tightlist
\item[$\square$]
  \textbf{Palpação superolateral:} massa firme/lobulada sugere glândula
  lacrimal prolapsada.
\end{itemize}
\end{quote}

\begin{quote}
\begin{itemize}
\tightlist
\item[$\square$]
  \textbf{Schirmer (se indicado):} documentar produção lacrimal basal.
\end{itemize}
\end{quote}

\begin{quote}
\begin{itemize}
\tightlist
\item[$\square$]
  \textbf{Fenômeno de Bell:} presente/ausente (segurança corneana).
\end{itemize}
\end{quote}

\begin{quote}
\begin{itemize}
\tightlist
\item[$\square$]
  Padrão de volume: sulco profundo (preservar tudo) vs.~plenitude nasal
  (pode reduzir com parcimônia).
\end{itemize}
\end{quote}

\begin{quote}
\begin{itemize}
\tightlist
\item[$\square$]
  Assimetria: sulco, altura palpebral, sobrancelha (documentar).
\end{itemize}
\end{quote}

\begin{quote}
\begin{itemize}
\tightlist
\item[$\square$]
  Lagoftalmo prévio: relato de dormir com olho entreaberto.
\end{itemize}
\end{quote}

\section{Anatomia aplicada (o mapa da
mina)}\label{anatomia-aplicada-o-mapa-da-mina}

\begin{itemize}
\item
  \textbf{Compartimentos de gordura clinicamente relevantes:}
\item
  \textbf{Nasal:} mais pálida/fibrosa; frequentemente protrui e pode ser
  reduzida \textbf{conservadoramente}.
\item
  Central: mais amarela/fluida; importante para plenitude jovem; evitar
  ressecção rotineira (reduzir apenas se herniação franca).
\item
  Volume superolateral: com frequência corresponde a glândula lacrimal
  prolapsada e/ou ROOF, e não a uma ``bolsa lateral'' típica.
\item
  Glândula lacrimal: compartimento lateral, aspecto rosado/acinzentado,
  firme e lobulado; diferente da gordura (amarela, macia).
\item
  ROOF (Retro-Orbicularis Oculi Fat): gordura sob o orbicular e acima do
  septo; dá volume ao supercílio e pode descer com a idade.
\end{itemize}

\begin{quote}
\textbf{📎 FIGURA NECESSÁRIA (Cap. 15):}
\end{quote}

\begin{quote}
Anatomia: Gordura pré-aponeurótica, glândula lacrimal, ROOF
\end{quote}

\begin{quote}
\emph{Estilo: Diagrama técnico-didático, cores neutras, legendas claras}
\end{quote}

\section{Técnica (preservação e
escultura)}\label{tuxe9cnica-preservauxe7uxe3o-e-escultura}

\subsection{Visão geral}\label{visuxe3o-geral-6}

\begin{enumerate}
\def\labelenumi{\arabic{enumi}.}
\item
  \textbf{Incisão e excisão de pele:} conforme marcação (Cap. 11).
  Preservar músculo sempre que possível.
\item
  \textbf{Abertura do septo:} mínima e dirigida ao ponto de herniação.
  Evitar abrir de ponta a ponta sem necessidade.
\item
  \textbf{Gordura nasal:} identificar (mais pálida). Infiltrar na base,
  controlar pedículo, reduzir com parcimônia e hemostasia meticulosa.
\item
  Gordura central: regra é preservar. Se proeminente, preferir redução
  mínima e controlada; evitar ``esvaziamento''.
\item
  Glândula lacrimal (se prolapsada): não ressecar. Realizar pexia com
  pontos (ex.: Nylon 5-0/6-0) fixando cápsula/tecido de suporte ao
  periósteo da fossa lacrimal superolateral.
\item
  Fechamento: sutura contínua ou pontos separados (Nylon/Prolene 6-0),
  com eversão suave das bordas.
\end{enumerate}

\subsection{Variações e
indicações}\label{variauxe7uxf5es-e-indicauxe7uxf5es-7}

\begin{itemize}
\item
  \textbf{Fixação do sulco (ancoragem):} em casos selecionados (pálpebra
  espessa / asiáticos / necessidade de sulco), ancorar derme/orbicular
  ao tarso/aponeurose conforme planejamento.
\item
  \textbf{ROOF:} se houver espessamento lateral significativo, pode-se
  reduzir uma pequena faixa com cautela (evitar agressividade; respeitar
  planos e estruturas sensitivas).
\end{itemize}

\begin{quote}
\textbf{PÉROLA CLÍNICA}
\end{quote}

\begin{quote}
\mbox{}%
\subsection{Zona de Risco: Glândula Lacrimal (o ``Erro Nota
7'')}\label{zona-de-risco-gluxe2ndula-lacrimal-o-erro-nota-7}
\end{quote}

\begin{quote}
\textbf{Diferenciação prática}
\end{quote}

\begin{quote}
\begin{itemize}
\tightlist
\item
  Gordura: amarela, mole, ``desmancha'' na pinça.
\end{itemize}
\end{quote}

\begin{quote}
\begin{itemize}
\tightlist
\item
  Glândula: rosa/cinza, firme, lobulada.
\end{itemize}
\end{quote}

\begin{quote}
\textbf{Regra de ouro:} nunca ressecar tecido no compartimento
superolateral sem \textbf{identificação positiva}. Na dúvida, preserve
e/ou faça pexia.
\end{quote}

\section{Erros comuns (e como
resgatar)}\label{erros-comuns-e-como-resgatar-10}

\begin{itemize}
\item
  \textbf{Erro:} ressecção excessiva do compartimento central (A-frame)
  \textbf{Consequência:} sulco profundo central, aspecto esqueletizado,
  ``olhar operado'' \textbf{Prevenção:} preservação do volume central
  como padrão; reduzir apenas herniação franca e com parcimônia Resgate:
  preenchimento profundo (HA) e/ou lipoenxertia (micro/nanofat) em plano
  adequado
\item
  Erro: confundir glândula lacrimal com gordura e lesá-la
  \textbf{Consequência:} massa dolorosa crônica lateral, cisto, fístula
  ou irritação persistente \textbf{Prevenção:} identificação visual +
  regra de preservação no compartimento lateral; evitar energia térmica
  próxima à glândula \textbf{Resgate:} avaliação especializada;
  tratamento do trajeto/cisto (excisão/marsupialização) conforme caso
\end{itemize}

\section{Notas de ``arte'' (volume e
luz)}\label{notas-de-arte-volume-e-luz}

A pálpebra jovem reflete luz de forma difusa. A pálpebra ``operada
demais'' cria sombras duras no sulco orbitário superior. O objetivo é
manter uma transição convexa suave até o supercílio, resistindo à
tentação de remover ``tudo o que protrui''.

\section{Pós-operatório}\label{puxf3s-operatuxf3rio-2}

\begin{itemize}
\item
  \textbf{Gelo:} compressas frias por 48h para reduzir edema.
\item
  \textbf{Lubrificação:} colírio durante o dia e gel à noite; edema pode
  reduzir piscamento completo nos primeiros dias.
\item
  \textbf{Retirada de pontos:} 5--7 dias.
\end{itemize}

\section{Referências}\label{referuxeancias-9}

\begin{itemize}
\item
  Compartimentos de gordura superior {[}{[}REF:ROHRICH-2008{]}{]}
\item
  Reposicionamento/pexy da glândula lacrimal {[}{[}REF:MCCORD-1995{]}{]}
\item
  Princípios de preservação de volume na blefaroplastia superior
  {[}{[}REF:FAGIEN-1999{]}{]}
\end{itemize}

\begin{center}\rule{0.5\linewidth}{0.5pt}\end{center}

\chapter{Capítulo 16 --- Ptose associada no superior: quando reconhecer
e como integrar ao
plano}\label{capuxedtulo-16-ptose-associada-no-superior-quando-reconhecer-e-como-integrar-ao-plano}

\begin{figure}
\centering
\pandocbounded{\includegraphics[keepaspectratio,alt={Figura 16.1 --- Ilustração principal do capítulo}]{/Users/humbertolopes/Dev/work/marcelo-cury/the_art_of_eyelid_surgery_scaffold/projects/eyelid-surgery/assets/figures/FIG-16-01_ptose-avaliacao-integracao.png}}
\caption{Figura 16.1 --- Ilustração principal do capítulo}
\end{figure}

\textbf{Parte:} Parte III --- Terço Superior

\section{Objetivo do capítulo}\label{objetivo-do-capuxedtulo-14}

Ao final, o leitor saberá diagnosticar a ptose palpebral oculta sob a
dermatocalase e integrar sua correção ao ato da blefaroplastia, evitando
o resultado frustrante de um ``olho limpo, mas ainda cansado'' e
prevenindo assimetrias tardias pela Lei de Hering.

\section{O que muda na decisão (o
``porquê'')}\label{o-que-muda-na-decisuxe3o-o-porquuxea-15}

\begin{itemize}
\item
  \textbf{A falácia da ``pálpebra pesada'':} pacientes confundem excesso
  de pele com fraqueza do elevador. Remover pele em um olho ptótico pode
  \textbf{expor} a ptose e fazer o olho parecer menor no pós-op. O
  diagnóstico é por \textbf{medida (MRD1)} e exame, não por
  ``sensação''.
\item
  Lei de Hering (a gangorra): o comando neural tende a ser compartilhado
  entre ambos os elevadores. Corrigir o lado mais ptótico pode revelar
  queda do contralateral. Ignorar isso aumenta muito a chance de
  assimetria e retrabalho.
\item
  O plano anatômico muda: blefaroplastia ``simples'' costuma parar em
  septo/gordura. Ptose via anterior exige identificar e trabalhar a
  aponeurose do elevador. A decisão de entrar nesse plano deve ser
  definida antes da incisão.
\end{itemize}

\section{Indicações e
contra-indicações}\label{indicauxe7uxf5es-e-contra-indicauxe7uxf5es-9}

\textbf{Indicar correção concomitante quando:} - MRD1 \textless{} 3,5 mm
\textbf{ou} assimetria \textgreater{} 1 mm entre os olhos;

\begin{itemize}
\item
  sinais de deiscência aponeurótica (sulco alto, boa função do elevador
  com queda, ``show'' de gordura pré-aponeurótica);
\item
  teste da fenilefrina positivo (candidato a via posterior/Müller em
  ptoses leves).
\end{itemize}

\textbf{Evitar / adiar quando:} - \textbf{pseudoptose:} dermatocalase
mecânica, retração contralateral, hipotropia, enoftalmo (o alvo não é o
elevador);

\begin{itemize}
\item
  \textbf{olho seco severo:} elevar a pálpebra pode piorar exposição;
  prioridade é segurança;
\item
  miastenia gravis: ptose flutuante; cirurgia só após estabilização e
  critério rigoroso.
\end{itemize}

\begin{quote}
\textbf{PÉROLA CLÍNICA}
\end{quote}

\begin{quote}
\mbox{}%
\subsection{Checklist Pré-op de
Diagnóstico}\label{checklist-pruxe9-op-de-diagnuxf3stico}
\end{quote}

\begin{quote}
\begin{itemize}
\tightlist
\item[$\square$]
  \textbf{MRD1 (Margem--Reflexo):} distância do reflexo corneano à
  margem ciliar superior (normal \textasciitilde4--5 mm).
\end{itemize}
\end{quote}

\begin{quote}
\begin{itemize}
\tightlist
\item[$\square$]
  \textbf{Função do elevador:} excursão bloqueando o frontal.
  \textgreater12 mm excelente; \textless4 mm sugere ptose miogênica
  grave (considerar suspensão frontal).
\end{itemize}
\end{quote}

\begin{quote}
\begin{itemize}
\tightlist
\item[$\square$]
  \textbf{Lei de Hering:} elevar manualmente o lado mais ptótico e
  observar queda do contralateral.
\end{itemize}
\end{quote}

\begin{quote}
\begin{itemize}
\tightlist
\item[$\square$]
  Teste da fenilefrina: se subir em 5--10 min, favorece via posterior
  (ptose leve, seleção criteriosa).
\end{itemize}
\end{quote}

\begin{quote}
\begin{itemize}
\tightlist
\item[$\square$]
  Posição do sulco: sulco alto/profundo sugere desinserção aponeurótica.
\end{itemize}
\end{quote}

\section{Anatomia aplicada (o ``sanduíche'' do
elevador)}\label{anatomia-aplicada-o-sanduuxedche-do-elevador}

\begin{itemize}
\item
  \textbf{Gordura pré-aponeurótica:} marco anatômico (``farol''). Ao
  expô-la e retraí-la, você aproxima o campo da aponeurose do elevador
  (branca/perolada). Se você não identificou a gordura pré-aponeurótica
  corretamente, provavelmente ainda não está no plano certo.
\item
  \textbf{Ligamento de Whitnall:} condensação fascial superior ao
  elevador; funciona como fulcro. Em casos selecionados, ajuda como
  referência/estabilidade.
\item
  \textbf{Músculo de Müller:} abaixo da aponeurose e aderido à
  conjuntiva; simpático-dependente. Manipulação costuma sangrar mais que
  a aponeurose e exige delicadeza.
\end{itemize}

\begin{quote}
\textbf{📎 FIGURA NECESSÁRIA (Cap. 16):}
\end{quote}

\begin{quote}
Diagrama diagnóstico: MRD1, função do elevador, teste de fenilefrina
\end{quote}

\begin{quote}
\emph{Estilo: Diagrama técnico-didático, cores neutras, legendas claras}
\end{quote}

\section{Técnica (integração na
blefaroplastia)}\label{tuxe9cnica-integrauxe7uxe3o-na-blefaroplastia}

\subsection{Visão geral (via anterior --- Levator
Advancement)}\label{visuxe3o-geral-via-anterior-levator-advancement}

\begin{enumerate}
\def\labelenumi{\arabic{enumi}.}
\item
  \textbf{Exposição:} realizar blefaroplastia superior conforme plano
  (pele ± gordura) com preservação de volume.
\item
  \textbf{Identificação:} retrair a gordura pré-aponeurótica; visualizar
  a aponeurose do elevador.
\item
  \textbf{Avaliação do padrão:} desinserida/retraída vs.~inserida porém
  frouxa.
\item
  Avanço/reinserção: fixar a aponeurose na face anterior do tarso com
  1--3 pontos (ex.: Nylon/Prolene 6-0; conforme preferência).
\item
  Ajuste intraoperatório: sentar o paciente e checar altura/contorno;
  ajustar a tensão antes do fechamento cutâneo (sedação profunda
  atrapalha).
\end{enumerate}

\subsection{Variações e
indicações}\label{variauxe7uxf5es-e-indicauxe7uxf5es-8}

\begin{itemize}
\item
  \textbf{Via posterior (Mülerectomia/Conjuntivectomia):} ptoses leves
  (1--2 mm) com fenilefrina positiva; boa para contorno, mas não
  substitui correções mais potentes quando indicado.
\item
  \textbf{Plicatura simples:} dobra aponeurótica sem grande
  descolamento; menos potente, pode ser útil em casos selecionados.
\end{itemize}

\begin{quote}
\textbf{PÉROLA CLÍNICA}
\end{quote}

\begin{quote}
\mbox{}%
\subsection{Erro Nota 7: ``Peaking'' (pico) no
contorno}\label{erro-nota-7-peaking-pico-no-contorno}
\end{quote}

\begin{quote}
Se o ponto central estiver excessivamente tenso ou mal posicionado, a
pálpebra assume formato triangular (\^{}).
\end{quote}

\begin{quote}
\textbf{Regra prática:} o pico fisiológico tende a ficar levemente nasal
à pupila, e o contorno é arredondado com pontos medial e lateral
complementares.
\end{quote}

\section{Erros comuns (e como
resgatar)}\label{erros-comuns-e-como-resgatar-11}

\begin{itemize}
\item
  \textbf{Erro:} ajustar a correção com o paciente totalmente deitado e
  sem cooperação \textbf{Consequência:} hipo/hipercorreção e assimetria
  no pós-op imediato \textbf{Prevenção:} ajuste exige comando
  voluntário; manter sedação leve e sentar o paciente para conferência
  final Resgate: reabordagem precoce (primeiros dias/1ª semana) para
  reposicionar pontos quando indicado
\item
  Erro: elevar demais em olhos com risco de exposição (olho seco/Bell
  fraco) \textbf{Consequência:} lagoftalmo, ceratite de exposição, piora
  importante do conforto ocular \textbf{Prevenção:} meta conservadora
  (MRD1 funcional) e respeito a margem de segurança; planejar
  lubrificação intensiva \textbf{Resgate:} medidas mecânicas
  (massagem/``stretching'' orientado), lubrificação agressiva; revisão
  cirúrgica se persistente e clinicamente relevante
\end{itemize}

\section{Notas de ``arte'' (o olhar
desperto)}\label{notas-de-arte-o-olhar-desperto}

A ``arte'' é a simetria do brilho: o reflexo corneano deve parecer
equilibrado entre os olhos. Em geral, a pálpebra superior deve cobrir
\textasciitilde1--2 mm da íris superior. Cobertura maior sugere ptose;
cobertura menor, risco de ``olhar assustado'' (retração).

\section{Pós-operatório}\label{puxf3s-operatuxf3rio-3}

\begin{itemize}
\item
  \textbf{Fechamento:} checar se o olho fecha durante o sono nas
  primeiras noites.
\item
  \textbf{Lubrificação:} gel oftálmico noturno por 3--4 semanas (ou
  conforme necessidade).
\item
  \textbf{Expectativa:} nos primeiros dias pode parecer ``aberto
  demais'' por edema e efeito simpático (Müller).
\end{itemize}

\section{Referências}\label{referuxeancias-10}

\begin{itemize}
\item
  Anatomia cirúrgica da aponeurose do elevador
  {[}{[}REF:MCCORD-1995{]}{]}.
\item
  Fenômeno de Hering e planejamento cirúrgico
  {[}{[}REF:BODIAN-1982{]}{]}.
\item
  Algoritmos: via anterior vs.~via posterior
  {[}{[}REF:PUTTERMAN-1975{]}{]}. * * *
\end{itemize}

\begin{center}\rule{0.5\linewidth}{0.5pt}\end{center}

\chapter{Capítulo 17 --- Pálpebra inferior transconjuntival:
preferências, septo e
bolsas}\label{capuxedtulo-17-puxe1lpebra-inferior-transconjuntival-preferuxeancias-septo-e-bolsas}

\begin{figure}
\centering
\pandocbounded{\includegraphics[keepaspectratio,alt={Figura 17.1 --- Ilustração principal do capítulo}]{/Users/humbertolopes/Dev/work/marcelo-cury/the_art_of_eyelid_surgery_scaffold/projects/eyelid-surgery/assets/figures/FIG-17-01_transconjuntival-septo-bolsas.png}}
\caption{Figura 17.1 --- Ilustração principal do capítulo}
\end{figure}

\begin{quote}
\textbf{Leitura guiada:} este capítulo aborda preservar) com menor risco
de retração palpebral (ectrópio/round eye) quando comparada à abordagem
externa em fenótipos selecionados.
\end{quote}

\textbf{Parte:} Parte IV --- Inferior e Terço Médio

\section{Objetivo do capítulo}\label{objetivo-do-capuxedtulo-15}

Ao final, o leitor dominará a via de acesso interna que elimina a
cicatriz cutânea e preserva a lamela média, aprendendo a modular as
bolsas de gordura (remover vs.~preservar) com menor risco de retração
palpebral (ectrópio/round eye) quando comparada à abordagem externa em
fenótipos selecionados.

\section{O que muda na decisão (o
``porquê'')}\label{o-que-muda-na-decisuxe3o-o-porquuxea-16}

\begin{itemize}
\item
  \textbf{Preservação da lamela média:} a via transconjuntival evita
  violar pele + orbicular quando o problema é predominantemente gordura.
  Isso reduz a chance de cicatrização retrátil anterior e ``pull-down''
  da pálpebra inferior.
\item
  \textbf{Acesso direto ao alvo:} quando a queixa principal é bolsa
  (gordura) e não pele, atravessar pele e músculo para chegar às bolsas
  costuma ser dissecção ``extra'' sem ganho anatômico.
\item
  \textbf{O mito da ``pele sobrando'':} em muitos pacientes, a aparência
  de sobra é secundária à protrusão de gordura. Ao tratar o volume por
  dentro, parte da pele melhora por retração/redistribuição --- mas isso
  depende de tônus (snap-back). A decisão correta separa pele real de
  pseudo-sobra por volume.
\end{itemize}

\section{Indicações e
contra-indicações}\label{indicauxe7uxf5es-e-contra-indicauxe7uxf5es-10}

\textbf{Indicar quando:} - bolsas de gordura evidentes (herniação) com
\textbf{bom tônus} (snap-back adequado);

\begin{itemize}
\item
  pacientes jovens ou meia-idade com pouca pele excedente;
\item
  pele com maior risco de discromia/cicatriz visível em incisão
  subciliar (ex.: fenótipos com tendência a hiperpigmentação
  pós-inflamatória);
\item
  intenção de associar \textbf{resurfacing} (laser/peeling) mantendo a
  pele íntegra e bem vascularizada.
\end{itemize}

\textbf{Evitar / preferir estratégia combinada quando:} -
\textbf{excesso real de pele} (pinch significativo) ou flacidez senil
com baixa retração;

\begin{itemize}
\item
  festoons/edema malar predominante (não é ``bolsa de gordura'' e pode
  exigir outro plano);
\item
  componente muscular importante (orbicular roll/hipertrofia dinâmica)
  como causa principal;
\item
  frouxidão horizontal relevante (distraction elevado / snap-back lento)
  sem plano de suporte (cantopexia/cantoplastia).
\end{itemize}

\begin{quote}
\textbf{PÉROLA CLÍNICA}
\end{quote}

\begin{quote}
\mbox{}%
\subsection{Checklist Pré-op
(Segurança)}\label{checklist-pruxe9-op-seguranuxe7a}
\end{quote}

\begin{quote}
\begin{itemize}
\tightlist
\item[$\square$]
  \textbf{Snap-back / Distraction:} define se transconjuntival será
  ``pura'' ou se precisa associar suporte/pinch/resurfacing.
\end{itemize}
\end{quote}

\begin{quote}
\begin{itemize}
\tightlist
\item[$\square$]
  \textbf{Dinâmica do volume:} aumenta ao sorrir? (orbicular)
  vs.~estático? (gordura).
\end{itemize}
\end{quote}

\begin{quote}
\begin{itemize}
\tightlist
\item[$\square$]
  \textbf{Bolsas por compartimento:} medial / central / lateral
  (planejar cada uma, não ``um bloco'').
\end{itemize}
\end{quote}

\begin{quote}
\begin{itemize}
\tightlist
\item[$\square$]
  Motilidade ocular: documentar para não confundir diplopia prévia com
  complicação.
\end{itemize}
\end{quote}

\begin{quote}
\begin{itemize}
\tightlist
\item[$\square$]
  Risco hemorrágico: anticoagulantes/antiagregantes e suplementos
  (protocolo de suspensão).
\end{itemize}
\end{quote}

\section{Anatomia aplicada (o campo
minado)}\label{anatomia-aplicada-o-campo-minado}

\begin{itemize}
\tightlist
\item
  \textbf{Bolsas de gordura (3 compartimentos):}
\end{itemize}

\begin{enumerate}
\def\labelenumi{\arabic{enumi}.}
\tightlist
\item
  \textbf{Medial:} mais pálida e fibrosa.
\item
  \textbf{Central:} amarela, mais fluida.
\item
  Lateral: menor/mais profunda e frequentemente subtratada.
\end{enumerate}

\begin{itemize}
\item
  Divisor crítico: músculo oblíquo inferior (entre medial e central).
  Lesão → risco de diplopia.
\item
  Plano conjuntiva--retratores: incisão geralmente alguns milímetros
  abaixo da borda inferior do tarso para reduzir sangramento e facilitar
  a exposição.
\item
  Arcadas vasculares: atenção ao controle fino com bipolar; sangramento
  ``chato'' atrapalha a identificação do oblíquo e dos compartimentos.
\end{itemize}

\begin{quote}
\textbf{📎 FIGURA NECESSÁRIA (Cap. 17):}
\end{quote}

\begin{quote}
Corte sagital: Acesso transconjuntival pré e pós-septal
\end{quote}

\begin{quote}
\emph{Estilo: Diagrama técnico-didático, cores neutras, legendas claras}
\end{quote}

\section{Técnica (acesso e gestão)}\label{tuxe9cnica-acesso-e-gestuxe3o}

\subsection{Visão geral}\label{visuxe3o-geral-7}

\begin{enumerate}
\def\labelenumi{\arabic{enumi}.}
\item
  \textbf{Exposição:} eversão adequada com Desmarres e/ou pontos de
  tração. Proteger córnea conforme preferência (shield).
\item
  \textbf{Incisão conjuntival:} conjuntiva + retratores
  (capsulopalpebral) em linha horizontal, tipicamente alguns mm abaixo
  do tarso.
\item
  \textbf{Escolha do plano (pré-septal vs.~pós-septal):}
\end{enumerate}

\begin{itemize}
\tightlist
\item
  Pós-septal (direto na gordura): mais rápido, menos dissecção.
\item
  Pré-septal (``book view''): melhor leitura anatômica e facilita
  reposicionamento/transposição quando indicado.
\end{itemize}

\begin{enumerate}
\def\labelenumi{\arabic{enumi}.}
\setcounter{enumi}{3}
\tightlist
\item
  Herniação controlada: pressão suave no globo para expor gordura por
  compartimento.
\item
  Manejo por compartimento: abrir cápsula, individualizar, tratar
  medial/central/lateral separadamente.
\item
  Ressecção conservadora (quando indicada): clamp/controle + secção +
  hemostasia antes de soltar. Evitar ``deixar flush'' no rebordo.
\item
  Fechamento: muitas vezes sem sutura; se necessário, 1 ponto absorvível
  frouxo para coaptação sem tensionar.
\end{enumerate}

\subsection{Variações e
indicações}\label{variauxe7uxf5es-e-indicauxe7uxf5es-9}

\begin{itemize}
\item
  \textbf{Transposição/Reposicionamento de gordura:} --- preferir quando
  há sulco lacrimal/junção pálpebra-malar marcada (objetivo:
  continuidade).
\item
  \textbf{Pinch cutâneo associado:} após transconjuntival, remover 1--2
  mm de pele (skin-only) quando há excesso real mínimo, sem descolar
  orbicular.
\item
  \textbf{Suporte canthal associado:} em tônus limítrofe, planejar
  cantopexia/cantoplastia conforme algoritmo (capítulos de exame
  físico/planejamento).
\end{itemize}

\begin{quote}
\textbf{PÉROLA CLÍNICA}
\end{quote}

\begin{quote}
\mbox{}%
\subsection{Zona de Risco: O Oblíquo
Inferior}\label{zona-de-risco-o-obluxedquo-inferior}
\end{quote}

\begin{quote}
\textbf{Onde:} entre a bolsa medial e a central.
\end{quote}

\begin{quote}
\textbf{Erro clássico:} confundir músculo com septos/fáscias e
lesar/cauterizar.
\end{quote}

\begin{quote}
\textbf{Regra prática:} antes de clampear a medial, procure o oblíquo
ativamente. Estrutura vermelho-carnosa, com orientação característica.
Na dúvida: pare, irrigue, reidentifique.
\end{quote}

\section{Erros comuns (e como
resgatar)}\label{erros-comuns-e-como-resgatar-12}

\begin{itemize}
\item
  \textbf{Erro:} subtratar a bolsa lateral \textbf{Consequência:}
  ``caroço'' persistente no canto externo apesar do centro liso
  \textbf{Prevenção:} exploração ativa do compartimento lateral (pressão
  adequada + dissecção suficiente) Resgate: retoque focal sob anestesia
  local, com abordagem direcionada
\item
  Erro: ressecção excessiva (olho cavo) \textbf{Consequência:}
  esqueletização/vales profundos, estigma de ``olhar operado''
  \textbf{Prevenção:} conservadorismo; tratar por compartimento e evitar
  ressecar até ``zerar'' no rebordo \textbf{Resgate:} lipoenxertia
  (micro/nanofat) ou preenchimento profundo (seleção criteriosa)
\item
  Erro: ignorar tônus ruim e não planejar suporte \textbf{Consequência:}
  round eye/ectrópio funcional ou estético \textbf{Prevenção:} usar
  snap-back/distraction como ``semáforo'' para cantopexia/cantoplastia
  e/ou associar pinch/resurfacing em vez de insistir em transconjuntival
  pura \textbf{Resgate:} suporte tardio (tarsal strip/pexia) e, em casos
  selecionados, enxertos espaçadores/pele
\end{itemize}

\section{Notas de ``arte''
(continuidade)}\label{notas-de-arte-continuidade-1}

O objetivo não é ``retificar'' a pálpebra inferior, mas manter uma
convexidade suave e continuidade com o malar. A transconjuntival é uma
via de \emph{sculpting}: você controla volume sem criar uma cicatriz
cutânea que ``puxe'' a margem inferior.

\section{Pós-operatório}\label{puxf3s-operatuxf3rio-4}

\begin{itemize}
\item
  \textbf{Quemose:} comum. Manejo com lubrificação intensa; considerar
  colírios conforme protocolo. A expectativa deve ser alinhada (pode
  durar semanas).
\item
  \textbf{Conforto e proteção:} orientar sinais de alarme (dor intensa,
  proptose, queda visual) dentro do protocolo geral de segurança.
\item
  \textbf{Massagem:} geralmente não é foco (ausência de cicatriz cutânea
  tensionada), mas conduta final depende do caso e do edema.
\end{itemize}

\section{Referências}\label{referuxeancias-11}

\begin{itemize}
\item
  Via transconjuntival clássica {[}{[}REF:GOLDBERG-1998{]}{]}.
\item
  Pré-septal vs.~pós-septal {[}{[}REF:HAMRA-1995{]}{]}.
\item
  Preservação/redistribuição de gordura e arcus marginalis
  {[}{[}REF:HAMRA-1995{]}{]}. * * *
\end{itemize}

\begin{center}\rule{0.5\linewidth}{0.5pt}\end{center}

\chapter{Capítulo 18 --- Transposição/redistribuição de gordura: sulco
nasojugal e transição
pálpebra-malar}\label{capuxedtulo-18-transposiuxe7uxe3oredistribuiuxe7uxe3o-de-gordura-sulco-nasojugal-e-transiuxe7uxe3o-puxe1lpebra-malar}

\begin{figure}
\centering
\pandocbounded{\includegraphics[keepaspectratio,alt={Figura 18.1 --- Ilustração principal do capítulo}]{/Users/humbertolopes/Dev/work/marcelo-cury/the_art_of_eyelid_surgery_scaffold/projects/eyelid-surgery/assets/figures/FIG-18-01_transposicao-gordura-lid-cheek..png}}
\caption{Figura 18.1 --- Ilustração principal do capítulo}
\end{figure}

\textbf{Parte:} Parte IV --- Inferior e Terço Médio

\section{Objetivo do capítulo}\label{objetivo-do-capuxedtulo-16}

Ao final, o leitor dominará os princípios de \textbf{nivelar convexidade
(bolsa)} e \textbf{preencher concavidade (sulco)} com abordagem de
preservação, aprendendo a liberar as estruturas retentoras e
reposicionar gordura para restaurar uma transição contínua na junção
pálpebra--malar.

\section{O que muda na decisão (o
``porquê'')}\label{o-que-muda-na-decisuxe3o-o-porquuxea-17}

\begin{itemize}
\item
  \textbf{Paradigma da preservação:} ressecar bolsa trata a convexidade,
  mas pode \textbf{acentuar} a concavidade do sulco. A decisão por
  reposicionar/redistribuir busca corrigir \textbf{monte + vale} com o
  próprio tecido disponível.
\item
  A barreira retentora: falhas no tratamento do sulco costumam ser
  consequência de liberação insuficiente das estruturas retentoras na
  junção pálpebra--malar (ex.: ORL/TTL, dependendo da escola anatômica e
  da técnica). Sem mobilidade real, o reposicionamento fica instável ou
  irregular.
\item
  Volume autólogo vs.~camuflagem temporária: ao contrário de
  preenchedores, a gordura reposicionada é tecido autólogo; pode
  oferecer continuidade de relevo mais robusta. Isso não ``apaga cor''
  por pigmento, mas pode reduzir a aparência de olheira estrutural ao
  suavizar sombra e, em alguns casos, atenuar a visualização vascular
  por aumento de espessura de cobertura.
\end{itemize}

\section{Indicações e
contra-indicações}\label{indicauxe7uxf5es-e-contra-indicauxe7uxf5es-11}

\textbf{Indicar quando:} - coexistirem \textbf{bolsas} e \textbf{sulco
nasojugal profundo} (ex.: classes moderadas a avançadas em
classificações usuais);

\begin{itemize}
\item
  houver risco de olho cavo/deflação (incluindo fenótipos com tendência
  a esqueletização), onde ressecção agressiva é indesejável;
\item
  a junção pálpebra--malar estiver longa/abrupta (lid--cheek junction
  marcada), com objetivo de continuidade.
\end{itemize}

\textbf{Evitar / adiar quando:} - não houver gordura septal suficiente
para reposicionar (planejar alternativa: micro/nanofat, outras manobras
de suporte/terço médio);

\begin{itemize}
\item
  a queixa for predominantemente \textbf{pigmentar} (melanina) sem
  componente de sombra/relevo relevante;
\item
  houver processo infeccioso ativo na região (ex.: celulite, sinusopatia
  relevante com repercussão local) --- priorizar estabilidade clínica.
\end{itemize}

\begin{quote}
\textbf{PÉROLA CLÍNICA}
\end{quote}

\begin{quote}
\mbox{}%
\subsection{Checklist Pré-op (Avaliação de Volume e
Viabilidade)}\label{checklist-pruxe9-op-avaliauxe7uxe3o-de-volume-e-viabilidade}
\end{quote}

\begin{quote}
\begin{itemize}
\tightlist
\item[$\square$]
  \textbf{Mapa do defeito:} medial (tear trough) vs.~lateral (lid--cheek
  junction).
\end{itemize}
\end{quote}

\begin{quote}
\begin{itemize}
\tightlist
\item[$\square$]
  \textbf{Volume disponível:} gordura medial/central suficiente para
  cobertura do rebordo + sulco?
\end{itemize}
\end{quote}

\begin{quote}
\begin{itemize}
\tightlist
\item[$\square$]
  \textbf{Qualidade da pele:} muito fina aumenta risco de
  irregularidades visíveis → considerar refinamento (ex.: nanofat/skin
  quality adjunct) conforme estratégia.
\end{itemize}
\end{quote}

\begin{quote}
\begin{itemize}
\tightlist
\item[$\square$]
  Vetor e tônus: vetor negativo e/ou tônus limítrofe exigem plano
  explícito de suporte (cantopexia/cantoplastia e/ou estratégia de terço
  médio) além do volume.
\end{itemize}
\end{quote}

\begin{quote}
\begin{itemize}
\tightlist
\item[$\square$]
  Foto sem e com flash: separar sombra (relevo) de pigmento (cor) e
  documentar expectativa realista.
\end{itemize}
\end{quote}

\section{Anatomia aplicada (o obstáculo e o
alvo)}\label{anatomia-aplicada-o-obstuxe1culo-e-o-alvo}

\begin{itemize}
\item
  Tear Trough Ligament (TTL) / estruturas retentoras mediais: aderem
  pele/tecidos ao rebordo medial e contribuem para o sulco e a sombra. O
  objetivo funcional é \textbf{mobilidade} suficiente para permitir
  acomodação suave do volume.
\item
  \textbf{Orbicular Retaining Ligament (ORL):} contribui para a
  demarcação da junção pálpebra--malar (mais lateral) e para a ``linha''
  da transição. Se permanecer íntegro e tenso, pode manter o
  degrau/sombra.
\item
  \textbf{Arcus marginalis / rebordo orbitário:} marco anatômico. O
  reposicionamento efetivo geralmente visa ``cruzar'' essa transição de
  plano, suavizando a fronteira osso--tecido.
\item
  Espaço pré-malar / pré-maxilar: compartimento de acomodação (conceito
  varia com técnica: sub-orbicular, pré/periostal, subperiostal).
  Atenção a estruturas vasculares mediais e ao feixe infraorbitário.
\end{itemize}

\begin{quote}
\textbf{📎 FIGURA NECESSÁRIA (Cap. 18):}
\end{quote}

\begin{quote}
Técnica: Fat repositioning para sulco nasojugal
\end{quote}

\begin{quote}
\emph{Estilo: Diagrama técnico-didático, cores neutras, legendas claras}
\end{quote}

\section{Técnica (princípios de liberação, reposicionamento e
estabilidade)}\label{tuxe9cnica-princuxedpios-de-liberauxe7uxe3o-reposicionamento-e-estabilidade}

\subsection{Visão geral}\label{visuxe3o-geral-8}

\begin{enumerate}
\def\labelenumi{\arabic{enumi}.}
\item
  \textbf{Acesso:} preferencialmente transconjuntival quando o objetivo
  é preservar lamelas; transcutâneo quando há indicação cutânea/terço
  médio que justifique.
\item
  \textbf{Exposição e individualização:} identificar compartimentos
  (medial/central) e mobilizar gordura com respeito ao pedículo,
  evitando trauma por esmagamento.
\item
  \textbf{Criação do leito receptor:} realizar liberação adequada das
  estruturas retentoras conforme alvo (medial vs.~lid--cheek), criando
  um plano/lóculo que permita acomodação sem compressão e sem tensão.
\item
  Reposicionamento: acomodar o volume para cobrir o rebordo e suavizar o
  sulco, buscando transição contínua (evitar ``degrau inverso'').
\item
  Estabilidade (fixação quando indicada):
\end{enumerate}

\begin{itemize}
\tightlist
\item
  Fixação externa temporária (bolster): pode ser usada em algumas
  técnicas abertas, com controle de vetor e duração curta.
\item
  Fixação interna: suturas de estabilização em tecidos profundos/planos
  de suporte conforme técnica preferida, evitando transformar a gordura
  em ``cabo de tração''.
\end{itemize}

\subsection{Variações e
indicações}\label{variauxe7uxf5es-e-indicauxe7uxf5es-10}

\begin{itemize}
\item
  \textbf{Septal reset / variações clássicas (via aberta):} técnicas em
  que o septo/complexo é liberado e reposicionado para baixo, com maior
  dissecção.
\item
  \textbf{Complemento com enxertia de gordura:} quando o volume
  pediculado é insuficiente para o defeito (ou quando há necessidade de
  refinamento fino), considerar micro/nanofat como complemento
  planejado.
\end{itemize}

\begin{quote}
\textbf{PÉROLA CLÍNICA}
\end{quote}

\begin{quote}
\mbox{}%
\subsection{Zona de Risco: Irregularidades, nódulos e necrose
gordurosa}\label{zona-de-risco-irregularidades-nuxf3dulos-e-necrose-gordurosa}
\end{quote}

\begin{quote}
\begin{itemize}
\tightlist
\item
  \textbf{Mecanismo:} trauma mecânico (esmagamento),
  torção/estrangulamento do pedículo, leito receptor pequeno/compressivo
  ou hemostasia inadequada → pode gerar nódulos, irregularidade e
  endurecimento prolongado.
\end{itemize}
\end{quote}

\begin{quote}
\begin{itemize}
\tightlist
\item
  \textbf{Prevenção:} manipulação mínima (``no-crush''), leito amplo,
  acomodação sem tensão, hemostasia limpa e planejamento de pele fina
  (possível necessidade de refinamento de superfície).
\end{itemize}
\end{quote}

\begin{quote}
\begin{itemize}
\tightlist
\item
  \textbf{Conduta geral:} a maioria das irregularidades melhora com
  tempo; intervenções (massagem orientada, infiltração, retoque)
  dependem do caso e do protocolo do cirurgião.
\end{itemize}
\end{quote}

\section{Erros comuns (e como
resgatar)}\label{erros-comuns-e-como-resgatar-13}

\begin{itemize}
\item
  \textbf{Erro:} liberação incompleta das estruturas retentoras
  (TTL/ORL) \textbf{Consequência:} volume não assenta, ``efeito mola'',
  persistência do sulco ou abaulamento irregular \textbf{Prevenção:}
  testar mobilidade intraoperatória (o volume deve deslizar/assentar sem
  resistência elástica) e criar leito receptor suficiente Resgate:
  revisão para completar liberação em casos selecionados ou camuflagem
  com volume complementar (ex.: enxertia/preenchedor) quando apropriado
\item
  Erro: transformar a gordura em tração (fixação sob tensão)
  \textbf{Consequência:} distorção, irregularidade e, em cenários
  desfavoráveis, contribuição para retração palpebral
  \textbf{Prevenção:} gordura deve \textbf{repousar} no leito; suporte
  palpebral/terço médio deve ser planejado separadamente quando
  necessário Resgate: medidas conservadoras iniciais; liberação/revisão
  quando persistente e sintomático
\item
  Erro: subestimar pele muito fina e superfície ``denunciadora''
  \textbf{Consequência:} contorno visível, ondulações e ``edge'' de
  transição \textbf{Prevenção:} planejamento de refinamento (ex.:
  nanofat/skin quality adjunct) e volume bem distribuído
  \textbf{Resgate:} correções de superfície em segundo tempo conforme
  evolução cicatricial
\end{itemize}

\section{Notas de ``arte'' (a Curva em
S)}\label{notas-de-arte-a-curva-em-s}

O objetivo estético não é ``apagar bolsas'', mas recriar uma superfície
única e contínua da pálpebra inferior ao malar. O envelhecimento gera
dupla convexidade separada por concavidade; o reposicionamento bem
indicado busca reconstituir a leitura suave do contorno em 3/4
(continuidade da luz e redução de sombras duras).

\section{Pós-operatório}\label{puxf3s-operatuxf3rio-5}

\begin{itemize}
\item
  \textbf{Edema:} tende a ser mais prolongado do que ressecção simples,
  pela dissecção do leito receptor e estase local. Alinhar expectativa.
\item
  \textbf{Endurecimento/irregularidade inicial:} pode ocorrer por
  semanas; conduta depende do protocolo (tempo de observação antes de
  qualquer intervenção).
\item
  \textbf{Bolsters (se usados):} remoção tipicamente em poucos dias,
  conforme técnica.
\end{itemize}

\section{Referências}\label{referuxeancias-12}

\begin{itemize}
\item
  Preservação do arcus marginalis / conceitos clássicos de
  reposicionamento {[}{[}REF:HAMRA-1995{]}{]}.
\item
  Transposição transconjuntival de gordura e variações técnicas
  {[}{[}REF:GOLDBERG-1998{]}{]}.
\item
  Anatomia de TTL/ORL e implicações na junção pálpebra--malar
  {[}{[}REF:MENDELSON-2008{]}{]}. * * *
\end{itemize}

\begin{center}\rule{0.5\linewidth}{0.5pt}\end{center}

\chapter{Capítulo 19 --- Manejo de pele no inferior: skin pinch e
refinamentos sem descolamento
amplo}\label{capuxedtulo-19-manejo-de-pele-no-inferior-skin-pinch-e-refinamentos-sem-descolamento-amplo}

\begin{figure}
\centering
\pandocbounded{\includegraphics[keepaspectratio,alt={Figura 19.1 --- Ilustração principal do capítulo}]{/Users/humbertolopes/Dev/work/marcelo-cury/the_art_of_eyelid_surgery_scaffold/projects/eyelid-surgery/assets/figures/FIG-19-01_skin-pinch-refinamentos.png}}
\caption{Figura 19.1 --- Ilustração principal do capítulo}
\end{figure}

\textbf{Parte:} Parte IV --- Inferior e Terço Médio

\section{Objetivo do capítulo}\label{objetivo-do-capuxedtulo-17}

Ao final, o leitor dominará a técnica de \textbf{skin pinch} para
ressecar \textbf{excesso cutâneo real} da pálpebra inferior \textbf{sem
descolamento amplo} e, quando bem indicada, sem violar o orbicular,
reduzindo o risco de retração palpebral associada a dissecções extensas
e tensão excessiva.

\section{O que muda na decisão (o
``porquê'')}\label{o-que-muda-na-decisuxe3o-o-porquuxea-18}

\begin{itemize}
\item
  \textbf{Pele ≠ sustentação:} tentar ``corrigir frouxidão'' removendo
  pele é erro conceitual. A pele é cobertura; o suporte é
  \textbf{tarso--cantal/ligamentar}. Se há frouxidão horizontal, ela é
  tratada com \textbf{suporte cantal}, não com tração cutânea.
\item
  Preservação de planos e fisiologia: descolamentos amplos aumentam
  edema, equimose, risco de fibrose e podem comprometer a dinâmica do
  orbicular. O \emph{pinch} busca tratar o vetor vertical de excesso
  cutâneo com mínima violação de planos.
\item
  Vertical vs.~horizontal: o \emph{pinch} atua no excesso vertical de
  pele. Frouxidão horizontal (snap-back lento / distraction elevado)
  exige plano concomitante de suporte (cantopexia/cantoplastia) ---
  nunca ``compensar'' cortando mais pele.
\end{itemize}

\section{Indicações e
contra-indicações}\label{indicauxe7uxf5es-e-contra-indicauxe7uxf5es-12}

\textbf{Indicar quando:} - como complemento após abordagem
transconjuntival, quando há pele creponada residual;

\begin{itemize}
\item
  em dermatochalase inferior leve a moderada, com \textbf{bom tônus} e
  sem indicação de retalho miocutâneo;
\item
  quando há excesso cutâneo real que persiste mesmo sob máxima tensão
  vertical (teste boca aberta + olhar para cima).
\end{itemize}

\textbf{Evitar / adiar quando:} - \textbf{festoons/edema malar
predominante:} remover pele não resolve a etiologia e aumenta risco de
cicatriz alargada/retração;

\begin{itemize}
\item
  pele muito espessa/sebácea com alto risco de cicatriz visível e
  benefício estético limitado;
\item
  retração prévia / \emph{scleral show} já presente: remover pele pode
  piorar o quadro sem suporte/estratégia de terço médio.
\end{itemize}

\begin{quote}
\textbf{PÉROLA CLÍNICA}
\end{quote}

\begin{quote}
\mbox{}%
\subsection{\texorpdfstring{Checklist Pré-op (Segurança do
\emph{Pinch})}{Checklist Pré-op (Segurança do Pinch)}}\label{checklist-pruxe9-op-seguranuxe7a-do-pinch}
\end{quote}

\begin{quote}
\begin{itemize}
\tightlist
\item[$\square$]
  \textbf{Snap-back:} retorno imediato vs.~lento (lento → planejar
  suporte cantal).
\end{itemize}
\end{quote}

\begin{quote}
\begin{itemize}
\tightlist
\item[$\square$]
  \textbf{Teste em máxima tensão vertical:} paciente olhando para cima +
  boca aberta (ou manobra equivalente).
\end{itemize}
\end{quote}

\begin{quote}
\begin{itemize}
\tightlist
\item[$\square$]
  \textbf{Vetor:} vetor negativo → conservadorismo adicional
  (sub-correção intencional é preferível).
\end{itemize}
\end{quote}

\begin{quote}
\begin{itemize}
\tightlist
\item[$\square$]
  Expectativa: explicar que o \emph{pinch} atenua rugas de repouso, mas
  não ``apaga'' linhas dinâmicas.
\end{itemize}
\end{quote}

\begin{quote}
\begin{itemize}
\tightlist
\item[$\square$]
  Foto macro: documentar textura (pele fina tipo ``papel'' vs.~espessa)
  e linhas pré-existentes.
\end{itemize}
\end{quote}

\section{Anatomia aplicada (o ``plano
zero'')}\label{anatomia-aplicada-o-plano-zero}

\begin{itemize}
\item
  \textbf{Interface pele--orbicular:} na pálpebra inferior há pouca
  gordura subcutânea; a pele é relativamente aderida ao orbicular. O
  \emph{pinch} explora essa relação \textbf{sem descolar amplamente}.
\item
  \textbf{Orbicular pré-tarsal:} componente crítico da dinâmica
  palpebral. O objetivo é não incorporar músculo no tecido ressecado. Se
  há fibras musculares evidentes no fuso removido, o plano foi profundo
  demais.
\item
  Inervação/vascularização: ramos motores e perfurantes vasculares são
  mais relevantes em dissecções profundas e amplas; o \emph{pinch}
  superficial tende a ser de menor agressão, desde que executado sem
  violação muscular.
\end{itemize}

\begin{quote}
\textbf{📎 FIGURA NECESSÁRIA (Cap. 19):}
\end{quote}

\begin{quote}
Técnica: Skin pinch --- demarcação e excisão
\end{quote}

\begin{quote}
\emph{Estilo: Diagrama técnico-didático, cores neutras, legendas claras}
\end{quote}

\section{Técnica (passo a passo de
precisão)}\label{tuxe9cnica-passo-a-passo-de-precisuxe3o}

\subsection{Visão geral}\label{visuxe3o-geral-9}

\begin{enumerate}
\def\labelenumi{\arabic{enumi}.}
\item
  \textbf{Momento:} após a etapa transconjuntival (se houver) e com
  sedação apenas suficiente para cooperação (quando aplicável).
\item
  \textbf{Tensão vertical:} padronizar a manobra (olhar para cima + boca
  aberta) antes de definir ressecção.
\item
  \textbf{Pinçamento:} apreender somente pele a \textasciitilde1--2 mm
  abaixo da linha ciliar (subciliar), mantendo a margem ciliar neutra.
\item
  Teste de segurança: se houver eversão/tração evidente da margem,
  reduzir a quantidade pinçada (sub-correção é preferível).
\item
  Excisão: ressecar rente à linha do pinçamento, com cuidado para manter
  o plano estritamente cutâneo.
\item
  Hemostasia: meticulosa e pontual (evitar energia excessiva).
\item
  Fechamento: bordas ``boca a boca'' sem tensão; sutura fina (ex.: 6-0)
  contínua ou pontos separados conforme preferência/tensão. Cola pode
  ser opção quando a tensão é virtualmente nula e a hemostasia é
  impecável.
\end{enumerate}

\subsection{Variações e
indicações}\label{variauxe7uxf5es-e-indicauxe7uxf5es-11}

\begin{itemize}
\item
  \textbf{Extensão lateral controlada:} pode seguir ruga natural para
  tratar excesso lateral estático, com vetor neutro a discretamente
  ascendente.
\item
  \textbf{Pinch assimétrico:} geralmente mais ressecção lateral do que
  medial; no terço medial, conservadorismo para reduzir risco de
  cicatriz evidente/\emph{webbing}.
\end{itemize}

\section{Erros comuns (e como
resgatar)}\label{erros-comuns-e-como-resgatar-14}

\begin{itemize}
\item
  \textbf{Erro:} ressecar baseado em repouso (sem teste em tensão
  vertical) \textbf{Consequência:} falta de pele funcional em mímica
  (olhar para cima/boca aberta), arredondamento do olho e risco de
  retração \textbf{Prevenção:} sempre definir quantidade com tensão
  máxima vertical antes de cortar Resgate: conduta conservadora inicial;
  em casos relevantes e persistentes, considerar estratégias
  reconstrutivas conforme gravidade (ex.: enxerto/retalhos), dentro do
  protocolo do cirurgião
\item
  Erro: violar orbicular (pinch profundo) \textbf{Consequência:}
  sangramento desnecessário, maior edema/fibrose e possível
  comprometimento da dinâmica palpebral \textbf{Prevenção:} confirmar
  que o fuso é cutâneo (transiluminação/observação do plano; ``se viu
  músculo, foi longe demais'') \textbf{Resgate:} hemostasia pontual e
  vigilância; evitar ``corrigir'' com mais ressecção
\item
  Erro: estender incisão demais medialmente \textbf{Consequência:}
  cicatriz visível, \emph{webbing} e irregularidade na região próxima ao
  aparato lacrimal \textbf{Prevenção:} respeitar limite medial e
  encerrar alguns milímetros lateral ao ponto lacrimal, conforme
  anatomia do caso \textbf{Resgate:} medidas de manejo cicatricial;
  revisões tardias selecionadas quando necessário
\end{itemize}

\begin{quote}
\textbf{PÉROLA CLÍNICA}
\end{quote}

\begin{quote}
\mbox{}%
\subsection{Zona de Risco: o ``Canine Tooth'' (queda lateral
iatrogênica)}\label{zona-de-risco-o-canine-tooth-queda-lateral-iatroguxeanica}
\end{quote}

\begin{quote}
Evite curvar a incisão lateralmente para baixo (``dente de cachorro'').
Isso cruza linhas de tensão e pode puxar o canto para inferior.
\end{quote}

\begin{quote}
\textbf{Regra de ouro:} término lateral horizontal ou levemente
ascendente, acompanhando pés de galinha / RSTL.
\end{quote}

\section{Notas de ``arte'' (textura e
naturalidade)}\label{notas-de-arte-textura-e-naturalidade}

O objetivo do \emph{pinch} não é criar uma pálpebra ``sem textura''. Em
faces maduras, alguma microtextura e linhas finas são compatíveis com
naturalidade. O melhor resultado costuma ser \textbf{seguro e coerente}
com a pele da bochecha --- e não uma superfície artificialmente lisa que
denuncia intervenção.

\section{Pós-operatório}\label{puxf3s-operatuxf3rio-6}

\begin{itemize}
\item
  \textbf{Milium/cistos de inclusão:} podem ocorrer ao longo da linha de
  fechamento; manejo é ambulatorial conforme prática do serviço.
\item
  \textbf{Equimose linear:} comum pela vascularização dérmica.
\item
  \textbf{Retirada de pontos:} tipicamente 5--7 dias (se sutura), ou
  queda espontânea em \textasciitilde10 dias (se cola), conforme técnica
  e evolução.
\end{itemize}

\section{Referências}\label{referuxeancias-13}

\begin{itemize}
\item
  Técnica de \emph{skin pinch} e segurança em blefaroplastia inferior
  {[}{[}REF:PARK-2008{]}{]} Park / Fagien.
\item
  Inervação do orbicular e implicações das dissecções amplas
  {[}{[}REF:RAMIREZ-2000{]}{]} Ramirez.
\item
  Combinação transconjuntival + \emph{pinch} em rejuvenescimento
  inferior {[}{[}REF:HIDALGO-2011{]}{]} Hidalgo. * * *
\end{itemize}

\begin{center}\rule{0.5\linewidth}{0.5pt}\end{center}

\chapter{Capítulo 20 --- Festoon / edema malar: fisiopatologia e opções
(orbicular, espaço pré-malar, resurfacing,
excisão)}\label{capuxedtulo-20-festoon-edema-malar-fisiopatologia-e-opuxe7uxf5es-orbicular-espauxe7o-pruxe9-malar-resurfacing-excisuxe3o}

\begin{figure}
\centering
\pandocbounded{\includegraphics[keepaspectratio,alt={Figura 20.1 --- Ilustração principal do capítulo}]{/Users/humbertolopes/Dev/work/marcelo-cury/the_art_of_eyelid_surgery_scaffold/projects/eyelid-surgery/assets/figures/FIG-20-01_festoons-edema-malar.png}}
\caption{Figura 20.1 --- Ilustração principal do capítulo}
\end{figure}

\textbf{Parte:} Parte IV --- Inferior e Terço Médio

\section{Objetivo do capítulo}\label{objetivo-do-capuxedtulo-18}

Ao final, o leitor saberá diferenciar \textbf{bolsa palpebral (gordura
orbitária)} de \textbf{malar mound / festoon (edema + frouxidão
cutâneo-muscular sobre o malar)}, entendendo por que a blefaroplastia
``padrão'' frequentemente \textbf{não trata} (e, em predispostos, pode
piorar) o festoon, exigindo estratégia específica (multimodal e, às
vezes, em estágios).

\section{O que muda na decisão (o
``porquê'')}\label{o-que-muda-na-decisuxe3o-o-porquuxea-19}

\begin{itemize}
\item
  \textbf{Erro topográfico:} bolsa palpebral é um problema
  \textbf{orbitário} (acima do rebordo). Festoon/malar mound é um
  problema \textbf{pré-zigomático} (abaixo do rebordo), na unidade
  malar. Tratar ``em cima'' não corrige o que está ``embaixo''.
\item
  Natureza do tecido: bolsa é volume relativamente sólido (gordura).
  Festoon é, em geral, mistura variável de pele redundante + frouxidão
  do orbicular/lamela anterior + estase linfática/fluido. ``Cortar
  pele'' não resolve um espaço que continua acumulando fluido.
\item
  Risco de agravamento por dissecção: descolamentos amplos subciliares e
  manipulação extensiva podem aumentar edema e fibrose, e em pacientes
  com tendência a edema malar podem prolongar ou intensificar a
  tumefação. A decisão deve ser conservadora e orientada por fenótipo,
  não por hábito.
\end{itemize}

\section{Indicações e
contra-indicações}\label{indicauxe7uxf5es-e-contra-indicauxe7uxf5es-13}

\textbf{Indicar abordagem específica quando:} - há \textbf{``dupla
convexidade''} (bolsa palpebral + abaulamento malar abaixo do rebordo);

\begin{itemize}
\item
  o volume é \textbf{flutuante} (piora pela manhã, com sal/álcool,
  alergias, rinossinusite, etc.) sugerindo componente de fluido;
\item
  histórico de preenchedores (especialmente HA) ou procedimentos prévios
  na região malar, com edema persistente.
\end{itemize}

\textbf{Evitar blefaroplastia padrão isolada quando:} - o festoon/malar
mound é a queixa principal;

\begin{itemize}
\tightlist
\item
  o paciente exige ``pele lisa'' e resultado imediato (tratamento é
  frequentemente gradual e combinado).
\end{itemize}

\begin{quote}
\textbf{PÉROLA CLÍNICA}
\end{quote}

\begin{quote}
\mbox{}%
\subsection{Checklist de Diagnóstico Diferencial (bolsa vs.~malar mound
vs.~festoon)}\label{checklist-de-diagnuxf3stico-diferencial-bolsa-vs.-malar-mound-vs.-festoon}
\end{quote}

\begin{quote}
\begin{itemize}
\tightlist
\item[$\square$]
  \textbf{Localização:} acima do rebordo (bolsa orbitária)
  vs.~sobre/abaixo do rebordo zigomático (malar mound/festoon).
\end{itemize}
\end{quote}

\begin{quote}
\begin{itemize}
\tightlist
\item[$\square$]
  \textbf{Consistência:} lobulada/``gordurosa'' vs.~esponjosa/flutuante.
\end{itemize}
\end{quote}

\begin{quote}
\begin{itemize}
\tightlist
\item[$\square$]
  \textbf{Variação diurna:} estável (mais típico de gordura)
  vs.~variável (mais típico de edema).
\end{itemize}
\end{quote}

\begin{quote}
\begin{itemize}
\tightlist
\item[$\square$]
  Teste dinâmico (squinch/sorriso): aumenta com contração do orbicular →
  componente muscular/lamelar; não muda e ``balança'' → maior componente
  de fluido/pele.
\end{itemize}
\end{quote}

\begin{quote}
\begin{itemize}
\tightlist
\item[$\square$]
  Histórico de HA: edema tardio iatrogênico é comum e muda completamente
  o plano (primeiro desfazer o iatrogênico).
\end{itemize}
\end{quote}

\section{Anatomia aplicada (por que acumula
ali?)}\label{anatomia-aplicada-por-que-acumula-ali}

\begin{itemize}
\item
  \textbf{ORL e ligamentos zigomático-cutâneos:} funcionam como pontos
  de \textbf{retenção}. A área imediatamente abaixo pode se comportar
  como ``bolso'' onde fluido/tecido se acumula.
\item
  \textbf{Espaço pré-zigomático / pré-malar:} plano potencial associado
  ao orbicular e tecidos malares; quando há frouxidão/estase, ele vira
  reservatório clínico.
\item
  ``Malar septum'' (conceito): descrições anatômicas sugerem
  barreiras/fixações que compartimentalizam o edema no malar. Na
  prática, pense em compartimentos + retenções que favorecem ``poças''
  localizadas.
\end{itemize}

\section{Técnica (menu de opções --- escolha por
fenótipo)}\label{tuxe9cnica-menu-de-opuxe7uxf5es-escolha-por-fenuxf3tipo}

\textbf{Princípio:} tratar o componente predominante (\textbf{pele /
músculo / fluido / estrutura}) e aceitar que casos moderados a severos
tendem a exigir \textbf{combinação}.

\subsection{Fenótipo A --- Predomínio de fluido/edema (malar mound
``puffy'')}\label{fenuxf3tipo-a-predomuxednio-de-fluidoedema-malar-mound-puffy}

\begin{itemize}
\item
  \textbf{Primeiro:} corrigir causas iatrogênicas e inflamatórias (ex.:
  HA prévio → considerar hialuronidase antes de qualquer cirurgia;
  rinossinusite/alergias → otimizar).
\item
  \textbf{Opções adjuvantes:} protocolos não excisionais (resurfacing
  selecionado, energia fracionada, medidas de pele) com expectativa
  realista.
\end{itemize}

\subsection{Fenótipo B --- Predomínio cutâneo-muscular (festoon
verdadeiro,
``drape'')}\label{fenuxf3tipo-b-predomuxednio-cutuxe2neo-muscular-festoon-verdadeiro-drape}

\begin{itemize}
\item
  \textbf{Excisão direta selecionada:} maior previsibilidade em casos
  severos com pele redundante clara e paciente que aceita cicatriz.
\item
  \textbf{Reposicionamento/suspensão do orbicular / midface:} quando há
  ptose estrutural associada (corrige a fundação, não só a cobertura).
\end{itemize}

\subsection{Fenótipo C --- Mistos (mais
comum)}\label{fenuxf3tipo-c-mistos-mais-comum}

\begin{itemize}
\tightlist
\item
  Combinação por etapas: tratar \textbf{bolsa orbitária} (se houver) de
  forma conservadora + abordar \textbf{malar/festoon} com técnica
  específica (não ``puxar bochecha pela pálpebra'').
\end{itemize}

\section{Opções clássicas (com
indicações)}\label{opuxe7uxf5es-cluxe1ssicas-com-indicauxe7uxf5es}

\begin{enumerate}
\def\labelenumi{\arabic{enumi}.}
\tightlist
\item
  \textbf{Excisão direta (variações supramalares)}
\end{enumerate}

\begin{itemize}
\tightlist
\item
  \textbf{Indicação:} festoons grandes com redundância de pele bem
  definida, especialmente em pacientes com rugas/textura que camuflam.
\item
  \textbf{Trade-off:} cicatriz (qualidade variável); exige técnica de
  fechamento meticulosa e boa seleção.
\end{itemize}

\begin{enumerate}
\def\labelenumi{\arabic{enumi}.}
\setcounter{enumi}{1}
\tightlist
\item
  \textbf{Resurfacing (laser/peeling)}
\end{enumerate}

\begin{itemize}
\tightlist
\item
  Indicação: casos leves a moderados com componente cutâneo/textural
  relevante.
\item
  \textbf{Limite:} não ``seca'' um reservatório linfático sozinho;
  melhora textura e contrai pele, mas não resolve todo fenótipo.
\end{itemize}

\begin{enumerate}
\def\labelenumi{\arabic{enumi}.}
\setcounter{enumi}{2}
\tightlist
\item
  Escleroterapia intralesional (ex.: tetraciclina/doxiciclina --- uso
  não padronizado)
\end{enumerate}

\begin{itemize}
\tightlist
\item
  Indicação: casos selecionados com forte componente de espaço/fluido e
  falha de medidas conservadoras.
\item
  \textbf{Cautelas:} dor, inflamação prolongada, risco de
  irregularidade/fibrose imprevisível; exige consentimento robusto e
  experiência.
\end{itemize}

\begin{enumerate}
\def\labelenumi{\arabic{enumi}.}
\setcounter{enumi}{3}
\tightlist
\item
  \textbf{Midface lift / suspensão malar}
\end{enumerate}

\begin{itemize}
\tightlist
\item
  Indicação: quando há ptose do terço médio e a ``bolsa'' é parte do
  colapso estrutural.
\item
  \textbf{Ponto:} trata a causa mecânica do ``drape'' em muitos mistos.
\end{itemize}

\begin{quote}
\textbf{Nota clínica:} ``micro-lipoaspiração'' superficial na região
malar é uma zona de risco para irregularidades; se considerada, deve ser
extremamente criteriosa e dentro de protocolo do cirurgião.
\end{quote}

\section{Erros comuns (e como
resgatar)}\label{erros-comuns-e-como-resgatar-15}

\begin{itemize}
\item
  \textbf{Erro:} tentar ``esticar festoon'' pelo acesso subciliar
\item
  \textbf{Consequência:} aumento do vetor vertical de tensão → risco de
  retração palpebral / ectrópio cicatricial
\item
  \textbf{Prevenção:} não usar a pálpebra como alavanca para levantar
  bochecha; tratar o malar com técnica própria
\item
  Resgate: liberar tensão vertical e reconstruir suporte conforme
  gravidade (pode exigir enxertos/elevação de terço médio)
\item
  Erro: ignorar HA antigo/procedimentos prévios
\item
  \textbf{Consequência:} edema persistente ``inexplicável'' no pós-op
\item
  \textbf{Prevenção:} anamnese ativa + considerar manejo do iatrogênico
  antes de operar
\item
  \textbf{Resgate:} protocolos seriados de reversão quando aplicável +
  replanejamento (não ``reoperar no escuro'')
\end{itemize}

\section{Notas de ``arte'' (camuflagem
inteligente)}\label{notas-de-arte-camuflagem-inteligente}

Quando o paciente recusa cicatriz direta e o fenótipo não é totalmente
responsivo a resurfacing, a ``arte'' pode ser \textbf{regularizar a
transição} (lid--cheek junction) e reduzir contraste de sombra, às vezes
com \textbf{enxerto de gordura} bem indicado. Isso é
\textbf{camuflagem}, não cura: troca-se alguma definição por
continuidade.

\section{Pós-operatório}\label{puxf3s-operatuxf3rio-7}

\begin{itemize}
\item
  \textbf{Edema prolongado:} comum; preparar expectativa (semanas a
  meses dependendo da agressão e do fenótipo).
\item
  \textbf{Eritema (resurfacing):} pode persistir por semanas.
\item
  \textbf{Drenagem/massagem:} pode ser útil quando indicada e no timing
  correto, mas não é universal; seguir protocolo do serviço.
\end{itemize}

\section{Referências}\label{referuxeancias-14}

\begin{itemize}
\item
  Fisiopatologia e compartimentalização do edema malar / malar mounds
  {[}{[}REF:PESSA-2008{]}{]} Pessa.
\item
  Discussões e séries sobre manejo de festoons (opções excisionais e
  combinadas) {[}{[}REF:PERRY-2013{]}{]} Perry.
\item
  Excisão direta de festoons e seleção de casos
  {[}{[}REF:KPODZO-2014{]}{]} Kpodzo. * * *
\end{itemize}

\begin{center}\rule{0.5\linewidth}{0.5pt}\end{center}

\chapter{Capítulo 21 --- Sustentação: quando cantopexia resolve e quando
não
resolve}\label{capuxedtulo-21-sustentauxe7uxe3o-quando-cantopexia-resolve-e-quando-nuxe3o-resolve}

\begin{figure}
\centering
\pandocbounded{\includegraphics[keepaspectratio,alt={Figura 21.1 --- Ilustração principal do capítulo}]{/Users/humbertolopes/Dev/work/marcelo-cury/the_art_of_eyelid_surgery_scaffold/projects/eyelid-surgery/assets/figures/FIG-21-01_sustentacao-cantopexia-decisao.png}}
\caption{Figura 21.1 --- Ilustração principal do capítulo}
\end{figure}

\textbf{Parte:} Parte V --- Canto Lateral e Sustentação

\textbf{Objetivo do capítulo:} Ao final, o leitor saberá
\textbf{estratificar a frouxidão cantal} (leve / moderada / severa) e
escolher entre \textbf{reforço (cantopexia)} vs reconstrução
(cantoplastia / tarsal strip), reduzindo recidiva de ectrópio e evitando
distorção do canto lateral.

\section{O que muda na decisão (o
``porquê'')}\label{o-que-muda-na-decisuxe3o-o-porquuxea-20}

\begin{itemize}
\item
  \textbf{A falácia da ``sutura salvadora'':} Cantopexia funciona quando
  existe \textbf{tecido nativo competente} (tarso + tendão/cápsula
  cantal com resistência). Em tendões alongados/degenerados, ``apertar''
  sem encurtar/reconstruir é, na prática, \textbf{manutenção temporária}
  --- pode ceder com edema, cicatrização e tração vertical do pós-op.
\item
  A fixação errada cria o problema que você queria evitar: O erro
  técnico clássico é ancorar ``na frente'' do rebordo orbitário lateral.
  Isso cria gapping (afastamento pálpebra--globo), piora
  lagoftalmo/epífora e pode gerar \emph{round eye}. A sustentação
  funcional exige um vetor póstero-superior, com ancoragem no periósteo
  firme do aspecto interno do rebordo (região do tubérculo orbital
  lateral).
\item
  Estética vs.~segurança é um falso dilema (quando você escolhe certo):
\item
  Cantopexia bem indicada preserva o ``V'' cantal e melhora inclinação.
\item
  Cantoplastia bem executada não precisa arredondar o canto; o
  \emph{rounding} é consequência de indicação errada, tensão cutânea ou
  reconstrução com geometria ruim.
\end{itemize}

A decisão real é: grau de frouxidão + risco de tração vertical (pele,
midface, vetor negativo) + superfície ocular.

\section{Indicações e
contra-indicações}\label{indicauxe7uxf5es-e-contra-indicauxe7uxf5es-14}

\textbf{Indicar Cantopexia (Reforço):} * Profilaxia em blefaroplastia
inferior (transconjuntival ou transcutânea) quando há \textbf{laxidade
leve}.

\begin{itemize}
\item
  \emph{Distraction test} \textbf{\textless{} 6 mm} e \emph{snap-back}
  imediato ou discretamente retardado, sem ``flutuação''.
\item
  Ausência de ectrópio/retração prévia.
\item
  Boa saúde de superfície ocular (sem olho seco importante / sem
  exposição).
\end{itemize}

\textbf{Indicar Cantoplastia / Lateral Tarsal Strip (Reconstrução):} *
\emph{Distraction test} \textbf{≥ 6--8 mm}, \emph{snap-back}
\textbf{lento/ausente} (precisa piscar para voltar) ou margem que
``boia''.

\begin{itemize}
\item
  Ectrópio/retração prévia, \emph{round eye}, cirurgia prévia falha.
\item
  Vetor negativo + midface ptótico (alta demanda de suporte): o sistema
  precisa ser estrutural, não só ``amarrado''.
\item
  Sintomas funcionais: epífora por malaposição, exposição, sensação de
  areia (após excluir causas de olho seco).
\end{itemize}

\begin{quote}
\textbf{Regra de ouro:} quando a cirurgia no inferior vai gerar
\textbf{tração vertical} (pinch, retalho cutâneo, cicatriz, midface
pesado), a barra de indicação sobe: você tende a precisar de
\textbf{reconstrução}, não ``reforço''.
\end{quote}

\begin{quote}
\textbf{PÉROLA CLÍNICA}
\end{quote}

\begin{quote}
\mbox{}%
\subsection{Checklist de Decisão (3 testes + 1 olhar
clínico)}\label{checklist-de-decisuxe3o-3-testes-1-olhar-cluxednico}
\end{quote}

\begin{quote}
\begin{itemize}
\tightlist
\item[$\square$]
  \textbf{Distraction test:} \textgreater{} 6 mm sugere frouxidão
  relevante.
\end{itemize}
\end{quote}

\begin{quote}
\begin{itemize}
\tightlist
\item[$\square$]
  \textbf{Snap-back:} retorno sem piscar (ok) vs retorno com
  piscar/atraso (alto risco).
\end{itemize}
\end{quote}

\begin{quote}
\begin{itemize}
\tightlist
\item[$\square$]
  \textbf{Simulação com pinça:} elevar o canto lateral com vetor
  póstero-superior:
\end{itemize}
\end{quote}

\begin{quote}
\begin{itemize}
\tightlist
\item
  se a margem ``assenta'' no globo e o V preserva → pexia pode bastar
  (se testes leves);
\end{itemize}
\end{quote}

\begin{quote}
\begin{itemize}
\tightlist
\item
  se a margem continua ``solta''/arredonda → precisa strip.
\end{itemize}
\end{quote}

\begin{quote}
\begin{itemize}
\tightlist
\item[$\square$]
  Inclinação cantal: canto lateral deve ficar \textasciitilde1--2 mm
  acima do medial (na maioria dos fenótipos). Canto ``triste'' exige
  reposição superior (não apenas encurtar).
\end{itemize}
\end{quote}

\section{Anatomia aplicada (onde ancorar de
verdade)}\label{anatomia-aplicada-onde-ancorar-de-verdade}

\begin{itemize}
\item
  \textbf{Complexo cantal lateral:} não é só ``um tendão''; é um
  conjunto (tarso, cápsula, ramos cantais) que envelhece e perde
  rigidez.
\item
  \textbf{Tubérculo orbital lateral (Whitnall):} referência óssea de
  periósteo firme na parede lateral, na região da sutura
  frontozigomática. O objetivo é \textbf{ancorar no aspecto interno do
  rebordo}, evitando fixação anterior que ``descola'' a pálpebra do
  globo.
\item
  Tarso inferior: é a ``alça'' estrutural. Se você puxa pele/orbicular
  superficial, você deforma; se você puxa tarso, você sustenta.
\item
  Lockwood / retratores inferiores: contribuem para posição, mas a
  sustentação anti-ectrópio é dominada pelo tarso ancorado ao osso.
\end{itemize}

\section{Técnica (Lógica de
Execução)}\label{tuxe9cnica-luxf3gica-de-execuuxe7uxe3o}

\subsection{1) Cantopexia (reforço /
reposicionamento)}\label{cantopexia-reforuxe7o-reposicionamento}

\textbf{Objetivo:} reforçar e/ou reposicionar o canto sem ``desmontar''
o canto.

\begin{itemize}
\item
  \textbf{Acesso:} pela própria incisão lateral (blefaroplastia) ou
  mini-incisão.
\item
  \textbf{Ponto de compra:} tarso lateral (porção firme) / cápsula
  cantal profunda --- não pele.
\item
  Ancoragem: periósteo firme do aspecto interno do rebordo lateral
  (vetor póstero-superior).
\item
  Fio: não absorvível (Prolene/Nylon) ou PDS de longa duração, conforme
  preferência.
\item
  Check intraoperatório: boa aposição globo--margem + preservação do V +
  inclinação leve positiva.
\end{itemize}

\textbf{Falha típica:} usar pexia para ``resolver'' frouxidão severa → o
tecido cede, a sutura ``corta'' cápsula, ou a pálpebra volta a descolar.

\subsection{2) Cantoplastia (Lateral Tarsal Strip) --- Objetivo:
encurtar e reconstruir o ``cinto'' tarsal, criando um novo ponto de
fixação
estrutural.}\label{cantoplastia-lateral-tarsal-strip-objetivo-encurtar-e-reconstruir-o-cinto-tarsal-criando-um-novo-ponto-de-fixauxe7uxe3o-estrutural.}

\begin{itemize}
\item
  \textbf{Passos-chave (macro):} cantotomia lateral + cantólise inferior
  → criação do strip tarsal (desepitelização) → ancoragem profunda
  (póstero-superior) → reconstrução do ângulo cutâneo preservando
  geometria.
\item
  Check intraoperatório: o canto deve ficar \textbf{agudo}, com boa
  aposição ao globo e sem tensão cutânea vertical.
\end{itemize}

\subsection{Onde a técnica falha}\label{onde-a-tuxe9cnica-falha}

\begin{itemize}
\item
  Pexia em tendão degenerado (erro de indicação).
\item
  Fixação anterior (erro de vetor).
\item
  Puxar pele para ``fazer canto'' (erro de plano).
\item
  Hipercorreção (canto alto demais) → deformidade estética e
  desconforto.
\end{itemize}

\section{Erros comuns (e como
resgatar)}\label{erros-comuns-e-como-resgatar-16}

\begin{itemize}
\item
  \textbf{Fixação muito anterior}
\item
  \emph{Consequência:} gapping, epífora, lagoftalmo, olho ``seco'' por
  exposição.
\item
  \emph{Prevenção:} ancoragem no aspecto \textbf{interno} do rebordo,
  vetor póstero-superior.
\item
  \emph{Resgate:} reabordagem para reposicionar a sutura (quanto mais
  cedo, melhor, antes de fibrosar).
\item
  \textbf{``Round eye'' por tensão errada}
\item
  \emph{Consequência:} canto arredonda, esclera aparece
  lateral/inferior.
\item
  \emph{Prevenção:} tracionar tarso, não pele; reconstruir ângulo com
  geometria correta; controlar vetor vertical.
\item
  \emph{Resgate:} cantoplastia formal + liberação de cicatriz/retalhos
  conforme necessidade.
\item
  \textbf{Subcorreção em frouxidão severa}
\item
  \emph{Consequência:} recidiva de ectrópio/retração em semanas/meses.
\item
  \emph{Prevenção:} reconhecer indicação de strip (distraction alto +
  snap-back ruim).
\item
  \emph{Resgate:} tarsal strip secundário (muitas vezes o tratamento
  definitivo).
\end{itemize}

\section{Notas de ``arte'' (ângulo de beleza e
``naturalidade'')}\label{notas-de-arte-uxe2ngulo-de-beleza-e-naturalidade}

A meta estética não é ``olho puxado''; é \textbf{inclinação discreta} e
\textbf{apoio firme}.

\begin{itemize}
\item
  Canto lateral \textasciitilde1--2 mm acima do medial costuma parecer
  mais desperto.
\item
  Canto alto demais cria sinal cirúrgico (``Spock'').
\item
  Canto horizontal em face envelhecida pode parecer cansado.
\end{itemize}

A arte está em \textbf{poucos milímetros} com vetor correto.

\section{Pós-operatório}\label{puxf3s-operatuxf3rio-8}

\begin{itemize}
\item
  Sensação de aperto e assimetria inicial são comuns por edema; orientar
  hipercorreção leve como normal nas primeiras 1--2 semanas.
\item
  Lubrificação (colírio/gel) quando houver exposição transitória.
\item
  Quemose pode ocorrer; manejo conservador + proteção de superfície
  ocular.
\end{itemize}

\section{Referências}\label{referuxeancias-15}

\begin{itemize}
\item
  Distinção biomecânica entre pexia e plastia; algoritmos de indicação
  {[}{[}REF:FAGIEN-1999{]}{]} Fagien.
\item
  Vetores de fixação cantal e anatomia do rebordo lateral
  {[}{[}REF:JELKS-1993{]}{]} Jelks.
\item
  Complicações por indicação incorreta e ancoragem anterior
  {[}{[}REF:MCCORD-1995{]}{]} McCord. * * *
\end{itemize}

\begin{center}\rule{0.5\linewidth}{0.5pt}\end{center}

\chapter{Capítulo 22 --- Cantopexia vs.~Cantoplastia: indicações por
vetor e
flacidez}\label{capuxedtulo-22-cantopexia-vs.-cantoplastia-indicauxe7uxf5es-por-vetor-e-flacidez}

\begin{figure}
\centering
\pandocbounded{\includegraphics[keepaspectratio,alt={Figura 22.1 --- Ilustração principal do capítulo}]{/Users/humbertolopes/Dev/work/marcelo-cury/the_art_of_eyelid_surgery_scaffold/projects/eyelid-surgery/assets/figures/FIG-22-01_cantopexia-vs-cantoplastia.png}}
\caption{Figura 22.1 --- Ilustração principal do capítulo}
\end{figure}

\textbf{Parte:} Parte V --- Canto Lateral e Sustentação

\textbf{Objetivo do capítulo:} Ao final, o leitor saberá abandonar a
``cantopexia para todos'', usando \textbf{critérios objetivos}
(frouxidão, \emph{snap-back}, posição cantal e vetor orbital) para
indicar \textbf{reforço (pexia)} quando há tecido competente --- e
reconstrução formal (cantoplastia / Lateral Tarsal Strip) quando a pexia
tende à recidiva.

\section{O que muda na decisão (o
``porquê'')}\label{o-que-muda-na-decisuxe3o-o-porquuxea-21}

\begin{itemize}
\item
  \textbf{A biomecânica da falha (o ``elástico velho''):} A cantopexia
  apenas \textbf{re-tensiona} o sistema nativo. Se o complexo cantal
  está alongado/degenerado (idoso, pós-cirurgia, ectrópio prévio), a
  pexia pode oferecer melhora inicial, mas é mais suscetível a
  \textbf{ceder} sob edema, cicatrização e tração vertical do
  pós-operatório. A cantoplastia remove/encurta o componente ``doente''
  e recria uma ancoragem estrutural.
\item
  O imperativo do vetor negativo: No vetor negativo (globo relativamente
  anterior ao malar), a pálpebra inferior opera ``contra a gravidade'' e
  contra a falta de prateleira óssea. Aqui, o sistema exige um vetor de
  fixação póstero-superior e, com frequência, uma solução mais
  estrutural (Strip) --- especialmente se houver frouxidão relevante.
\item
  Distorção nasce de indicação errada + plano errado: Tentar resolver
  grande flacidez com pexia leva a plicaturas excessivas e pode gerar
  fish-mouthing, enterramento do canto ou \emph{rounding}. A
  cantoplastia permite ajustar tensão e geometria do ``V'' com mais
  controle, desde que a reconstrução do ângulo seja meticulosa.
\end{itemize}

\section{Indicações e
contra-indicações}\label{indicauxe7uxf5es-e-contra-indicauxe7uxf5es-15}

\textbf{Indicar Cantopexia (Reforço):} * \textbf{Vetor positivo/neutro}
+ suporte malar aceitável.

\begin{itemize}
\item
  Frouxidão \textbf{leve}: \emph{Distraction test} \textless{} 6 mm e
  \emph{snap-back} bom (retorna sem ``flutuar'').
\item
  Profilaxia em blefaroplastia inferior quando haverá tração cicatricial
  (ex.: subciliar), desde que o tendão não seja francamente
  incompetente.
\item
  Paciente sem ectrópio/retração prévia significativa.
\end{itemize}

\textbf{Indicar Cantoplastia (Lateral Tarsal Strip):} * \textbf{Vetor
negativo} + qualquer frouxidão moderada/severa (risco alto).

\begin{itemize}
\item
  Frouxidão \textbf{moderada/severa}: \emph{Distraction} ≥ 6--8 mm ou
  \emph{snap-back} lento/ausente.
\item
  Ectrópio, retração, ``olho redondo'' (sequela), falha de pexia prévia.
\item
  Canto lateral baixo/distópico (exige liberação + reposicionamento
  controlado).
\end{itemize}

\textbf{Cautela (não é ``não fazer'', é planejar):} * Olho seco
relevante / exposição (o ``aperto'' pode piorar sintomas) → ajustar
objetivo funcional, lubrificação agressiva e evitar hipercorreção.

\begin{itemize}
\tightlist
\item
  Tarso muito flácido (\emph{floppy eyelid}) → pode exigir reforços
  adicionais (técnicas complementares, enxertos/espaciadores em casos
  selecionados).
\end{itemize}

\begin{quote}
\textbf{PÉROLA CLÍNICA}
\end{quote}

\begin{quote}
\mbox{}%
\subsection{Checklist de Decisão: Pexy vs.~Strip (objetivo e
rápido)}\label{checklist-de-decisuxe3o-pexy-vs.-strip-objetivo-e-ruxe1pido}
\end{quote}

\begin{quote}
\begin{itemize}
\tightlist
\item[$\square$]
  \textbf{Distraction Test:} \textless{} 6 mm (Pexy provável) vs.~≥ 6--8
  mm (Strip provável).
\end{itemize}
\end{quote}

\begin{quote}
\begin{itemize}
\tightlist
\item[$\square$]
  \textbf{Snap-back:} imediato (Pexy) vs.~lento/incompleto (Strip).
\end{itemize}
\end{quote}

\begin{quote}
\begin{itemize}
\tightlist
\item[$\square$]
  \textbf{Vetor orbital:} positivo/neutro (Pexy aceitável) vs.~negativo
  (limiar mais baixo para Strip).
\end{itemize}
\end{quote}

\begin{quote}
\begin{itemize}
\tightlist
\item[$\square$]
  Posição do canto: normal vs.~baixo/distópico (Strip com
  reposicionamento).
\end{itemize}
\end{quote}

\begin{quote}
\begin{itemize}
\tightlist
\item[$\square$]
  Histórico: ectrópio/retração/pexia falha = Strip quase sempre.
\end{itemize}
\end{quote}

\begin{quote}
\begin{itemize}
\tightlist
\item[$\square$]
  Superfície ocular: olho seco/exposição → evitar hipercorreção; meta
  funcional primeiro.
\end{itemize}
\end{quote}

\section{Anatomia aplicada (o ``corte
estratégico'')}\label{anatomia-aplicada-o-corte-estratuxe9gico}

\begin{itemize}
\item
  \textbf{Crus do tendão cantal lateral:} o LCT bifurca-se em crus
  superior e inferior. A cantoplastia clássica utiliza cantólise
  inferior (secção seletiva do crus inferior) para liberar a pálpebra
  inferior e permitir encurtamento/realinhamento.
\item
  \textbf{Linha cinzenta:} guia anatômico para separar lamela anterior
  (pele/orbicular) e posterior (tarso/conjuntiva) com precisão e menor
  distorção.
\item
  \textbf{Periósteo do aspecto interno do rebordo lateral:} ancoragem
  profunda (póstero-superior) mantém a pálpebra aposta ao globo. Fixação
  muito anterior favorece \emph{gapping} e epífora.
\end{itemize}

\section{Técnica: Lateral Tarsal Strip (padrão
ouro)}\label{tuxe9cnica-lateral-tarsal-strip-padruxe3o-ouro}

\subsection{Visão geral}\label{visuxe3o-geral-10}

\begin{enumerate}
\def\labelenumi{\arabic{enumi}.}
\item
  \textbf{Cantotomia lateral:} incisão horizontal controlada (≈ 8--12 mm
  conforme necessidade) para expor o canto e permitir manobra sem
  tensão.
\item
  \textbf{Cantólise inferior (passo crítico):} identificar a ``corda''
  do crus inferior e seccioná-la até que a pálpebra inferior esteja
  \textbf{livre e móvel} (se não mobiliza, a técnica falha por
  definição).
\item
  Preparo do Strip:
\end{enumerate}

\begin{itemize}
\tightlist
\item
  Separar lamela anterior da posterior nos milímetros laterais.
\item
  Desepitelizar (raspar) a margem e remover conjuntiva do segmento
  tarsal --- se sobrar epitélio, aumenta risco de cisto de inclusão.
\end{itemize}

\begin{enumerate}
\def\labelenumi{\arabic{enumi}.}
\setcounter{enumi}{3}
\item
  Encurtamento medido: tracionar o strip no vetor póstero-superior e
  definir quanto tarso deve ser ressecado para obter tensão firme sem
  deformar.
\item
  Fixação profunda: sutura do strip ao periósteo profundo/interno do
  rebordo lateral, no vetor póstero-superior, buscando:
\end{enumerate}

\begin{itemize}
\tightlist
\item
  boa aposição ao globo,
\item
  canto agudo preservado,
\item
  inclinação final geralmente 1--2 mm acima do canto medial (respeitando
  fenótipo).
\end{itemize}

\begin{enumerate}
\def\labelenumi{\arabic{enumi}.}
\setcounter{enumi}{5}
\item
  Reconstrução do ângulo (``V''): reaproximação precisa da linha
  cinzenta e alinhamento cutâneo para evitar \emph{webbing} e
  arredondamento.
\item
  Check intraoperatório: olhar primário + fechamento suave; confirmar
  que não houve gapping e que a tensão não ``puxa'' a pele
  verticalmente.
\end{enumerate}

\subsection{Variações}\label{variauxe7uxf5es}

\begin{itemize}
\tightlist
\item
  \textbf{Cantopexia (Olivari / Fagien):} útil como reforço quando
  frouxidão é leve e o tecido é competente. Em laxidade maior, aumenta
  risco de \emph{buckling} e recidiva.
\end{itemize}

\section{Erros comuns (e como
resgatar)}\label{erros-comuns-e-como-resgatar-17}

\begin{itemize}
\item
  \textbf{Cantólise incompleta}
\item
  \emph{Consequência:} você fixa sob tensão, o ponto rasga ou o canto
  deforma.
\item
  \emph{Prevenção:} só avance para fixação quando a pálpebra estiver
  claramente ``solta''.
\item
  \emph{Resgate:} reabrir e completar a liberação (quanto antes,
  melhor).
\item
  \textbf{Webbing (membrana) no canto}
\item
  \emph{Consequência:} teia cicatricial cobrindo o ângulo.
\item
  \emph{Prevenção:} alinhamento cinza-cinza preciso e evitar avanço
  indevido de pele inferior sobre superior.
\item
  \emph{Resgate:} Z-plastia tardia / revisão de canto.
\item
  \textbf{Fixação anterior (gapping)}
\item
  \emph{Consequência:} afastamento pálpebra--globo, epífora, exposição.
\item
  \emph{Prevenção:} ancoragem no aspecto interno do rebordo + vetor
  póstero-superior.
\item
  \emph{Resgate:} reoperação para reposicionamento profundo da sutura.
\item
  \textbf{Hipercorreção estética}
\item
  \emph{Consequência:} ``olho de gato'' artificial, desconforto.
\item
  \emph{Prevenção:} respeitar o fenótipo; meta é inclinação discreta,
  não ``tension lift''.
\item
  \emph{Resgate:} afrouxar/revisar fixação (idealmente precoce).
\end{itemize}

\section{Notas de ``arte'' (geometria do
olhar)}\label{notas-de-arte-geometria-do-olhar}

Cantoplastia muda o olho \textbf{se você deixar}. O objetivo é preservar
amendoado e naturalidade:

\begin{itemize}
\item
  Fixar baixo → ``olho triste'' / rounding.
\item
  Fixar alto demais → sinal cirúrgico (``Spock/cat eye'').
\item
  \textbf{Alvo funcional-estético:} canto lateral firme, agudo, com
  inclinação discreta compatível com o paciente.
\end{itemize}

\section{Pós-operatório}\label{puxf3s-operatuxf3rio-9}

\begin{itemize}
\item
  Assimetria inicial por edema é esperada; explicar janela realista
  (semanas).
\item
  Pontos cutâneos: remoção em \textasciitilde7 dias (conforme sua
  rotina).
\item
  Lubrificação e vigilância de exposição/quemose quando houve
  manipulação cantal significativa.
\end{itemize}

\section{Referências}\label{referuxeancias-16}

\begin{itemize}
\item
  Técnica clássica do Lateral Tarsal Strip {[}{[}REF:ANDERSON-1979{]}{]}
  Anderson \& Gordy.
\item
  Vetor orbital e seleção de técnica {[}{[}REF:JELKS-1993{]}{]} Jelks.
\item
  Cantopexias (simples vs.~complexas) e critérios de indicação
  {[}{[}REF:FAGIEN-1999{]}{]} Fagien. * * *
\end{itemize}

\begin{center}\rule{0.5\linewidth}{0.5pt}\end{center}

\chapter{Capítulo 23 --- Técnicas de canto lateral: Mladick, Tarsal
Strip e McCord (e
variações)}\label{capuxedtulo-23-tuxe9cnicas-de-canto-lateral-mladick-tarsal-strip-e-mccord-e-variauxe7uxf5es}

\begin{figure}
\centering
\pandocbounded{\includegraphics[keepaspectratio,alt={Figura 23.1 --- Ilustração principal do capítulo}]{/Users/humbertolopes/Dev/work/marcelo-cury/the_art_of_eyelid_surgery_scaffold/projects/eyelid-surgery/assets/figures/FIG-23-01_mladick-strip-mccord.png}}
\caption{Figura 23.1 --- Ilustração principal do capítulo}
\end{figure}

\textbf{Parte:} Parte V --- Canto Lateral e Sustentação

\textbf{Objetivo do capítulo:} Ao final, o leitor saberá navegar o
``degrau de complexidade'' das fixações laterais, escolhendo
\textbf{suspensão de lamela anterior (Mladick)}, \textbf{reconstrução de
lamela posterior (Tarsal Strip/Anderson)} ou fixações robustas de
revisão (McCord e variações, incluindo ancoragem óssea e \emph{spacers})
conforme frouxidão, vetor orbital e risco de retração/recidiva.

\section{O que muda na decisão (o
``porquê'')}\label{o-que-muda-na-decisuxe3o-o-porquuxea-22}

\begin{itemize}
\item
  \textbf{Hierarquia do problema (lamela anterior vs.~posterior):}
\item
  \textbf{Mladick} atua primariamente na \textbf{lamela anterior}
  (músculo/pele) e é útil como \emph{suporte} / profilaxia.
\item
  Tarsal Strip atua na lamela posterior (tarso/tendão), corrigindo
  frouxidão real e ectrópio/retração por falha estrutural.
\item
  McCord e revisões entram quando há retração significativa, periósteo
  ``cansado'' por cirurgias prévias, necessidade de vetor forte e,
  frequentemente, alongamento vertical com \emph{spacer}.
\item
  \textbf{Integridade do ângulo cantal:}
\item
  Técnicas de ``pexy'' preservam mais o ``V'' nativo.
\item
  Técnicas de ``plasty'' (Strip e reconstruções) \textbf{desmontam e
  reconstroem} o canto. Isso é uma ferramenta --- não um castigo ---
  quando o tendão não é confiável.
\item
  \textbf{O fator recidiva não é azar:}
\item
  Recidiva costuma vir de \textbf{indicação subestimada} (pexia em
  tendão senil), ancoragem anterior (gapping) ou \textbf{ausência de
  \emph{release}} (tensão vertical puxando a pálpebra para baixo no
  pós-op).
\end{itemize}

\begin{quote}
\textbf{PÉROLA CLÍNICA}
\end{quote}

\begin{quote}
\mbox{}%
\subsection{Checklist Pré-op (Seleção de
Técnica)}\label{checklist-pruxe9-op-seleuxe7uxe3o-de-tuxe9cnica}
\end{quote}

\begin{quote}
\begin{itemize}
\tightlist
\item[$\square$]
  \textbf{Distraction Test:} leve (\textless{} \textasciitilde6mm) →
  considerar Mladick/pexy; moderado-severo (≥ \textasciitilde6--8mm) →
  Strip.
\end{itemize}
\end{quote}

\begin{quote}
\begin{itemize}
\tightlist
\item[$\square$]
  \textbf{Snap-back:} imediato vs.~lento/incompleto (quanto pior, mais
  você migra para Strip/reconstrução).
\end{itemize}
\end{quote}

\begin{quote}
\begin{itemize}
\tightlist
\item[$\square$]
  \textbf{Vetor orbital:} negativo reduz seu ``limiar'' para técnicas
  estruturais.
\end{itemize}
\end{quote}

\begin{quote}
\begin{itemize}
\tightlist
\item[$\square$]
  Comprimento da fenda horizontal: fenda curta/olho pequeno → evitar
  encurtamentos agressivos; preferir correções com \emph{release},
  midface e tensão bem dosada.
\end{itemize}
\end{quote}

\begin{quote}
\begin{itemize}
\tightlist
\item[$\square$]
  Qualidade do tarso: rígido (ótimo para Strip) vs.~``floppy'' (planejar
  reforços/variações).
\end{itemize}
\end{quote}

\begin{quote}
\begin{itemize}
\tightlist
\item[$\square$]
  Canto distópico/baixo: exige liberação completa (cantólise) +
  reposicionamento superior controlado.
\end{itemize}
\end{quote}

\begin{quote}
\begin{itemize}
\tightlist
\item[$\square$]
  Histórico de cirurgias / periósteo comprometido: considerar variações
  tipo McCord (ancoragem mais robusta / \emph{drill hole}).
\end{itemize}
\end{quote}

\section{Anatomia aplicada (onde o ponto
``morde'')}\label{anatomia-aplicada-onde-o-ponto-morde}

\begin{itemize}
\item
  \textbf{Cinta do orbicular (Mladick):} cria um ``tirante'' dinâmico.
  Funciona melhor quando há \textbf{tônus muscular preservado} e o
  objetivo é \textbf{suporte leve} ou prevenção.
\item
  Tarso lateral (Strip): colágeno denso = melhor ``material de
  construção''. Ao denudar epitélio/conjuntiva, você transforma tarso em
  um neo-tendão controlável.
\item
  Periósteo interno do rebordo lateral (alvo funcional): fixação
  profunda mantém aposição ao globo. Periósteo frágil (revisões) pode
  falhar por ``cheese-wiring'' (fio rasgando).
\item
  Tubérculo de Whitnall / face interna do rebordo: referência anatômica
  útil para evitar fixação anterior demais.
\end{itemize}

\section{Técnica (passo a passo diferenciado por
``degrau'')}\label{tuxe9cnica-passo-a-passo-diferenciado-por-degrau}

\subsection{Degrau 1 --- Mladick (Modified Muscle Suspension / ``sling''
do
orbicular)}\label{degrau-1-mladick-modified-muscle-suspension-sling-do-orbicular}

\textbf{Indicação típica:} profilaxia; leve necessidade de
elevação/suporte lateral com tarso competente.

\begin{enumerate}
\def\labelenumi{\arabic{enumi}.}
\item
  \textbf{Acesso:} pela via inferior (transcutânea curta) ou pela via
  superior com extensão lateral (conforme planejamento).
\item
  \textbf{Isolamento:} identificar uma alça/segmento do orbicular
  lateral (preferencialmente com boa espessura e sem desvascularizar
  excessivamente).
\item
  \textbf{Sutura de suspensão:} fio não absorvível fino ou absorvível de
  longa duração, capturando a alça muscular e ancorando em periósteo
  póstero-superior.
\item
  Check: melhora discreta de inclinação e sustentação; não deve
  ``enterrar'' canto nem tensionar pele verticalmente.
\end{enumerate}

\textbf{Limite conceitual:} Mladick \textbf{não substitui} reconstrução
tarsal quando há frouxidão real. Ele ajuda, mas não ``vira osso''.

\begin{center}\rule{0.5\linewidth}{0.5pt}\end{center}

\subsection{Degrau 2 --- Tarsal Strip (Anderson) --- o ``cavalo de
batalha''}\label{degrau-2-tarsal-strip-anderson-o-cavalo-de-batalha}

Indicação típica: \emph{distraction} elevado, \emph{snap-back} ruim,
ectrópio/retração leve a moderada, vetor negativo, sequelas de
blefaroplastia.

\begin{enumerate}
\def\labelenumi{\arabic{enumi}.}
\item
  \textbf{Cantotomia lateral:} incisão controlada para exposição.
\item
  \textbf{Cantólise inferior completa:} liberar crus inferior até a
  pálpebra inferior ficar realmente móvel.
\item
  \textbf{Confecção do strip:}
\end{enumerate}

\begin{itemize}
\tightlist
\item
  separar lamela anterior da posterior nos mm laterais,
\item
  desepitelizar margem ciliar e remover conjuntiva do segmento que será
  enterrado.
\end{itemize}

\begin{enumerate}
\def\labelenumi{\arabic{enumi}.}
\setcounter{enumi}{3}
\item
  Encurtamento medido: ressecar apenas o necessário para obter tensão
  firme sem fimose.
\item
  Fixação profunda: sutura colchoeira do strip ao periósteo
  interno/lateral com vetor póstero-superior.
\item
  Reconstrução do ângulo: alinhamento preciso ``cinza-cinza'' para
  manter o ``V'' agudo.
\item
  Check: aposição ao globo, inclinação final compatível com o paciente e
  fechamento confortável.
\end{enumerate}

\begin{center}\rule{0.5\linewidth}{0.5pt}\end{center}

\subsection{\texorpdfstring{Degrau 3 --- McCord e variações (Revisão
robusta: \emph{release} + ancoragem forte ±
\emph{spacer})}{Degrau 3 --- McCord e variações (Revisão robusta: release + ancoragem forte ± spacer)}}\label{degrau-3-mccord-e-variauxe7uxf5es-revisuxe3o-robusta-release-ancoragem-forte-spacer}

Indicação típica: retração importante, cicatriz vertical, periósteo
ruim, múltiplas cirurgias, necessidade de grande mudança de vetor.

O princípio aqui é \textbf{não depender só de encurtamento horizontal}.

\begin{enumerate}
\def\labelenumi{\arabic{enumi}.}
\item
  \textbf{Liberação ampla (\emph{release}):} soltar aderências
  cicatriciais e reduzir tensão vertical (frequentemente incluindo
  manejo de retratores e espaço pré-malar, conforme o caso).
\item
  \textbf{Restaurar altura (quando necessário):} em retração
  significativa, planejar \textbf{aumento vertical} com \emph{spacer}
  (enxerto/implante) antes ou junto da fixação lateral.
\item
  Ancoragem robusta:
\end{enumerate}

\begin{itemize}
\tightlist
\item
  quando periósteo não ``segura'', usar ancoragem mais profunda ou
  perfuração óssea (\emph{drill hole}) no rebordo lateral para fixação
  estável,
\item
  manter vetor póstero-superior e evitar fixação anterior.
\end{itemize}

\begin{enumerate}
\def\labelenumi{\arabic{enumi}.}
\setcounter{enumi}{3}
\tightlist
\item
  Complementos frequentes: midface lift / suporte malar para reduzir
  carga descendente na pálpebra inferior (especialmente no vetor
  negativo).
\end{enumerate}

\textbf{Nota prática:} McCord não é ``uma sutura mais forte''; é um
\textbf{pacote de revisão}: liberar, reposicionar, (às vezes) alongar e
só então fixar.

\section{Erros comuns (e como
resgatar)}\label{erros-comuns-e-como-resgatar-18}

\begin{itemize}
\item
  \textbf{Cisto de inclusão no Strip}
\item
  \emph{Causa:} epitélio/conjuntiva remanescentes no segmento enterrado.
\item
  \emph{Prevenção:} desepitelização completa e revisão visual do strip
  antes de fixar.
\item
  \emph{Resgate:} drenagem + curetagem/capsulectomia.
\item
  \textbf{Fish-mouthing / canto aberto}
\item
  \emph{Causa:} reconstrução imprecisa do ângulo, tensão mal
  distribuída, excesso de pele não manejado.
\item
  \emph{Prevenção:} ponto ``cinza-cinza'' preciso; ajustar excesso
  cutâneo com parcimônia.
\item
  \emph{Resgate:} revisão cantal formal (recriar ângulo).
\item
  \textbf{Fimose palpebral (olho ``encurtado'')}
\item
  \emph{Causa:} encurtamento horizontal exagerado como tentativa de
  compensar tensão vertical.
\item
  \emph{Prevenção:} medir, encurtar o mínimo; priorizar \emph{release} e
  suporte malar quando houver tração vertical.
\item
  \emph{Resgate:} revisão com liberação, reposicionamento e, se
  necessário, alongamento (\emph{spacer}).
\item
  \textbf{Gapping por fixação anterior}
\item
  \emph{Causa:} sutura ``na borda'' palpável do rebordo, sem
  profundidade.
\item
  \emph{Prevenção:} buscar ancoragem interna/póstero-superior.
\item
  \emph{Resgate:} reposicionar fixação (idealmente precoce).
\item
  \textbf{Assimetria vertical (um olho ``cat-eye'' unilateral)}
\item
  \emph{Prevenção:} referência objetiva bilateral (marcação/medida
  relativa ao canto medial/pupila), e checagem intraoperatória sempre
  que possível.
\item
  \emph{Resgate:} revisão do ponto de ancoragem; tratar a causa (não só
  ``baixar'' o lado bom).
\end{itemize}

\section{Notas de ``arte'' (preservando a
fenda)}\label{notas-de-arte-preservando-a-fenda}

O erro estético clássico é tentar ganhar sustentação apenas
``apertando'' horizontalmente.

\textbf{Regra de arte:} quanto maior a tendência a retração (vetor
negativo, pele senil, cicatriz), mais você deve pensar em:

\begin{enumerate}
\def\labelenumi{\arabic{enumi})}
\item
  \textbf{reduzir tensão vertical} (\emph{release}, suporte malar),
\item
  \textbf{distribuir forças} (fixação póstero-superior),
\item
  encurtar o mínimo necessário (preservar largura da fenda).
\end{enumerate}

Olho pequeno + Strip agressivo = aparência operada e risco funcional.
Faça o olho ``encostar'' no globo, não ``diminuir''.

\section{Pós-operatório}\label{puxf3s-operatuxf3rio-10}

\begin{itemize}
\item
  \textbf{Dor cantal:} mais intensa nas técnicas com fixação em
  periósteo/ancoragem óssea; analgesia planejada nas primeiras 48h.
\item
  \textbf{Sensação de areia / quemose:} comum após manipulação cantal;
  lubrificação é obrigatória.
\item
  \textbf{Rigidez temporária:} a pálpebra pode parecer ``puxada'' por
  2--3 semanas; alinhar expectativa e monitorar exposição.
\end{itemize}

\section{Referências}\label{referuxeancias-17}

\begin{itemize}
\item
  Suspensão muscular lateral e prevenção de ectrópio
  {[}{[}REF:MLADICK-1979{]}{]} Mladick.
\item
  Lateral Tarsal Strip e indicação biomecânica
  {[}{[}REF:ANDERSON-1979{]}{]} Anderson.
\item
  Revisões complexas: \emph{release}, ancoragens robustas e manejo de
  retração {[}{[}REF:MCCORD-1995{]}{]} McCord. * * *
\end{itemize}

\begin{center}\rule{0.5\linewidth}{0.5pt}\end{center}

\chapter{Capítulo 24 --- Microfat: coleta, preparo e injeção; zonas e
volumes
(justa-periostal)}\label{capuxedtulo-24-microfat-coleta-preparo-e-injeuxe7uxe3o-zonas-e-volumes-justa-periostal}

\begin{figure}
\centering
\pandocbounded{\includegraphics[keepaspectratio,alt={Figura 24.1 --- Ilustração principal do capítulo}]{/Users/humbertolopes/Dev/work/marcelo-cury/the_art_of_eyelid_surgery_scaffold/projects/eyelid-surgery/assets/figures/FIG-24-01_microfat-fluxo-zonas.png}}
\caption{Figura 24.1 --- Ilustração principal do capítulo}
\end{figure}

\textbf{Parte:} Parte VI --- Adjuntos

\textbf{Objetivo do capítulo:} Ao final, o leitor dominará a
lipoenxertia estrutural (\textbf{Microfat}) como ferramenta para
restaurar a convexidade orbitária e o suporte malar, aprendendo a
depositar gordura no plano profunda/justa-periostal com microalíquotas,
reduzindo irregularidades visíveis e evitando o erro clássico: ``volume
no plano errado''.

\begin{quote}
\textbf{Nota de escopo:} conteúdo educacional para cirurgiões em
treinamento formal. Não substitui supervisão, anatomia aplicada e
protocolos institucionais.
\end{quote}

\section{O que muda na decisão (o
``porquê'')}\label{o-que-muda-na-decisuxe3o-o-porquuxea-23}

\begin{itemize}
\item
  \textbf{Volume é suporte (não maquiagem):} Em face desinflada,
  ``tirar'' (blef subtrativa) apenas revela o esqueleto. Microfat bem
  indicado \textbf{encurta visualmente} a pálpebra inferior ao ``subir''
  a bochecha com volume profundo e suaviza o rebordo infraorbitário por
  transição.
\item
  \textbf{Profundidade é segurança estética:} Gordura volumétrica
  superficial (subcutânea / intramuscular superficial) na região
  periorbital vira nódulo, sombra e contorno iatrogênico. O alvo
  seguro/estável é profundo: abaixo do orbicular, próximo ao osso.
\item
  Gordura ≠ AH: A gordura integra e pode ser duradoura, mas é
  irreversível na prática (tirar excesso é difícil e imprevisível). A
  estratégia correta é sub-corrigir e, se necessário, retocar após
  maturação.
\end{itemize}

\begin{quote}
\textbf{PÉROLA CLÍNICA}
\end{quote}

\begin{quote}
\mbox{}%
\subsection{Checklist de Planejamento
(Pré-op)}\label{checklist-de-planejamento-pruxe9-op}
\end{quote}

\begin{quote}
\begin{itemize}
\tightlist
\item[$\square$]
  \textbf{Defeito dominante:} concavidade (sulco/``A-frame''/rebordo)
  vs.~convexidade/edema (festoon).
\end{itemize}
\end{quote}

\begin{quote}
\begin{itemize}
\tightlist
\item[$\square$]
  \textbf{Histórico de HA (mesmo antigo):} dissolver previamente
  (idealmente guiado) para enxergar o volume real.
\end{itemize}
\end{quote}

\begin{quote}
\begin{itemize}
\tightlist
\item[$\square$]
  \textbf{Instabilidade ponderal:} grande oscilação de peso = resultado
  menos previsível.
\end{itemize}
\end{quote}

\begin{quote}
\begin{itemize}
\tightlist
\item[$\square$]
  Edema malar/festoon: cautela --- volume pode piorar drenagem/estase.
\end{itemize}
\end{quote}

\begin{quote}
\begin{itemize}
\tightlist
\item[$\square$]
  Escolha do produto:
\end{itemize}
\end{quote}

\begin{quote}
\begin{itemize}
\tightlist
\item
  Microfat = volume profundo
\end{itemize}
\end{quote}

\begin{quote}
\begin{itemize}
\tightlist
\item
  Nanofat = qualidade de pele (não volume) \textgreater{} \textbf{Área
  doadora:} abdome inferior (frequente escolha) / face interna de coxa.
\end{itemize}
\end{quote}

\begin{quote}
\textbf{Instrumental:} coleta 2.4--3.0mm multi-furos; injeção romba
22--25G (≈0.7--0.9mm) + seringas de 1ml (Luer-lock).
\end{quote}

\begin{quote}
\textbf{Volumes práticos (conservadores):}
\end{quote}

\begin{quote}
\begin{itemize}
\tightlist
\item
  Tear trough (medial): \textbf{0.2 -- 0.6 ml} / lado
\end{itemize}
\end{quote}

\begin{quote}
\begin{itemize}
\tightlist
\item
  Junção pálpebra-malar (lateral): \textbf{0.8 -- 2.0 ml} / lado
\end{itemize}
\end{quote}

\begin{quote}
\begin{itemize}
\tightlist
\item
  ROOF / cauda do supercílio: 0.5 -- 1.5 ml / lado
\end{itemize}
\end{quote}

\begin{quote}
\emph{(Retoque tardio é preferível à sobrecorreção.)}
\end{quote}

\section{Anatomia aplicada (o alvo
seguro)}\label{anatomia-aplicada-o-alvo-seguro}

\begin{itemize}
\item
  \textbf{Compartimentos profundos-alvo:}
\item
  \textbf{Espaço pré-zigomático} e \textbf{pré-maxilar}: planos
  profundos que aceitam volume com menor risco de ``bolinhas'' visíveis.
\item
  Justa-periostal: a cânula ``raspa'' o osso --- isso é feedback táctil
  de profundidade correta.
\item
  \textbf{Zonas de interdição / respeito anatômico:}
\item
  \textbf{Forame infraorbital:} geralmente na linha pupilar, abaixo do
  rebordo (varia). Evitar trajetória agressiva/cega.
\item
  \textbf{Veia/Artéria angular (medial):} risco em trajetos muito
  mediais.
\item
  ORL (Ligamento Retentor do Orbicular): funciona como ``borda''
  biomecânica; volume profundo pode suportar a transição, mas o
  ligamento limita ``migração'' se você tentar preencher \emph{acima}
  dele.
\end{itemize}

\begin{quote}
\textbf{📎 FIGURA NECESSÁRIA (Cap. 24):}
\end{quote}

\begin{quote}
Técnica de coleta: Seringa, cânula, processamento
\end{quote}

\begin{quote}
\emph{Estilo: Diagrama técnico-didático, cores neutras, legendas claras}
\end{quote}

\section{Técnica (do preparo à
injeção)}\label{tuxe9cnica-do-preparo-uxe0-injeuxe7uxe3o}

\subsection{1) Coleta (Harvest)}\label{coleta-harvest}

\begin{itemize}
\item
  \textbf{Tumescer} a área doadora de forma homogênea.
\item
  \textbf{Aspiração de baixa pressão:} seringa 10ml ou vácuo baixo.
  Pressão alta = maior trauma adipocitário.
\item
  \textbf{Cânula pequena multi-furos:} para lóbulos menores e mais
  uniformes (Microfat já ``nasce'' mais fino).
\end{itemize}

\subsection{2) Preparo (Refino)}\label{preparo-refino}

Objetivo: remover óleo livre, sangue e anestésico mantendo tecido viável
e injetável.

\begin{itemize}
\item
  \textbf{Decantação:} seringa vertical por alguns minutos → descartar
  sobrenadante oleoso e infranadante (soro/sangue), usar o ``miolo''.
\item
  \textbf{Ou centrifugação suave (Coleman):} preferir descrever em
  \textbf{RCF (g)}, pois rpm depende do raio. Estratégia: curta duração
  e força moderada para não ``secar'' demais nem destruir células.
\item
  Transferência para seringas de 1ml: precisão \textgreater{}
  velocidade.
\end{itemize}

\subsection{3) Injeção
(Justa-periostal)}\label{injeuxe7uxe3o-justa-periostal}

\begin{itemize}
\item
  \textbf{Ponto de entrada:} geralmente lateral (1--2cm do canto
  externo) ou malar lateral --- evita marca central e dá ângulo seguro.
\item
  \textbf{Instrumento:} cânula romba (22--25G). \textbf{Evitar agulha
  afiada} para volume profundo periorbitário.
\item
  Profundidade: avançar até contato ósseo claro.
\item
  Depósito: sempre em retrografia, em microtúneis cruzados
  (\emph{cross-hatching}), com microalíquotas.
\item
  Regra prática: ``muitas passadas, pouquíssima gordura por passada''.
\item
  Evitar bolus: bolus = necrose central / irregularidade.
\item
  \textbf{Controle visual/tátil:} a superfície não deve ``empipocar''
  com a passagem da cânula. Se a pele ``levanta'' facilmente, você ficou
  superficial demais.
\end{itemize}

\subsection{Variações e
indicações}\label{variauxe7uxf5es-e-indicauxe7uxf5es-12}

\begin{itemize}
\item
  \textbf{Microfat + liberação ligamentar:} quando o ORL/TTL cria degrau
  rígido, liberação adequada (no plano correto) melhora a integração do
  volume.
\item
  \textbf{Microfat + Nanofat:} Microfat faz rampa/convexidade; Nanofat
  complementa qualidade de pele (textura e olheira pigmentar) .
\end{itemize}

\begin{quote}
\textbf{PÉROLA CLÍNICA}
\end{quote}

\begin{quote}
\mbox{}%
\subsection{Zona de Risco: Vascular (evento grave e raro, mas
real)}\label{zona-de-risco-vascular-evento-grave-e-raro-mas-real}
\end{quote}

\begin{quote}
Mesmo com cânula romba e plano profundo, \textbf{injeção intravascular}
é um risco teórico/possível em qualquer preenchimento facial.
\end{quote}

\begin{quote}
\textbf{Prevenção por princípio:}
\end{quote}

\begin{quote}
\begin{itemize}
\tightlist
\item
  cânula romba + baixa pressão + retrografia + microalíquotas
\end{itemize}
\end{quote}

\begin{quote}
\begin{itemize}
\tightlist
\item
  evitar trajetos mediais agressivos e ``pontos cegos''
\end{itemize}
\end{quote}

\begin{quote}
\begin{itemize}
\tightlist
\item
  nunca ``forçar'' quando houver resistência
\end{itemize}
\end{quote}

\begin{quote}
\textbf{Regra mental:} se acontecer o improvável, o dano é enorme;
portanto, a técnica deve ser construída para que o improvável seja ainda
menos provável.
\end{quote}

\section{Erros comuns (e como
resgatar)}\label{erros-comuns-e-como-resgatar-19}

\begin{itemize}
\item
  Injeção superficial (subcutânea / intramuscular superficial)
\item
  \emph{Consequência:} nódulos, ``salsicha'', contorno amarelo visível,
  irregularidade crônica.
\item
  \emph{Prevenção:} contato ósseo + microalíquotas + múltiplos túneis.
\item
  \emph{Resgate:} imediato: moldagem suave. Tardio: manejo é difícil e
  variável (às vezes microaspiração/revisão). Evitar prometer ``remoção
  fácil''.
\item
  \textbf{Sobrecorreção (face ``pesada'')}
\item
  \emph{Consequência:} perda de definição, aparência inchada
  persistente, assimetria.
\item
  \emph{Prevenção:} sub-corrigir e \textbf{retocar} após maturação.
\item
  \emph{Resgate:} geralmente expectante; correção ativa pode ser
  imprevisível.
\item
  \textbf{Depósito em bolus}
\item
  \emph{Consequência:} necrose gordurosa, nódulo duro, calcificação,
  irregularidade palpável.
\item
  \emph{Prevenção:} abolir bolus. ``Fiozinhos'' em vez de ``pelotas''.
\item
  \emph{Resgate:} tempo + observação; intervenções tardias selecionadas.
\end{itemize}

\section{Notas de ``arte'' (convexidade e
luz)}\label{notas-de-arte-convexidade-e-luz}

A juventude é uma superfície contínua. O objetivo do Microfat na junção
pálpebra-malar é criar uma \textbf{rampa}: a luz toca a bochecha e
``escorre'' até a margem infraorbital sem tropeçar num sulco.

\textbf{Critério estético simples:} qualquer ``degrau'' que cria sombra
sob flash vira ``olheira'' na vida real. O enxerto profundo deve apagar
o degrau --- não criar um novo.

\section{Pós-operatório}\label{puxf3s-operatuxf3rio-11}

\begin{itemize}
\item
  \textbf{Edema e assimetria:} esperados. Pico nos primeiros dias;
  retorno social costuma ocorrer antes da maturação real.
\item
  \textbf{Induração (``phantom lumps''):} endurecimento transitório pode
  durar semanas. Alinhar isso antes da cirurgia evita pânico
  desnecessário.
\item
  Resultado final: 3--6 meses. Evitar ``retoques ansiosos'' precoces
  (erro comum que leva à sobrecorreção em cascata).
\end{itemize}

\section{Referências}\label{referuxeancias-18}

\begin{itemize}
\item
  Princípios de lipoestrutura e técnica clássica
  {[}{[}REF:COLEMAN-1997{]}{]} Coleman.
\item
  Compartimentos profundos e arquitetura de volume facial
  {[}{[}REF:ROHRICH-2008{]}{]} Rohrich \& Pessa.
\item
  Microfat vs.~Nanofat: indicações e preparo
  {[}{[}REF:TONNARD-2013{]}{]} Tonnard \& Verpaele. * * *
\end{itemize}

\begin{center}\rule{0.5\linewidth}{0.5pt}\end{center}

\chapter{Capítulo 25 --- Nanofat e qualidade de pele: cicatrizes,
olheiras, textura e
microagulhamento}\label{capuxedtulo-25-nanofat-e-qualidade-de-pele-cicatrizes-olheiras-textura-e-microagulhamento}

\begin{figure}
\centering
\pandocbounded{\includegraphics[keepaspectratio,alt={Figura 25.1 --- Ilustração principal do capítulo}]{/Users/humbertolopes/Dev/work/marcelo-cury/the_art_of_eyelid_surgery_scaffold/projects/eyelid-surgery/assets/figures/FIG-25-01_nanofat-qualidade-pele.png}}
\caption{Figura 25.1 --- Ilustração principal do capítulo}
\end{figure}

\textbf{Parte:} Parte VI --- Adjuntos

\textbf{Objetivo do capítulo:} Ao final, o leitor diferenciará
\textbf{regeneração tecidual} de \textbf{preenchimento volumétrico},
dominando o processamento mecânico da gordura (emulsificação +
filtragem) para tratar discromias, transparência vascular e linhas finas
com Nanofat, sem incorrer no risco de nódulos superficiais típico do uso
inadequado de Microfat.

\begin{quote}
\textbf{Nota de escopo:} conteúdo educacional para prática cirúrgica
supervisionada. Nanofat melhora pele; não corrige ``arquitetura''
(sulcos profundos) e não substitui diagnóstico de causa
sistêmica/dermatológica de hiperpigmentação.
\end{quote}

\section{O que muda na decisão (o
``porquê'')}\label{o-que-muda-na-decisuxe3o-o-porquuxea-24}

\begin{itemize}
\item
  \textbf{Biologia vs.~Volume:}
\item
  \textbf{Microfat} = efeito \textbf{mecânico} (preenche, sustenta,
  ``faz rampa'' profunda).
\item
  Nanofat = efeito biológico/parácrino (melhora derme, qualidade,
  espessura, vascularização aparente). O erro clássico é esperar
  ``projeção'' do Nanofat (não entrega) ou tentar ``melhorar pele'' com
  Microfat superficial (entrega grumo).
\item
  \textbf{A olheira é 3 doenças em uma:}
\item
  \textbf{Estrutural} (sombra por degrau/contorno): resolve com
  reposicionamento/volume profundo.
\item
  \textbf{Vascular/transparência} (roxo/azulado): pode melhorar com
  espessamento dérmico e barreira óptica (Nanofat).
\item
  Pigmentar (marrom/melanina): pode melhorar parcialmente (Nanofat +
  modalidades dermatológicas), mas exige expectativa realista.
\item
  \textbf{Cicatrização modulada:} Em áreas de tensão (canto lateral,
  subciliar), Nanofat intradérmico no fechamento pode \textbf{suavizar a
  resposta inflamatória} e melhorar qualidade de cicatriz --- não por
  ``milagre'', mas por microambiente de reparo.
\end{itemize}

\begin{quote}
\textbf{PÉROLA CLÍNICA}
\end{quote}

\begin{quote}
\mbox{}%
\subsection{Checklist de Planejamento
(Pré-op)}\label{checklist-de-planejamento-pruxe9-op-1}
\end{quote}

\begin{quote}
\begin{itemize}
\tightlist
\item[$\square$]
  \textbf{Tipo de olheira (documentar com flash):} Pigmentar (marrom)
  vs.~Vascular (roxa/azul) vs.~Estrutural (sombra).
\end{itemize}
\end{quote}

\begin{quote}
\begin{itemize}
\tightlist
\item[$\square$]
  \textbf{Textura da pele:} Fina (``papel de cigarro'') = melhor
  resposta visual; Espessa/sebácea = efeito mais discreto.
\end{itemize}
\end{quote}

\begin{quote}
\begin{itemize}
\tightlist
\item[$\square$]
  \textbf{Histórico de HA / bioestimulador na região:} dissolver/avaliar
  antes para não confundir edema/coloração.
\end{itemize}
\end{quote}

\begin{quote}
\begin{itemize}
\tightlist
\item[$\square$]
  Risco infeccioso: acne ativa, dermatite, blefarite não tratada =
  adiar.
\end{itemize}
\end{quote}

\begin{quote}
\begin{itemize}
\tightlist
\item[$\square$]
  Plano de entrega: injeção intradérmica (pápulas/linhas) e/ou
  ``drug-delivery'' por microagulhamento.
\end{itemize}
\end{quote}

\begin{quote}
\begin{itemize}
\tightlist
\item[$\square$]
  Kit: conexões Luer-lock rosqueadas, transferidores (estreitos),
  filtros/malha estéril, seringas 1 ml, agulhas finas (30--32G).
\end{itemize}
\end{quote}

\section{Anatomia aplicada (o alvo
microscópico)}\label{anatomia-aplicada-o-alvo-microscuxf3pico}

\begin{itemize}
\item
  \textbf{Derme papilar/reticular:} alvo do Nanofat. A aplicação correta
  gera \textbf{branqueamento transitório} (\emph{blanching}) e
  micro-pápulas que desaparecem em minutos--horas.
\item
  Rede vascular subdérmica e transparência do orbicular: em pálpebra
  inferior muito fina, a cor violácea é ``músculo aparecendo''. Nanofat
  funciona como \textbf{opacificação biológica} (espessamento +
  remodelamento).
\item
  Zona de cautela vascular: glabela, dorso nasal e territórios terminais
  têm risco teórico de eventos isquêmicos com injeções intradérmicas sob
  alta pressão. O princípio aqui é microdose + baixa pressão + sem
  bolus.
\end{itemize}

\begin{quote}
\textbf{📎 FIGURA NECESSÁRIA (Cap. 25):}
\end{quote}

\begin{quote}
Processamento: Microfat → Nanofat (emulsificação)
\end{quote}

\begin{quote}
\emph{Estilo: Diagrama técnico-didático, cores neutras, legendas claras}
\end{quote}

\section{Técnica (do ``ouro líquido'' à
aplicação)}\label{tuxe9cnica-do-ouro-luxedquido-uxe0-aplicauxe7uxe3o}

\subsection{Visão geral do
processamento}\label{visuxe3o-geral-do-processamento}

\begin{enumerate}
\def\labelenumi{\arabic{enumi}.}
\item
  \textbf{Coleta:} semelhante ao Microfat (baixa pressão, cânulas finas,
  trauma mínimo).
\item
  \textbf{Lavagem/decantação:} remover sangue, anestésico e óleo livre.
\item
  \textbf{Emulsificação (o passo-chave):} processamento mecânico
  repetido até perder a granularidade e virar um creme fluido.
\item
  Filtragem: remover fibras/conectivo que entopem agulhas finas e viram
  ``microgrumos''.
\item
  Transferência para 1 ml: precisão e controle de pressão.
\end{enumerate}

\subsection{Emulsificação (princípio, não
ritual)}\label{emulsificauxe7uxe3o-princuxedpio-nuxe3o-ritual}

\begin{itemize}
\item
  O objetivo não é ``contar repetições'', e sim alcançar \textbf{fluidez
  real}:
\item
  Se não passa com conforto por agulha fina, \textbf{não está pronto}.
\item
  Conexões \textbf{Luer-lock} sempre, bem rosqueadas. Pressão alta +
  encaixe fraco = acidente de bancada.
\end{itemize}

\subsection{Filtragem}\label{filtragem}

\begin{itemize}
\item
  Use filtro/malha estéril apropriada para remover ``fiapos'' e
  microfragmentos.
\item
  Filtragem insuficiente = entupimento, bolus forçado, irregularidade.
\end{itemize}

\subsection{Aplicação intradérmica
(injeção)}\label{aplicauxe7uxe3o-intraduxe9rmica-injeuxe7uxe3o}

\begin{itemize}
\item
  \textbf{Instrumento:} agulha fina (30--32G) com seringa 1 ml.
\item
  \textbf{Plano:} intradérmico superficial (pápulas) ou intradérmico em
  ``linhas'' (threading).
\item
  \textbf{Técnica:} microalíquotas, baixa pressão, retrogradamente
  quando aplicável.
\item
  Ponto crítico: não perseguir ``volume''. O endpoint é uniformidade de
  pele, não projeção.
\end{itemize}

\subsection{Aplicação por microagulhamento
(drug-delivery)}\label{aplicauxe7uxe3o-por-microagulhamento-drug-delivery}

\begin{itemize}
\item
  \textbf{Conceito:} criar microcanais superficiais e usar Nanofat como
  ``meio biológico'' para contato dérmico amplo.
\item
  \textbf{Indicações práticas:} textura difusa, poros, cicatrizes finas
  e área grande onde a injeção ponto a ponto é impraticável.
\item
  \textbf{Cautela:} região palpebral exige conservadorismo e proteção
  ocular; a meta é estímulo controlado, não trauma.
\end{itemize}

\subsection{Variações (quando fizer
sentido)}\label{variauxe7uxf5es-quando-fizer-sentido}

\begin{itemize}
\item
  \textbf{Nanofat + PRP (``Smart Nanofat''):} popular, mas protocolos
  variam; tratar como adjunto e não como promessa.
\item
  \textbf{Nanofat em cicatriz (no fechamento):} microinfiltração
  intradérmica ao longo da linha de sutura, sem tensão.
\end{itemize}

\begin{quote}
\textbf{PÉROLA CLÍNICA}
\end{quote}

\begin{quote}
\mbox{}%
\subsection{Erro Nota 7: ``Entupimento'' (e o falso
Nanofat)}\label{erro-nota-7-entupimento-e-o-falso-nanofat}
\end{quote}

\begin{quote}
\textbf{Regra simples:} se não passa livremente por agulha 30G,
\textbf{não é Nanofat} --- é gordura mal processada (microgrumos).
\end{quote}

\begin{quote}
\textbf{Risco:} cistos de óleo, nódulos e irregularidades visíveis na
pele fina periorbital.
\end{quote}

\begin{quote}
Correção: voltar à bancada → emulsificar/filtrar até fluidez real.
\end{quote}

\section{Erros comuns (e como
resgatar)}\label{erros-comuns-e-como-resgatar-20}

\begin{itemize}
\item
  \textbf{Usar Nanofat para volumizar sulco profundo}
\item
  \emph{Consequência:} frustração (quase nenhum efeito no relevo).
\item
  \emph{Prevenção:} algoritmo: relevo = Microfat profundo /
  reposicionamento; cor/textura = Nanofat.
\item
  \emph{Resgate:} replanejar com técnica correta após maturação.
\item
  \textbf{Injetar profundo demais (subcutâneo)}
\item
  \emph{Consequência:} menor efeito dérmico; melhora discreta.
\item
  \emph{Prevenção:} endpoint de branqueamento/pápulas + resistência
  intradérmica.
\item
  \emph{Resgate:} repetir no plano correto após intervalo biológico
  (meses).
\item
  \textbf{Pressão excessiva / bolus}
\item
  \emph{Consequência:} risco teórico de compressão vascular em
  territórios sensíveis; além de irregularidade.
\item
  \emph{Prevenção:} microalíquotas + baixa pressão + sem ``forçar
  passagem''.
\item
  \emph{Resgate:} interrupção imediata do ato e condução conforme
  protocolo de segurança/urgência institucional, se houver sinais de
  isquemia.
\item
  \textbf{Contaminação por acne/blefarite ativa}
\item
  \emph{Consequência:} infecção profunda (desastre raro, mas evitável).
\item
  \emph{Prevenção:} pele ``limpa'' e condição controlada antes do
  procedimento.
\item
  \emph{Resgate:} manejo clínico/cirúrgico conforme gravidade.
\end{itemize}

\section{Notas de ``arte'' (o ``glow''
realista)}\label{notas-de-arte-o-glow-realista}

Nanofat não ``muda a forma''; muda \textbf{como a luz se comporta} na
pele.

A melhora costuma ser \textbf{progressiva} (colágeno/derme remodelando),
e não imediata. O acabamento ideal é ``acetinado'', não
``plastificado''.

\section{Pós-operatório e
follow-up}\label{puxf3s-operatuxf3rio-e-follow-up-3}

\begin{itemize}
\item
  \textbf{Eritema/edema local:} esperado; pode ser mais evidente se
  associado a microagulhamento.
\item
  \textbf{Equimose puntiforme:} possível por punção intradérmica.
\item
  \textbf{Tempo biológico:} reforçar que o pico de melhora de
  textura/cor é tardio (meses), não dias.
\item
  Sessões: em olheiras pigmentares/vasculares, melhora pode exigir
  abordagem seriada e multimodal (ex.: cuidado dermatológico + nanofat),
  sem promessa de ``zerar'' a cor.
\end{itemize}

\section{Referências}\label{referuxeancias-19}

\begin{itemize}
\item
  Definição e processamento mecânico do Nanofat
  {[}{[}REF:TONNARD-2013{]}{]} Tonnard \& Verpaele.
\item
  Uso de enxerto emulsificado em cicatrizes/qualidade cutânea
  {[}{[}REF:COHEN-2017{]}{]} Cohen.
\item
  Arquitetura de compartimentos faciais e implicações ópticas
  {[}{[}REF:ROHRICH-2008{]}{]} Rohrich. * * *
\end{itemize}

\begin{center}\rule{0.5\linewidth}{0.5pt}\end{center}

\chapter{Capítulo 26 --- Funcional e reconstrução:
ectrópio/entrópio/retração e princípios das
lamelas}\label{capuxedtulo-26-funcional-e-reconstruuxe7uxe3o-ectruxf3pioentruxf3pioretrauxe7uxe3o-e-princuxedpios-das-lamelas}

\begin{figure}
\centering
\pandocbounded{\includegraphics[keepaspectratio,alt={Figura 26.1 --- Ilustração principal do capítulo}]{/Users/humbertolopes/Dev/work/marcelo-cury/the_art_of_eyelid_surgery_scaffold/projects/eyelid-surgery/assets/figures/FIG-26-01_ectropio-entropio-lamelas.png}}
\caption{Figura 26.1 --- Ilustração principal do capítulo}
\end{figure}

\textbf{Parte:} Parte VII --- Funcional e Reconstrução

\textbf{Objetivo do capítulo:} Ao final, o leitor será capaz de
diagnosticar a etiologia das malposições palpebrais pela análise das
\textbf{lamelas} (anterior, média e posterior) e aplicar o princípio
cirúrgico central da reconstrução palpebral: ``alongar o que está curto
e tencionar o que está frouxo'' --- com um algoritmo reprodutível que
reduz recidivas.

\section{O que muda na decisão (o
``porquê'')}\label{o-que-muda-na-decisuxe3o-o-porquuxea-25}

\begin{itemize}
\item
  O Princípio das Lamelas (diagnóstico antes do bisturi): O erro-mãe em
  reconstrução é tratar \textbf{a lamela errada}.
\item
  \textbf{Lamela anterior curta} (pele/orbicular cicatrizados) →
  pálpebra ``puxa'' para fora (\textbf{ectrópio cicatricial}) e piora
  com boca aberta.
\item
  Lamela posterior curta (tarso/conjuntiva/retratores) → margem roda
  para dentro (entrópio cicatricial) ou desce (retração), com fórnice
  raso e simbléfaro.
\item
  Frouxidão horizontal (tendão cantal/tarso ``solto'') → ``permite''
  qualquer deformidade acontecer e recidivar.
\item
  Frouxidão horizontal é o denominador comum (mas não é a única doença):
  Em ectrópio/entrópio involucionais, a laxidão do canto lateral é quase
  sempre permissiva. Sem estabilizar a horizontal (Strip), qualquer
  correção vertical vira paliativa. Porém: em cicatriciais,
  \textbf{Strip sozinho falha} se a pele ou conjuntiva estão encurtadas.
\item
  \textbf{A estabilidade dos retratores define o entrópio involucional:}
  No entrópio senil, a desinserção/afrouxamento dos retratores (fáscia
  capsulopalpebral) permite que o orbicular pré-septal ``suba'' sobre o
  pré-tarsal e \textbf{rode a margem para dentro}. Remover pele não
  trata essa mecânica; é uma correção ``cosmética'' para um defeito
  estrutural.
\end{itemize}

\begin{quote}
\textbf{📎 FIGURA NECESSÁRIA (Cap. 26):}
\end{quote}

\begin{quote}
Diagnóstico: Ectrópio involucional vs cicatricial vs paralítico
\end{quote}

\begin{quote}
\emph{Estilo: Diagrama técnico-didático, cores neutras, legendas claras}
\end{quote}

\section{Indicações e
contra-indicações}\label{indicauxe7uxf5es-e-contra-indicauxe7uxf5es-16}

\textbf{Indicar correção funcional quando:}

\begin{itemize}
\item
  \textbf{Ectrópio:} exposição conjuntival, epífora por eversão do ponto
  lacrimal, ceratite de exposição, lagoftalmo.
\item
  \textbf{Entrópio:} triquíase (cílios na córnea), dor/corpo estranho,
  risco de úlcera e opacidade corneana.
\item
  \textbf{Retração palpebral inferior:} \emph{scleral show}
  pós-blefaroplastia, cicatriz de retratores, ou doença tireoidiana
  estável (fase não inflamatória).
\end{itemize}

\textbf{Evitar / adiar quando:}

\begin{itemize}
\item
  \textbf{Doença cicatrizante ocular ativa} (ex.: Penfigoide Cicatricial
  Ocular/OCP, Stevens--Johnson): operar só em fase quiescente com doença
  controlada.
\item
  \textbf{Infecção ativa} (conjuntivite purulenta, dacriocistite):
  tratar antes.
\item
  \textbf{Tireoidopatia ocular ativa:} primeiro estabilizar/``esfriar''
  a fase inflamatória; a cirurgia muda de resultado quando o tecido
  ainda está em remodelamento.
\end{itemize}

\begin{quote}
\textbf{PÉROLA CLÍNICA}
\end{quote}

\begin{quote}
\mbox{}%
\subsection{Checklist de Diagnóstico Lamelar (o mapa da
decisão)}\label{checklist-de-diagnuxf3stico-lamelar-o-mapa-da-decisuxe3o}
\end{quote}

\begin{quote}
\begin{itemize}
\tightlist
\item[$\square$]
  \textbf{Distraction Test / Snap-back:} quantifica frouxidão horizontal
  (canto/tarso).
\end{itemize}
\end{quote}

\begin{quote}
\begin{itemize}
\tightlist
\item[$\square$]
  \textbf{Teste da Elevação Malar:} empurre a bochecha para cima →
  melhora do ectrópio sugere componente involucional/tensional (suporte)
  e necessidade de suporte lateral/midface; pouca melhora sugere
  encurtamento real (pele/conjuntiva).
\end{itemize}
\end{quote}

\begin{quote}
\begin{itemize}
\tightlist
\item[$\square$]
  \textbf{Teste da Boca Aberta:} piora do ectrópio com boca aberta =
  lamela anterior curta (pele).
\end{itemize}
\end{quote}

\begin{quote}
\begin{itemize}
\tightlist
\item[$\square$]
  Fórnice inferior: está raso? há simbléfaro? = lamela posterior curta
  (conjuntiva/retratores).
\end{itemize}
\end{quote}

\begin{quote}
\begin{itemize}
\tightlist
\item[$\square$]
  Excursão dos retratores: olhando para baixo, a pálpebra desce? pouca
  mobilidade = fibrose/retração.
\end{itemize}
\end{quote}

\begin{quote}
\begin{itemize}
\tightlist
\item[$\square$]
  Ponto lacrimal: evertido? (ectrópio medial frequentemente precisa de
  correção específica além do Strip).
\end{itemize}
\end{quote}

\section{Anatomia aplicada (o sistema de
camadas)}\label{anatomia-aplicada-o-sistema-de-camadas}

\begin{itemize}
\item
  \textbf{Lamela Anterior (cobertura):} pele + orbicular. Encurtamento →
  ectrópio cicatricial (puxa margem para fora).
\item
  \textbf{Lamela Posterior (forro e ``trilho''):} tarso + conjuntiva +
  retratores. Encurtamento → entrópio cicatricial / \textbf{retração
  palpebral} (descida vertical, fórnice raso).
\item
  \textbf{Lamela Média (plano de deslizamento):} septo + gordura. A
  cicatrização ``colando'' septo/retratores é um motor comum de
  \textbf{retração pós-blefaroplastia}: perde-se o deslizamento e a
  pálpebra fica ``presa'' inferiormente.
\end{itemize}

\section{Técnica (Algoritmo de
Correção)}\label{tuxe9cnica-algoritmo-de-correuxe7uxe3o}

\subsection{Visão geral (ordem que reduz
recidiva)}\label{visuxe3o-geral-ordem-que-reduz-recidiva}

\begin{enumerate}
\def\labelenumi{\arabic{enumi}.}
\item
  \textbf{Diagnóstico:} é \textbf{frouxidão}, \textbf{encurtamento}
  (anterior/posterior) ou ambos?
\item
  Passo 1 (quase universal): estabilizar a horizontal com Lateral Tarsal
  Strip (ou técnica equivalente) quando houver laxidade patológica.
\item
  Passo 2 (específico por lamela):
\end{enumerate}

\begin{itemize}
\tightlist
\item
  Ectrópio involucional: Strip ± correção medial (ver zona de risco) ±
  suporte adicional se necessário.
\item
  Ectrópio cicatricial (lamela anterior curta): Strip + liberação
  completa de cicatrizes + enxerto de pele total (FTSG) adequadamente
  dimensionado.
\item
  Entrópio involucional: Strip (se laxidade) + reinserção/plicatura dos
  retratores (Jones/Weiss) ± suturas rotacionais.
\item
  Entrópio cicatricial (lamela posterior curta): liberar
  simbléfaro/aderências + substituir lamela posterior
  (mucosa/conjuntiva/cartilagem, conforme defeito).
\item
  Retração palpebral inferior: Strip + liberação de retratores + enxerto
  espaçador (spacer) quando houver encurtamento posterior/vertical real.
\end{itemize}

\subsection{``Pacotes'' práticos (o que normalmente anda
junto)}\label{pacotes-pruxe1ticos-o-que-normalmente-anda-junto}

\begin{itemize}
\item
  \textbf{Ectrópio involucional + ponto lacrimal evertido:} Strip
  lateral \textbf{não garante} inversão do ponto → frequentemente
  precisa de correção medial específica (ver BOX abaixo).
\item
  \textbf{Retração pós-blefaroplastia:} raramente é ``só frouxidão''. Em
  geral há cicatriz vertical + laxidade → Strip + release + spacer
  costuma ser o tripé.
\end{itemize}

\begin{quote}
\textbf{PÉROLA CLÍNICA}
\end{quote}

\begin{quote}
\mbox{}%
\subsection{Zona de Risco: O Ponto Lacrimal (ectrópio
medial)}\label{zona-de-risco-o-ponto-lacrimal-ectruxf3pio-medial}
\end{quote}

\begin{quote}
No ectrópio medial, o ponto lacrimal ``vira para fora'' e para de drenar
→ epífora mesmo com canto lateral firme.
\end{quote}

\begin{quote}
\textbf{Regra de ouro:} o Strip lateral melhora a tensão global, mas
pode não ``rodar'' o ponto medial.
\end{quote}

\begin{quote}
\textbf{Solução clássica:} associar \textbf{Plastia em Fuso Medial
(Medial Spindle)} --- ressecção de um losango de conjuntiva abaixo do
ponto lacrimal para rotacioná-lo de volta em direção ao globo,
restabelecendo a captação da lágrima.
\end{quote}

\section{Variações úteis (quando escolher o ``atalho'' com
consciência)}\label{variauxe7uxf5es-uxfateis-quando-escolher-o-atalho-com-consciuxeancia}

\begin{itemize}
\item
  \textbf{Suturas rotacionais (Wies / Quickert):} úteis como solução
  rápida, temporária, ou em pacientes frágeis. Criam eversão por
  cicatriz/rotação. Não substituem correção estrutural em casos graves.
\item
  \textbf{Suporte adicional (midface):} em vetor negativo e retrações
  importantes, suporte da bochecha pode ser decisivo para reduzir
  dependência de ``tensão palpebral pura''.
\item
  \textbf{Spacer de palato duro / cartilagem:} opção robusta quando a
  lamela posterior precisa de rigidez e altura (defeito vertical
  significativo).
\end{itemize}

\section{Erros comuns (e como
resgatar)}\label{erros-comuns-e-como-resgatar-21}

\begin{itemize}
\item
  \textbf{Tratar ectrópio cicatricial sem enxerto (só Strip):}
\item
  \emph{Consequência:} melhora curta; recidiva quando a falta de pele
  vence a tensão horizontal.
\item
  \emph{Prevenção:} se \textbf{boca aberta piora}, pense ``lamela
  anterior curta'' → enxerto é parte do plano.
\item
  \emph{Resgate:} reoperação com liberação ampla + enxerto bem
  dimensionado.
\item
  \textbf{Hipercorreção no entrópio (ectrópio consecutivo):}
\item
  \emph{Consequência:} margem evertida, exposição e epífora.
\item
  \emph{Prevenção:} correção ``neutra a levemente evertida'', sem tensão
  forçada; contorno importa tanto quanto altura.
\item
  \emph{Resgate:} medidas conservadoras no pós-imediato; se persistente,
  ajuste cirúrgico (recuo/liberação).
\item
  \textbf{Enxerto ``pequeno por estética'':}
\item
  \emph{Consequência:} contratura secundária e retorno do ectrópio.
\item
  \emph{Prevenção:} o enxerto deve ser \textbf{generoso} e planejado
  para contrair; pensar função antes de ``invisibilidade''.
\item
  \emph{Resgate:} novo enxerto maior após liberação adequada.
\end{itemize}

\section{Notas de ``arte'' (camuflagem na
reconstrução)}\label{notas-de-arte-camuflagem-na-reconstruuxe7uxe3o}

Reconstrução não precisa ``denunciar'' reconstrução.

\begin{itemize}
\item
  \textbf{Escolha da pele (match):} pálpebra superior (melhor
  cor/textura) \textgreater{} retroauricular (boa alternativa)
  \textgreater{} supraclavicular (tende a ser mais espessa).
\item
  \textbf{Unidades estéticas:} sempre que possível, substitua de forma
  coerente com a unidade, escondendo cicatrizes em linhas naturais e
  evitando o ``patch'' central.
\item
  \textbf{Geometria do canto:} estabilizar sem arredondar. Em revisões,
  o risco de distorção do ângulo é alto; planeje o ``V'' antes de fechar
  pele.
\end{itemize}

\section{Pós-operatório (o que protege o
resultado)}\label{puxf3s-operatuxf3rio-o-que-protege-o-resultado}

\begin{itemize}
\item
  \textbf{Sutura de Frost / tração palpebral:} em casos com enxerto ou
  grande liberação, manter tração superior por 3--5 dias melhora a
  ``pega'' e reduz contratura vertical precoce.
\item
  \textbf{Lubrificação e proteção corneana:} mandatórias até estabilizar
  o fechamento e a apposição.
\item
  \textbf{Edema e assimetria:} esperados; explicar que função vem
  primeiro e refinamentos estéticos são etapa tardia.
\end{itemize}

\section{Referências}\label{referuxeancias-20}

\begin{itemize}
\item
  Classificação e manejo de ectrópio/entrópio e princípios
  reconstrutivos {[}{[}REF:COLLIN-1983{]}{]} Collin.
\item
  Reinserção/plicatura de retratores no entrópio involucional
  {[}{[}REF:WEISS-1979{]}{]} Weiss / Jones.
\item
  Enxertos espaçadores (palato duro/cartilagem) em retração palpebral
  {[}{[}REF:COHEN-2017{]}{]} Cohen / Goldberg. * * *
\end{itemize}

\begin{center}\rule{0.5\linewidth}{0.5pt}\end{center}

\chapter{Capítulo 27 --- Reconstrução pós-tumor: retalhos clássicos
(Tenzel, Hughes, Cutler-Beard,
Mustardé)}\label{capuxedtulo-27-reconstruuxe7uxe3o-puxf3s-tumor-retalhos-cluxe1ssicos-tenzel-hughes-cutler-beard-mustarduxe9}

\begin{figure}
\centering
\pandocbounded{\includegraphics[keepaspectratio,alt={Figura 27.1 --- Ilustração principal do capítulo}]{/Users/humbertolopes/Dev/work/marcelo-cury/the_art_of_eyelid_surgery_scaffold/projects/eyelid-surgery/assets/figures/FIG-27-01_retalhos-classicos.png}}
\caption{Figura 27.1 --- Ilustração principal do capítulo}
\end{figure}

\textbf{Parte:} Parte VII --- Funcional e Reconstrução

\textbf{Objetivo do capítulo:} Ao final, o leitor saberá estratificar
defeitos palpebrais de espessura total por \textbf{porcentagem de
perda}, \textbf{localização} e integridade cantal, selecionando o
algoritmo reconstrutivo (Tenzel, Hughes, Cutler-Beard ou Mustardé) para
restaurar lamela anterior e posterior, priorizando proteção corneana e
função acima da estética imediata.

\begin{quote}
\textbf{Nota de escopo:} Este capítulo é conteúdo técnico para
cirurgiões treinados em reconstrução palpebral. A execução exige
formação, instrumentais e ambiente apropriados.
\end{quote}

\section{O que muda na decisão (o
``porquê'')}\label{o-que-muda-na-decisuxe3o-o-porquuxea-26}

\begin{itemize}
\item
  A Matemática da Tensão (quando ``fechar direto'' vira iatrogenia):
  Defeitos marginais de espessura total \textbf{\textgreater{} 25--30\%}
  raramente toleram fechamento primário sem distorção. Forçar sutura
  direta cria:
\item
  tensão horizontal excessiva (deiscência, ``olho de botão de camisa''),
\item
  deformidade do canto,
\item
  lagoftalmo/ptose mecânica,
\item
  entropion/ectropion cicatricial tardio.
\item
  \textbf{A Regra de Ouro das Lamelas (vascularidade manda):}
  Reconstrução palpebral é \textbf{duas cirurgias em uma}:
\item
  \textbf{Lamela posterior} (tarso + conjuntiva) = trilho rígido e
  mucosa de deslizamento.
\item
  Lamela anterior (pele + músculo) = cobertura e fechamento. Regra
  prática: evite ``dois enxertos livres'' (graft + graft). Pelo menos
  uma lamela deve ser retalho vascularizado para garantir sobrevivência
  e reduzir contratura.
\item
  \textbf{Olho Único e Oclusão (decisão ética e funcional):} Retalhos
  compartilhados em \textbf{2 tempos} (Hughes / Cutler-Beard) ocluem a
  visão por semanas. Em paciente \textbf{monocular} (ou com baixa visão
  contralateral), essas técnicas podem ser funcionalmente inaceitáveis,
  exigindo alternativas (Tenzel estendido, retalhos de avanço, enxertos
  compostos selecionados).
\item
  \textbf{O ``Esqueleto Tarsal'' é inegociável:} Se o defeito envolve a
  margem, \textbf{tarso ou substituto rígido} é mandatário. Reconstruir
  só com pele cria pálpebra ``mole'', predisposta a:
\item
  instabilidade do bordo,
\item
  entropion/ectropion cicatricial,
\item
  atrito corneano crônico.
\end{itemize}

\section{Indicações e contra-indicações (Algoritmo por
tamanho)}\label{indicauxe7uxf5es-e-contra-indicauxe7uxf5es-algoritmo-por-tamanho}

Defeitos de espessura total envolvendo margem palpebral (regra geral):

\begin{itemize}
\item
  \textbf{\textless{} 25\%:} fechamento direto (pentágono) ± cantólise
  lateral se necessário.
\item
  \textbf{25\% a 50--60\%:} \textbf{Tenzel} (retalho semicircular de
  rotação/avanço).
\item
  \begin{quote}
  50\% (Pálpebra Inferior central): Hughes (tarsoconjuntival, 2 tempos).
  \end{quote}
\item
  \begin{quote}
  50\% (Pálpebra Superior central): Cutler-Beard (bridge flap, 2 tempos)
  + suporte rígido (cartilagem).
  \end{quote}
\item
  \begin{quote}
  75\% (Inferior quase total): Mustardé (retalho cérvico-facial) +
  reconstrução posterior (cartilagem/mucosa) e ancoragens altas.
  \end{quote}
\end{itemize}

\textbf{Evitar / adiar quando:}

\begin{itemize}
\item
  Margens tumorais \textbf{incertas} (aguardar ``clearance'' por
  congelação/Mohs antes de grandes retalhos).
\item
  Paciente monocular (Hughes/Cutler-Beard: oclusão temporária).
\item
  \textbf{Crianças pequenas} (risco de ambliopia por oclusão prolongada
  em 2 tempos).
\item
  Doença ocular inflamatória ativa ou superfície ocular instável
  (priorizar estabilização).
\end{itemize}

\begin{quote}
\textbf{PÉROLA CLÍNICA}
\end{quote}

\begin{quote}
\mbox{}%
\subsection{Checklist de Planejamento
(Pré-op)}\label{checklist-de-planejamento-pruxe9-op-2}
\end{quote}

\begin{quote}
\begin{itemize}
\tightlist
\item[$\square$]
  \textbf{Tamanho do defeito (\%)} medido com a pálpebra sob tensão
  fisiológica.
\end{itemize}
\end{quote}

\begin{quote}
\begin{itemize}
\tightlist
\item[$\square$]
  \textbf{Margens livres} confirmadas (Mohs/congelação/histopatologia)
  antes de ``mover'' tecido complexo.
\end{itemize}
\end{quote}

\begin{quote}
\begin{itemize}
\tightlist
\item[$\square$]
  \textbf{Canto medial/lateral}: tendões preservados? precisa
  reconstrução/ancoragem óssea?
\end{itemize}
\end{quote}

\begin{quote}
\begin{itemize}
\tightlist
\item[$\square$]
  Visão contralateral e aceitabilidade de oclusão por semanas.
\end{itemize}
\end{quote}

\begin{quote}
\begin{itemize}
\tightlist
\item[$\square$]
  Laxidez local (têmpora para Tenzel / bochecha e pescoço para
  Mustardé).
\end{itemize}
\end{quote}

\begin{quote}
\begin{itemize}
\tightlist
\item[$\square$]
  Plano lamelar: qual lamela está faltando e qual será retalho vs
  enxerto.
\end{itemize}
\end{quote}

\section{Anatomia aplicada (suprimento e
estrutura)}\label{anatomia-aplicada-suprimento-e-estrutura}

\begin{itemize}
\item
  \textbf{Arcada Vascular Marginal:} Fundamental em retalhos em ponte
  (Cutler-Beard). Incisões muito próximas da margem inferior comprometem
  a perfusão.
\item
  \textbf{Hughes --- doador superior (tarsoconjuntival):} O retalho
  nasce de \textbf{tarso + conjuntiva} superiores, com aporte da
  conjuntiva e estruturas associadas. A dissecção deve respeitar o
  complexo elevador/Müller para evitar sequela na pálpebra doadora.
\item
  \textbf{Mustardé --- risco neural:} Dissecção cérvico-facial profunda
  na bochecha aumenta o risco para ramos zigomáticos/bucais do nervo
  facial; a estratégia de plano e a gentileza na tração são parte da
  técnica.
\end{itemize}

\begin{quote}
\textbf{📎 FIGURA NECESSÁRIA (Cap. 27):}
\end{quote}

\begin{quote}
Retalho de Tenzel: Indicações e passos
\end{quote}

\begin{quote}
\emph{Estilo: Diagrama técnico-didático, cores neutras, legendas claras}
\end{quote}

\section{Técnica (Os 4 Cavaleiros da
Reconstrução)}\label{tuxe9cnica-os-4-cavaleiros-da-reconstruuxe7uxe3o}

\subsection{1) Tenzel (Semicircular
Rotation-Advancement)}\label{tenzel-semicircular-rotation-advancement}

\begin{itemize}
\item
  \textbf{Defeito:} \textasciitilde25--60\% (superior ou inferior),
  principalmente central/lateral.
\item
  \textbf{Ideia:} criar \textbf{reserva horizontal} com retalho
  semicircular lateral e liberar o canto para permitir avanço sem
  deformar a fenda.
\item
  Incisão: arco semicircular partindo do canto lateral, curvando para a
  região temporal (orientação conforme a pálpebra alvo).
\item
  Manobra chave: cantotomia + cantólise do braço correspondente do
  tendão, permitindo rotação/avanço medial.
\item
  Força: preserva anatomia e evita oclusão visual; excelente em
  pacientes onde Hughes/Cutler-Beard não são opção.
\end{itemize}

\subsection{\texorpdfstring{2) Hughes (Tarsoconjuntival) ---
\textbf{Para Pálpebra
Inferior}}{2) Hughes (Tarsoconjuntival) --- Para Pálpebra Inferior}}\label{hughes-tarsoconjuntival-para-puxe1lpebra-inferior}

\begin{itemize}
\item
  Defeito: \textgreater50\% inferior, sobretudo central.
\item
  \textbf{1º tempo (posterior):} Criar retalho de \textbf{tarso +
  conjuntiva} da pálpebra superior, preservando a margem e cílios
  superiores. Fixar o tarso doador ao remanescente tarsal inferior para
  reconstruir a lamela posterior.
\item
  \textbf{Cobertura (anterior):} Cobrir com avanço local de pele/músculo
  ou enxerto cutâneo livre (conforme necessidade e laxidez).
\item
  2º tempo (3--6 semanas): Abrir a fenda palpebral seccionando o
  pedículo quando houver maturação vascular e estabilidade.
\end{itemize}

\subsection{\texorpdfstring{3) Cutler-Beard (Bridge Flap) ---
\textbf{Para Pálpebra
Superior}}{3) Cutler-Beard (Bridge Flap) --- Para Pálpebra Superior}}\label{cutler-beard-bridge-flap-para-puxe1lpebra-superior}

\begin{itemize}
\item
  Defeito: \textgreater50\% superior central (quando não há tarso
  superior suficiente para fechar).
\item
  \textbf{1º tempo:} Retalho ``em ponte'' da pálpebra inferior
  (pele/músculo/conjuntiva) passado superiormente para preencher o
  defeito.
\item
  \textbf{Ponto crítico (estrutura):} A pálpebra superior precisa de
  rigidez posterior → geralmente requer \textbf{enxerto de cartilagem}
  (conchal/auricular ou septal) como substituto tarsal.
\item
  2º tempo: separação do retalho após integração.
\end{itemize}

\subsection{\texorpdfstring{4) Mustardé (Cervicofacial Rotation) ---
\textbf{Grandes defeitos do
inferior}}{4) Mustardé (Cervicofacial Rotation) --- Grandes defeitos do inferior}}\label{mustarduxe9-cervicofacial-rotation-grandes-defeitos-do-inferior}

\begin{itemize}
\item
  Defeito: \textgreater75\% inferior ou defeitos extensos com perda
  significativa de lamela anterior.
\item
  Ideia: trazer grande volume de pele de bochecha/pescoço para
  reconstruir a lamela anterior.
\item
  Posterior (obrigatório): Substituir tarso/conjuntiva com
  \textbf{cartilagem + mucosa} (ou equivalentes), além de
  \textbf{ancoragens periostais altas} para combater a gravidade e
  prevenir ectrópio tardio.
\end{itemize}

\section{Erros comuns (e como
resgatar)}\label{erros-comuns-e-como-resgatar-22}

\begin{itemize}
\item
  Fechar direto um defeito grande (\textgreater30\%) ``porque dá'':
\item
  \emph{Consequência:} distorção, deiscência, canto arredondado,
  lagoftalmo.
\item
  \emph{Prevenção:} respeitar o algoritmo por \% e testar tensão antes
  do ponto definitivo.
\item
  \emph{Resgate:} revisão com retalho apropriado e/ou suporte cantal.
\item
  \textbf{Necrose no Cutler-Beard:}
\item
  \emph{Causa típica:} desrespeito ao suprimento (ponte muito estreita /
  agressão vascular).
\item
  \emph{Prevenção:} planejamento milimétrico e manipulação atraumática.
\item
  \emph{Resgate:} desbridamento e reconstrução alternativa (complexidade
  alta).
\item
  \textbf{Sequela na pálpebra doadora do Hughes:}
\item
  \emph{Consequência:} retração/instabilidade superior, sintomas de
  superfície ocular.
\item
  \emph{Prevenção:} dissecção correta e preservação funcional do doador.
\item
  \emph{Resgate:} liberação de retratores e, se necessário, enxerto
  espaçador.
\item
  \textbf{Ectrópio tardio após Mustardé:}
\item
  \emph{Causa:} peso do retalho + ancoragem insuficiente + contratura.
\item
  \emph{Prevenção:} fixações periostais altas, hipercorreção planejada e
  suporte cantal.
\item
  \emph{Resgate:} cantoplastia, revisão de ancoragens e, em alguns
  casos, suporte de terço médio.
\end{itemize}

\begin{quote}
\textbf{PÉROLA CLÍNICA}
\end{quote}

\begin{quote}
\mbox{}%
\subsection{Nota de Arte: A ausência de
cílios}\label{nota-de-arte-a-ausuxeancia-de-cuxedlios}
\end{quote}

\begin{quote}
Nenhum desses retalhos restaura cílios de forma natural (exceções com
enxertos compostos pilosos são raras e esteticamente imprevisíveis).
\end{quote}

\begin{quote}
\textbf{Gestão:} alinhar expectativa desde o pré-op. Camuflagem tardia
pode incluir maquiagem, dermopigmentação médica ou soluções protéticas
após maturação completa da cicatriz.
\end{quote}

\section{Pós-operatório}\label{puxf3s-operatuxf3rio-12}

\begin{itemize}
\item
  \textbf{Retalhos em 2 tempos:} oclusão temporária do olho é esperada.
  Orientar higiene cuidadosa, lubrificação intensa e sinais de
  infecção/coleção.
\item
  \textbf{Proteção corneana:} prioridade absoluta (lubrificantes,
  pomadas, escudo, medidas para evitar atrito na nova margem).
\item
  \textbf{2º estágio (3--6 semanas):} abrir a fenda quando o tecido
  estiver vascularizado e maleável. Deixar pequena margem mucosa
  adequada para evitar atrito/queratinização.
\item
  Contratura: lembrar que retração tardia existe; planejar com ``margem
  de segurança'' (tensão e tamanho de enxertos).
\end{itemize}

\section{Referências}\label{referuxeancias-21}

\begin{itemize}
\item
  Retalho semicircular de {[}{[}REF:TENZEL-1975{]}{]} Tenzel.
\item
  Retalho tarsoconjuntival de {[}{[}REF:HUGHES-1937{]}{]} Hughes.
\item
  Retalho em ponte de pálpebra superior de {[}{[}REF:CUTLER-1946{]}{]}
  Cutler \& Beard.
\item
  Retalho cérvico-facial de rotação para grandes defeitos do inferior
  ({[}{[}REF:MUSTARDE-1966{]}{]} Mustardé. * * *
\end{itemize}

\begin{center}\rule{0.5\linewidth}{0.5pt}\end{center}

\chapter{Capítulo 28 --- Complicações, revisões e gestão: prevenção,
resgate e
precificação}\label{capuxedtulo-28-complicauxe7uxf5es-revisuxf5es-e-gestuxe3o-prevenuxe7uxe3o-resgate-e-precificauxe7uxe3o}

\begin{figure}
\centering
\pandocbounded{\includegraphics[keepaspectratio,alt={Figura 28.1 --- Ilustração principal do capítulo}]{/Users/humbertolopes/Dev/work/marcelo-cury/the_art_of_eyelid_surgery_scaffold/projects/eyelid-surgery/assets/figures/FIG-28-01_complicacoes-resgate-precificacao.png}}
\caption{Figura 28.1 --- Ilustração principal do capítulo}
\end{figure}

\textbf{Parte:} Parte VIII --- Complicações e Carreira

\textbf{Objetivo do capítulo:} Ao final, o leitor saberá instituir
\textbf{protocolos de segurança para emergências} (ex.: hematoma
retrobulbar), diferenciar com precisão o momento de \textbf{observar
vs.~intervir} em complicações estéticas, e estabelecer uma política
clara de cobrança para revisões que proteja a relação médico-paciente e
a saúde financeira da prática.

\begin{quote}
\textbf{Nota de escopo:} este capítulo é voltado a cirurgiões treinados.
Protocolos emergenciais pressupõem estrutura, equipe e retaguarda
oftalmológica.
\end{quote}

\section{O que muda na decisão (o
``porquê'')}\label{o-que-muda-na-decisuxe3o-o-porquuxea-27}

\begin{itemize}
\item
  \textbf{A Regra da Presença:} A causa número 1 de escalada ético-legal
  não é a complicação em si --- é a sensação de abandono. O cirurgião
  que \textbf{aparece, examina, documenta e agenda retorno} reduz
  drasticamente conflito. O que some cria vácuo: o paciente preenche com
  medo, internet e advogado.
\item
  \textbf{Tempo é tecido (e às vezes é visão):} Algumas decisões são
  \textbf{em minutos} (compressão orbitária). Outras são \textbf{em
  meses} (cicatriz, retração, assimetria fina). O erro clássico é
  inverter: esperar o que não pode e operar o que ainda vai melhorar.
\item
  \textbf{Revisão é um produto (precisa de regra):} ``Revisão totalmente
  gratuita'' vira ímã de queixas infinitas e destrói margem. ``Cobrar
  tudo'' destrói confiança. O modelo sustentável é: \textbf{honorários
  médicos isentos em janela definida}, com \textbf{custos de centro
  cirúrgico/anestesia/materiais} cobrados, e uma política explícita para
  \emph{casos funcionais}.
\end{itemize}

\section{Indicações e
contra-indicações}\label{indicauxe7uxf5es-e-contra-indicauxe7uxf5es-17}

\textbf{Indicar revisão cirúrgica quando:} * Comprometimento funcional:
lagoftalmo significativo, ectrópio, entrópio, triquíase corneana, ptose
obstrutiva.

\begin{itemize}
\item
  Assimetria \textbf{socialmente óbvia} (≈ \textgreater2 mm) persistente
  após fase inflamatória.
\item
  Nódulos/granulomas persistentes após conduta conservadora adequada.
\item
  Cicatriz madura (regra prática: \textbf{\textgreater{} 6 meses}, salvo
  emergência funcional).
\end{itemize}

\textbf{Evitar / adiar quando:} * \textbf{Fase inflamatória ativa:}
\textasciitilde3 semanas a 3 meses (tecido hiperêmico, friável, sangra e
fibrosa mais).

\begin{itemize}
\item
  Queixa ``milimétrica'' com hiperfoco (paquímetro/espelho de
  aumento/ritual diário de checagem).
\item
  Olho seco descompensado: trate superfície ocular primeiro.
\end{itemize}

\begin{quote}
\textbf{PÉROLA CLÍNICA}
\end{quote}

\begin{quote}
\mbox{}%
\subsection{Checklist de ``Crise'' (Kit de
Emergência)}\label{checklist-de-crise-kit-de-emerguxeancia}
\end{quote}

\begin{quote}
Este kit deve estar \textbf{na sala}, acessível, não ``no
almoxarifado'':
\end{quote}

\begin{quote}
\begin{itemize}
\tightlist
\item[$\square$]
  Tesoura ponta romba + lâmina 15 (remoção rápida de suturas /
  reabertura).
\end{itemize}
\end{quote}

\begin{quote}
\begin{itemize}
\tightlist
\item[$\square$]
  Retratores simples + afastador adequado para canto.
\end{itemize}
\end{quote}

\begin{quote}
\begin{itemize}
\tightlist
\item[$\square$]
  Hialuronidase (múltiplas ampolas) para edema iatrogênico / preenchedor
  prévio.
\end{itemize}
\end{quote}

\begin{quote}
\begin{itemize}
\tightlist
\item[$\square$]
  Colírio hipotensor (Timolol/Brimonidina) + lubrificantes.
\end{itemize}
\end{quote}

\begin{quote}
\begin{itemize}
\tightlist
\item[$\square$]
  Curativo compressivo + gelo + escudo ocular.
\end{itemize}
\end{quote}

\begin{quote}
\begin{itemize}
\tightlist
\item[$\square$]
  Termo específico de revisão (``melhoria'', não ``perfeição'') +
  câmera/fotos padronizadas.
\end{itemize}
\end{quote}

\begin{quote}
\begin{itemize}
\tightlist
\item[$\square$]
  Contato direto do oftalmologista (córnea/retina) e do centro
  cirúrgico.
\end{itemize}
\end{quote}

\section{Anatomia aplicada (O terreno hostil da
revisão)}\label{anatomia-aplicada-o-terreno-hostil-da-revisuxe3o}

\begin{itemize}
\item
  \textbf{Planos fundidos:} em revisão, o ``plano fácil'' não existe.
  Septo, retratores, orbicular e gordura podem estar colados. A
  dissecção é sob magnificação e com mentalidade de ``estrutura nobre
  deslocada''.
\item
  \textbf{Vascularização anárquica:} cicatriz sangra difusamente
  (\emph{oozing}). Hemostasia vira estratégia (compressão + bipolar
  pontual + paciência), não ``caça ao vaso''.
\item
  \textbf{Escassez de pele é a regra:} em ectrópio/lagoftalmo
  pós-subtração, assuma ``falta pele'' até prova em contrário. Entre com
  plano de enxerto (pálpebra superior / retroauricular) decidido antes
  da incisão.
\end{itemize}

\begin{quote}
\textbf{📎 FIGURA NECESSÁRIA (Cap. 28):}
\end{quote}

\begin{quote}
Fluxograma: Prevenção → Reconhecimento → Resgate
\end{quote}

\begin{quote}
\emph{Estilo: Diagrama técnico-didático, cores neutras, legendas claras}
\end{quote}

\begin{quote}
FIG-28B --- ``triângulo de revisão'': Função, Estética, Superfície
ocular.
\end{quote}

\section{Técnica de Gestão (Protocolo dos 3
A's)}\label{tuxe9cnica-de-gestuxe3o-protocolo-dos-3-as}

\begin{enumerate}
\def\labelenumi{\arabic{enumi}.}
\item
  \textbf{Acknowledge (Reconhecer):} Não negue a experiência do
  paciente. Use frases objetivas: ``Eu entendi a queixa. Vamos examinar
  e medir.''
\item
  \textbf{Analyze (Analisar):} Examine, meça, fotografe, compare com
  pré-op. Classifique: edema, hematoma, cicatriz, malposição, olho seco,
  infecção, nódulo.
\item
  \textbf{Act (Agir):} Plano com \textbf{datas} e \textbf{critérios de
  troca} (``se X não melhorar em Y semanas, faremos Z''). O paciente
  tolera incerteza biológica quando existe roteiro.
\end{enumerate}

\section{A Linha do Tempo (Regra prática de
decisão)}\label{a-linha-do-tempo-regra-pruxe1tica-de-decisuxe3o}

\begin{itemize}
\item
  \textbf{0--60 minutos:} emergências que ameaçam visão/tecido
  (compressão orbitária, isquemia).
\item
  \textbf{1--48 horas:} hematoma significativo, infecção precoce, dor
  desproporcional, piora progressiva.
\item
  \textbf{1--3 semanas:} quemose, irregularidades inflamatórias,
  assimetria por edema, cicatriz ``viva''.
\item
  3--6 meses: decisões de resgate estético funcional (retração
  persistente, ptose, contorno).
\item
  \begin{quote}
  6 meses: revisão formal eletiva (cicatriz madura, planos
  estabilizados).
  \end{quote}
\end{itemize}

\begin{center}\rule{0.5\linewidth}{0.5pt}\end{center}

\section{Manejo de complicações específicas (o
``menu'')}\label{manejo-de-complicauxe7uxf5es-especuxedficas-o-menu}

\subsection{1) Hematoma retrobulbar (Emergência
real)}\label{hematoma-retrobulbar-emerguxeancia-real}

\begin{itemize}
\item
  \textbf{Sinais de alerta:} dor intensa desproporcional, proptose,
  pálpebra ``tensa'', náusea/agitação, piora rápida de visão, pupila
  alterada (tardio).
\item
  \textbf{Princípio:} \textbf{descomprimir antes de investigar}. Exame
  de imagem não é prioridade se há sinais clínicos de compressão.
\item
  Conduta de alto nível: reabrir para drenagem/descompressão e acionar
  retaguarda oftalmológica imediatamente. Técnicas de descompressão
  cantal são recursos de urgência em mãos treinadas.
\end{itemize}

\begin{quote}
\textbf{Erro Nota 7:} ``Vamos observar e ver se reabsorve.''
\end{quote}

\begin{quote}
Em compressão orbitária, observar é perder tempo.
\end{quote}

\subsection{2) Isquemia cutânea / necrose de
retalho}\label{isquemia-cutuxe2nea-necrose-de-retalho}

\begin{itemize}
\item
  \textbf{Causa típica:} excesso de tensão, comprometimento vascular,
  tabagismo, dissecção agressiva.
\item
  \textbf{Conduta:} aliviar tensão, otimizar perfusão local, antibiótico
  se indicado, desbridamento no tempo correto (nem cedo demais, nem
  tarde demais), e reconstrução em segundo tempo.
\end{itemize}

\subsection{3) Lagoftalmo e olho seco
pós-op}\label{lagoftalmo-e-olho-seco-puxf3s-op}

\begin{itemize}
\item
  \textbf{Imediato:} lubrificação intensiva, oclusão noturna, proteção
  corneana.
\item
  \textbf{Persistente (\textgreater3--6 meses):} avaliar falta de
  pele/lamela anterior, retração lamelar média/posterior e necessidade
  de enxerto/liberação (ver Cap. 26).
\end{itemize}

\subsection{4) Ectrópio/retração inferior
pós-blefaroplastia}\label{ectruxf3pioretrauxe7uxe3o-inferior-puxf3s-blefaroplastia}

\begin{itemize}
\item
  \textbf{Diagnóstico funcional:} snap-back, distraction, teste de
  elevação da bochecha, boca aberta (ver Cap. 26).
\item
  Princípio: ``alongar o curto + tencionar o frouxo''. Muitas vezes
  envolve cantoplastia/strip + liberação + enxerto.
\end{itemize}

\subsection{5) Assimetria (sulco, contorno, show
escleral)}\label{assimetria-sulco-contorno-show-escleral}

\begin{itemize}
\item
  \textbf{Até 3 meses:} presuma edema e assimetria de cicatrização;
  documente e siga protocolo.
\item
  \textbf{Após 6 meses:} classifique causa (ptose, ressecção
  assimétrica, cicatriz) e escolha resgate (lipoenxertia, touch-up
  conservador, correção funcional).
\end{itemize}

\subsection{6) Nódulos, granulomas e irregularidades (gordura /
enxertos)}\label{nuxf3dulos-granulomas-e-irregularidades-gordura-enxertos}

\begin{itemize}
\item
  \textbf{Primeiro:} diferenciar nódulo inflamatório vs.~óleo/necrose
  gordurosa vs.~cicatriz.
\item
  Conduta: escalonar conservador → procedimentos focais. Evitar
  ``soluções agressivas'' em pálpebra fina.
\end{itemize}

\begin{center}\rule{0.5\linewidth}{0.5pt}\end{center}

\section{Revisões: política de cobrança (modelo
sustentável)}\label{revisuxf5es-poluxedtica-de-cobranuxe7a-modelo-sustentuxe1vel}

\subsection{1) Defina antes da cirurgia (não na
crise)}\label{defina-antes-da-cirurgia-nuxe3o-na-crise}

A regra precisa estar no consentimento e ser verbalizada:

\textbf{Modelo recomendado (prático e justo):} * \textbf{Dentro de 12
meses:} isenção de honorários médicos para revisões necessárias,
\textbf{com cobrança de custos} (anestesia, sala, materiais, taxas).

\begin{itemize}
\item
  Fora de 12 meses: honorários reduzidos ou tabela específica de
  revisão.
\item
  Casos funcionais/urgências: política diferenciada (prioridade clínica;
  cobrança transparente de custos).
\end{itemize}

\subsection{2) Classifique a revisão (para não virar
refém)}\label{classifique-a-revisuxe3o-para-nuxe3o-virar-refuxe9m}

\begin{itemize}
\item
  \textbf{Biologia previsível:} cicatriz/edema/assimetria leve →
  acompanhamento e eventual ajuste.
\item
  \textbf{Complicação técnica real:} correção com postura de
  responsabilidade.
\item
  \textbf{Expectativa/percepção (perfil dismórfico):} protocolo de
  comunicação + limites claros; evitar ``revisões em cascata''.
\end{itemize}

\begin{quote}
\textbf{PÉROLA CLÍNICA}
\end{quote}

\begin{quote}
\mbox{}%
\subsection{Regra Prática: ``Revisão não é reembolso do
valor''}\label{regra-pruxe1tica-revisuxe3o-nuxe3o-uxe9-reembolso-do-valor}
\end{quote}

\begin{quote}
Revisão é um \textbf{novo ato} (tempo, risco, equipe, estrutura).
\end{quote}

\begin{quote}
O que protege a relação é: \textbf{ser generoso no cuidado} e
\textbf{ser claro no contrato}.
\end{quote}

\begin{quote}
Frase padrão (sem arrogância):
\end{quote}

\begin{quote}
``Meu compromisso é te acompanhar e corrigir o que for necessário. Em
revisões dentro da janela combinada, eu não cobro meus honorários, mas
existem custos de centro cirúrgico e anestesia que são do ato.''
\end{quote}

\section{Erros comuns (e como
resgatar)}\label{erros-comuns-e-como-resgatar-23}

\begin{itemize}
\item
  \textbf{Postura defensiva imediata}
\item
  \emph{Consequência:} quebra de aliança terapêutica e escalada do
  conflito.
\item
  \emph{Prevenção:} linguagem objetiva + plano datado.
\item
  \emph{Resgate:} reaproximação ativa: retornos curtos, fotos seriadas,
  explicação de timeline.
\item
  \textbf{Operar cicatriz imatura}
\item
  \emph{Consequência:} sangramento, pior fibrose, resultado pior que o
  original.
\item
  \emph{Prevenção:} respeitar janela de maturação (≈6 meses) salvo
  urgência funcional.
\item
  \emph{Resgate:} tratamento conservador prolongado + revisão no tempo
  certo.
\item
  \textbf{Corticoide intralesional ``às cegas''}
\item
  \emph{Consequência:} atrofia, hipopigmentação, ``dent'' palpebral.
\item
  \emph{Prevenção:} micro-doses, diluição, plano correto e indicação
  precisa.
\item
  \emph{Resgate:} manejo tardio com enxertia/volume conforme
  necessidade.
\item
  \textbf{Prometer ``perfeição'' como meta}
\item
  \emph{Consequência:} qualquer imperfeição vira ``falha''.
\item
  \emph{Prevenção:} enquadrar objetivo como \textbf{melhoria +
  segurança}.
\item
  \emph{Resgate:} renegociar expectativa com fotos, medidas e critérios
  realistas.
\end{itemize}

\section{Notas de ``arte'' (A conversa
difícil)}\label{notas-de-arte-a-conversa-difuxedcil}

Complicação bem gerida pode fortalecer reputação --- não por
``carinho'', mas por \textbf{controle}:

\begin{itemize}
\item
  presença,
\item
  documentação,
\item
  roteiro,
\item
  retaguarda,
\item
  limites.
\end{itemize}

O paciente perdoa biologia. Não perdoa abandono.

\section{Referências}\label{referuxeancias-22}

\begin{itemize}
\item
  Protocolos de emergência para hematoma retrobulbar
  ({[}{[}REF:HASS-2004{]}{]} Hass).
\item
  Gestão psicológica do paciente insatisfeito em cirurgia plástica
  ({[}{[}REF:GORNEY-1999{]}{]} Gorney).
\item
  Algoritmos para correção de ectrópio pós-blefaroplastia
  ({[}{[}REF:MCCORD-1995{]}{]} McCord / Fagien). * * *
\end{itemize}

\begin{center}\rule{0.5\linewidth}{0.5pt}\end{center}

\chapter{Bibliografia Mestre}\label{bibliografia-mestre}

\begin{quote}
Referências organizadas por ID curto. Formato: Vancouver simplificado.
\end{quote}

\begin{quote}
Use \texttt{{[}{[}REF:ID{]}{]}} nos capítulos para referenciar.
\end{quote}

\begin{quote}
\textbf{Nota clínica:} Esta lista inclui tanto referências citadas no
texto quanto obras de
\end{quote}

\begin{quote}
consulta/leitura recomendada. IDs não citados diretamente servem como
base
\end{quote}

\begin{quote}
teórica complementar.
\end{quote}

\begin{center}\rule{0.5\linewidth}{0.5pt}\end{center}

\section{Anatomia \& Envelhecimento}\label{anatomia-envelhecimento}

\begin{itemize}
\item
  \textbf{{[}MENDELSON-2008{]}} Mendelson BC, Muzaffar AR, Adams WP
  Jr.~Surgical anatomy of the midcheek and malar mounds. Plast Reconstr
  Surg. 2002;110(3):885-96.
\item
  \textbf{{[}ROHRICH-2008{]}} Rohrich RJ, Pessa JE. The fat compartments
  of the face: anatomy and clinical implications for cosmetic surgery.
  Plast Reconstr Surg. 2007;119(7):2219-27.
\item
  \textbf{{[}PESSA-2008{]}} Pessa JE, Rohrich RJ. Discussion: aging of
  the facial skeleton. Plast Reconstr Surg. 2008;121(6):2239-40.
\item
  \textbf{{[}ZIDE-1985{]}} Zide BM, Jelks GW. Surgical anatomy of the
  orbit. Raven Press; 1985.
\item
  \textbf{{[}KNIZE-2001{]}} Knize DM. An anatomically based study of the
  mechanism of eyebrow ptosis. Plast Reconstr Surg. 1996;97(7):1321-33.
\item
  \textbf{{[}LAMBROS-2007{]}} Lambros V. Observations on periorbital and
  midface aging. Plast Reconstr Surg. 2007;120(5):1367-76.
\end{itemize}

\begin{center}\rule{0.5\linewidth}{0.5pt}\end{center}

\section{Exame \& Vetores}\label{exame-vetores}

\begin{itemize}
\item
  \textbf{{[}JELKS-1993{]}} Jelks GW, Jelks EB. Preoperative evaluation
  of the blepharoplasty patient: bypassing the pitfalls. Clin Plast
  Surg. 1993;20(2):213-23.
\item
  \textbf{{[}FLOWERS-1993{]}} Flowers RS. Upper blepharoplasty by eyelid
  invagination. Clin Plast Surg. 1993;20(2):193-207.
\item
  \textbf{{[}BODIAN-1982{]}} Bodian M. Blepharoptosis and Hering's law.
  Arch Ophthalmol. 1982;100(4):543-4.
\end{itemize}

\begin{center}\rule{0.5\linewidth}{0.5pt}\end{center}

\section{Consulta \& Expectativas}\label{consulta-expectativas}

\begin{itemize}
\item
  \textbf{{[}SARWER-2006{]}} Sarwer DB, Crerand CE. Body image and
  cosmetic medical treatments. Body Image. 2004;1(1):99-111.
\item
  \textbf{{[}GORNEY-1999{]}} Gorney M. Recognition and management of the
  patient unsuitable for aesthetic surgery. Plast Reconstr Surg.
  1999;104(1):327-31.
\end{itemize}

\begin{center}\rule{0.5\linewidth}{0.5pt}\end{center}

\section{Blefaroplastia Superior}\label{blefaroplastia-superior}

\begin{itemize}
\item
  \textbf{{[}FAGIEN-1999{]}} Fagien S. Algorithm for blepharoplasty.
  Plast Reconstr Surg. 1999;104(4):1093-100.
\item
  \textbf{{[}REES-1984{]}} Rees TD. Blepharoplasty and facialplasty. In:
  Rees TD, ed.~Aesthetic Plastic Surgery. WB Saunders; 1984.
\end{itemize}

\begin{center}\rule{0.5\linewidth}{0.5pt}\end{center}

\section{Blefaroplastia Inferior \&
Gordura}\label{blefaroplastia-inferior-gordura}

\begin{itemize}
\item
  \textbf{{[}HAMRA-1995{]}} Hamra ST. The role of orbital fat
  preservation in facial aesthetic surgery: a new concept. Clin Plast
  Surg. 1996;23(4):677-88.
\item
  \textbf{{[}GOLDBERG-1998{]}} Goldberg RA. Transconjunctival orbital
  fat repositioning: transposition of orbital fat pedicles into a
  subperiosteal pocket. Plast Reconstr Surg. 2000;105(2):743-8.
\item
  \textbf{{[}HIDALGO-2011{]}} Hidalgo DA. An integrated approach to
  lower blepharoplasty. Plast Reconstr Surg. 2011;127(1):386-95.
\end{itemize}

\begin{center}\rule{0.5\linewidth}{0.5pt}\end{center}

\section{Cantopexia \& Cantoplastia}\label{cantopexia-cantoplastia}

\begin{itemize}
\item
  \textbf{{[}ANDERSON-1979{]}} Anderson RL, Gordy DD. The tarsal strip
  procedure. Arch Ophthalmol. 1979;97(11):2192-6.
\item
  \textbf{{[}MLADICK-1979{]}} Mladick RA. Muscle suspension orbicularis
  oculi lift. Plast Reconstr Surg. 1993;91(3):406-15.
\item
  \textbf{{[}MCCORD-1995{]}} McCord CD, Codner MA. Eyelid and
  Periorbital Surgery. Quality Medical Publishing; 2008.
\end{itemize}

\begin{center}\rule{0.5\linewidth}{0.5pt}\end{center}

\section{Brow Lift \& Terço Superior}\label{brow-lift-teruxe7o-superior}

\begin{itemize}
\item
  \textbf{{[}CONNELL-1978{]}} Connell BF. Eyebrow, face, and neck lifts
  for males. Clin Plast Surg. 1978;5(1):15-28.
\item
  \textbf{{[}CASTANARES-1964{]}} Castañares S. Forehead wrinkles,
  glabellar frown and ptosis of the eyebrows. Plast Reconstr Surg.
  1964;34:406-13.
\end{itemize}

\begin{center}\rule{0.5\linewidth}{0.5pt}\end{center}

\section{Ptose Palpebral}\label{ptose-palpebral}

\begin{itemize}
\tightlist
\item
  \textbf{{[}PUTTERMAN-1975{]}} Putterman AM, Urist MJ. Müller
  muscle-conjunctival resection: technique for treatment of
  blepharoptosis. Arch Ophthalmol. 1975;93(8):619-23.
\end{itemize}

\begin{center}\rule{0.5\linewidth}{0.5pt}\end{center}

\section{Festoons \& Edema Malar}\label{festoons-edema-malar}

\begin{itemize}
\item
  \textbf{{[}PERRY-2013{]}} Perry JD. Treatment of malar bags (malar
  festoons). Facial Plast Surg Clin North Am. 2016;24(3):351-5.
\item
  \textbf{{[}KPODZO-2014{]}} Kpodzo DS, et al.~Algorithm for treatment
  of malar mounds and festoons. Plast Reconstr Surg.
  2014;134(3):490e-500e.
\end{itemize}

\begin{center}\rule{0.5\linewidth}{0.5pt}\end{center}

\section{Lipoenxertia}\label{lipoenxertia}

\begin{itemize}
\item
  \textbf{{[}COLEMAN-1997{]}} Coleman SR. Facial recontouring with
  lipostructure. Clin Plast Surg. 1997;24(2):347-67.
\item
  \textbf{{[}TONNARD-2013{]}} Tonnard P, Verpaele A, Peeters G, et
  al.~Nanofat grafting: basic research and clinical applications. Plast
  Reconstr Surg. 2013;132(4):1017-26.
\item
  \textbf{{[}MARTEN-2008{]}} Marten TJ, Elyassnia D. Fat grafting in
  facial rejuvenation. Clin Plast Surg. 2015;42(2):219-52.
\item
  \textbf{{[}COHEN-2017{]}} Cohen SR, et al.~Fat grafting strategies to
  rejuvenate the periorbital area. Clin Plast Surg. 2017;44(1):147-56.
\end{itemize}

\begin{center}\rule{0.5\linewidth}{0.5pt}\end{center}

\section{Funcional \& Reconstrução}\label{funcional-reconstruuxe7uxe3o}

\begin{itemize}
\item
  \textbf{{[}COLLIN-1983{]}} Collin JR. A Manual of Systematic Eyelid
  Surgery. Churchill Livingstone; 1983.
\item
  \textbf{{[}WEISS-1979{]}} Weiss JS, et al.~Involutional entropion: a
  unified approach. Trans Am Acad Ophthalmol Otolaryngol.
  1979;86(6):1018-24.
\item
  \textbf{{[}TENZEL-1975{]}} Tenzel RR. Reconstruction of the central
  one-half of an eyelid. Arch Ophthalmol. 1975;93(2):125-6.
\item
  \textbf{{[}HUGHES-1937{]}} Hughes WL. A new method for rebuilding a
  lower lid. Arch Ophthalmol. 1937;17:1008-17.
\item
  \textbf{{[}CUTLER-1946{]}} Cutler NL, Beard C. A method for partial
  and total upper lid reconstruction. Am J Ophthalmol. 1955;39(1):1-7.
\item
  \textbf{{[}MUSTARDE-1966{]}} Mustardé JC. Repair and Reconstruction in
  the Orbital Region. Churchill Livingstone; 1966.
\end{itemize}

\begin{center}\rule{0.5\linewidth}{0.5pt}\end{center}

\section{Complicações \&
Emergências}\label{complicauxe7uxf5es-emerguxeancias}

\begin{itemize}
\tightlist
\item
  \textbf{{[}HASS-2004{]}} Hass AN, Penne RB, Stefanyszyn MA, Flanagan
  JC. Incidence of postblepharoplasty orbital hemorrhage and associated
  visual loss. Ophthal Plast Reconstr Surg. 2004;20(6):426-32.
\end{itemize}

\begin{center}\rule{0.5\linewidth}{0.5pt}\end{center}

\section{Fotografia \&
Documentação}\label{fotografia-documentauxe7uxe3o}

\begin{itemize}
\item
  \textbf{{[}GUNTER-2007{]}} Gunter JP, Rohrich RJ, Adams WP Jr.~Dallas
  Rhinoplasty: Nasal Surgery by the Masters. Quality Medical Publishing;
  2007. (Protocolos de fotografia adaptados.)
\item
  \textbf{{[}HIRMAND-2010{]}} Hirmand H. Anatomy and nonsurgical
  correction of the tear trough deformity. Plast Reconstr Surg.
  2010;125(2):699-708.
\end{itemize}

\begin{center}\rule{0.5\linewidth}{0.5pt}\end{center}

\section{Anestesia \& Segurança}\label{anestesia-seguranuxe7a}

\begin{itemize}
\tightlist
\item
  \textbf{{[}MOST-2007{]}} Most SP. Analysis of outcomes following
  revision blepharoplasty. Arch Facial Plast Surg. 2008;10(4):226-30.
\end{itemize}

\begin{center}\rule{0.5\linewidth}{0.5pt}\end{center}

\section{Pinch \& Skin}\label{pinch-skin}

\begin{itemize}
\item
  \textbf{{[}PARK-2008{]}} Park DM. Skin-only lower lid blepharoplasty
  as an alternative to skin-muscle flap. Arch Facial Plast Surg.
  2008;10(6):411-6.
\item
  \textbf{{[}RAMIREZ-2000{]}} Ramirez OM. The central oval of the face:
  tridimensional endoscopic rejuvenation. Facial Plast Surg.
  2000;16(3):283-98.
\end{itemize}

\begin{center}\rule{0.5\linewidth}{0.5pt}\end{center}

\section{Massry (Midface/Festoons)}\label{massry-midfacefestoons}

\begin{itemize}
\tightlist
\item
  \textbf{{[}MASSRY-2012{]}} Massry GG. The inverse relationship of
  midface and lower eyelid/cheek complex. Ophthal Plast Reconstr Surg.
  2012;28(6):431-8.
\end{itemize}

\begin{center}\rule{0.5\linewidth}{0.5pt}\end{center}

\begin{quote}
\textbf{Nota clínica:} Anos e números de páginas devem ser
verificados/completados conforme a edição consultada.
\end{quote}

\begin{quote}
Adicionar novas referências seguindo o padrão \texttt{{[}AUTOR-ANO{]}}.
\end{quote}

\begin{center}\rule{0.5\linewidth}{0.5pt}\end{center}

\chapter{Fim do Manuscrito}\label{fim-do-manuscrito}

\begin{quote}
Total: 30 capítulos \textbar{} 0 faltando
\end{quote}

\backmatter
\end{document}
